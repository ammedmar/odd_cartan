% !TEX root = ../odd_cartan.tex

\section{Effective constructions}

In this section we construct explicit Cartan relator for algebras over the Barratt--Eccles operad, of which the cochains of simplicial sets are natural examples.

\subsection{Barratt--Eccles operad}

We review the main constructions introduced by Berger--Fresse in \cite{berger2004combinatorial}.

Let $\Set$ and $\sSet$ denote the categories of sets and simplicial sets respectively.
The symmetric monoidal functor $\EE \colon \Set \to \sSet$ is defined on objects by $\EE X_n = X^{n+1}$ with
\[
\begin{split}
	\face_i(x_0,\dots,x_n) &= (x_0,\dots,\widehat{x}_i,\dots,x_n), \\
	\dege_i(x_0,\dots,x_n) &= (x_0, \dots, x_i, x_i, \dots, x_n),
\end{split}
\]
and on morphisms by $\EE f_n = f^{\times n}$.
We notice that if $X$ is equipped with a group action then $\EE X$ is as well with
\[
(x_0,\dots,x_n) \cdot g = (x_0 \cdot g, \dots, x_n \cdot g).
\]

The usual block permutation map
\[
\bcirc_{\sym} \colon \sym_r \times \sym_{s_1} \times \cdots \times \sym_{s_r} \to \sym_{s_1+\dots+s_r}
\]
defined for any $r,s_1,\dots,s_r \in \N$ provides $\sym = \{\sym_r\}_{r>0}$ with the structure of an operad in $\Set$.

The simplicial Barratt--Eccles operad $\rE\sym$ is obtained by applying the functor $\EE$ to the operad $\sym$.
We denote its composition by
\[
\bcirc_{\rE\sym} \colon \rE\sym(r) \times \rE\sym(s_1) \times\dots\times \rE\sym(s_r) \to \rE\sym(s_1+\dots+s_r)
\]
for any $r,s_1,\dots,s_r \in \N$.

Let $\chains \colon \sSet \to \Ch$ be the functor of (normalized) chains with integer coefficients.
The linear Barratt--Eccles operad $\cE$ is defined by $\cE(r) = \chains\EE(r)$ and
\[
\bcirc_{\cE} \defeq \bcirc_{\EE} \circ \EZ \colon \cE(r) \ot \cE(s_1) \ot\dotsb\ot \cE(s_r) \to \cE(s_1+\dots+s_r)
\]
for any $r,s_1,\dots,s_r \in \N$.

For any simplicial set $X$, Berger and Fresse explicitly constructed a natural operad morphism
\[
\phi \colon \cE \to \End(\cochains(X))
\]
on its (normalized) cochains.
Since only the existence of this effective construction will be used here, we refer to the original source for details, and to the specialized computer algebra system \texttt{ComCH} for an implementation.

\subsection{May--Steenrod structure on Barratt--Eccles algebras}

To provide the cochains of simplicial sets, or more generally any $\cE$-algebra, with a natural May--Steenrod structure, it suffices to define a $\cyc$-module quasi-isomorphism
\[
\iota \colon \cW \to \cE.
\]
We recall from \cite{medina2021may_st} a closed form formula for one such map.
For $r,n\in\N$, let
\begin{equation*}
	\iota^{r}(e_{n}) = \begin{cases}
		\displaystyle{\sum_{r_1, \dots, r_m}} \big(\rho^0, \rho^{r_1}, \rho^{r_1+1}, \rho^{r_2}, \dots, \rho^{r_{m}}, \rho^{r_{m}+1} \big) & n = 2m, \\
		\displaystyle{\sum_{r_1, \dots, r_m}} \big(\rho^0, \rho^1, \rho^{r_1}, \rho^{r_1+1}, \dots, \rho^{r_{m}}, \rho^{r_{m}+1} \big) & n = 2m+1,
	\end{cases}
\end{equation*}
where the sum is over all $r_1, \dots, r_m \in \{0, \dots, r-1\}$.
Notice that $\iota$ factors through the $\cyc$-module $\{\chains\EE\cyc_r\}_{r\in\N}$.
Please consult \cref{f:small values of psi} for a few examples.

\begin{table}
	\centering
	\resizebox{0.8\columnwidth}{!}{%
\renewcommand{\arraystretch}{1.2}
\begin{tabular}{|c||c|c|c|}
	\hline
	$r$ & $n=2$ & $n=3$ & $n=4$ \\
	\hline
	2 & (0,1,0) & (0,1,0,1) & (0,1,0,1,0) \\
	\hline
	3 & (0,1,2) + (0,2,0) & (0,1,2,0) + (0,1,0,1) & \phantom{+} (0,1,2,0,1) + (0,1,2,1,2) \\
	& & & + (0,2,0,1,2) + (0,2,0,2,0) \\
	\hline
	4 & (0,1,2) + (0,2,3) & (0,1,2,3) + (0,1,3,0) & \phantom{+} (0,1,2,3,0) + (0,1,2,0,1) \\
	& + (0,3,0) & + (0,1,0,1) &
	+ (0,1,2,1,2) + (0,2,3,0,1) \\
	& & & + (0,2,3,1,2) + (0,2,3,2,3) \\
	& & & + (0,3,0,1,2) + (0,3,0,2,3) \\
	& & & + (0,3,0,3,0) \\
	\hline
\end{tabular}
}
\vspace*{3pt}
	\caption{The elements $\psi^r(e_n)$ for small values of $r$ and $n$ where we are denoting $(\rho^{r_0}, \dots, \rho^{r_n})$ simply by $(r_0, \dots, r_n)$.}
	\label{f:small values of psi}
\end{table}

\subsection{The lowercase homomorphisms}

We now describe group homomorphisms inducing two chain homotopic $\cyc$-module chain maps.
We will later relate these to the $\cyc$-module chain maps $F_\psi$ and $G_\psi$ associated to the May--Steenrod structure on $\cE$-algebras.
For $r \in \N$, let
\[
\begin{split}
	&f^r \colon \cyc_r \to \sym_r \xra{\id \times e^{\times r}} \sym_r \times \sym_2^{\times r} \xra{\bcirc_{\sym}} \sym_{2r}, \\
	&g^r \colon \cyc_r \to \sym_r \xra{e \times \mathrm{diag}} \sym_2 \times \sym_r \times \sym_r \xra{\bcirc_{\sym}} \sym_{2r}.
\end{split}
\]
Explicitly, if $\rho = (1,2,\dots,r)$ in $\sym_r$ then in $\sym_{2r}$
\begin{align*}
	f^r(\rho) &= (1,3,\dots,2r-1)(2,4,\dots,2r), \\
	g^r(\rho) &= (1,2,\dots,r)(r+1,r+2,\dots,2r).
\end{align*}
We notice that
\begin{equation}\label{eq:conjugation of little maps}
	f^r(\rho) \tau = \tau g^r(\rho)
\end{equation}
where $\tau$ is as in \cref{ss:reordering}.
We obtain two $\cyc$-module chain maps $\chains\EE f$ and $\chains\EE g$ where the $\cyc_r$ action on $\chains\EE\sym_{2r}$ is induced from Homomorphism \eqref{eq:cyclic to 2-symetric}.

We define a $\cyc$-module chain homotopy $H_{\psi,1}$ between $\chains \EE f \cdot \tau$ and $\chains \EE g$
explicitly as follows.
For $r \in \N$, let $H_{\psi,1}^r \colon \chains\EE\cyc_r \to \chains\EE\sym_{2r}$ be given by
\[
H_{\psi,1}^r(\sigma_0,\dots,\sigma_n) =
\sum_{i=0}^n \ (-1)^i (f^r(\sigma_0) \tau, \dots, f^r(\sigma_i) \tau, g^r(\sigma_i), \dots, g^r(\sigma_n)).
\]
It can be easily verified that $H_{\psi,1}^r$ is a homotopy between $\chains \EE f^r \cdot \tau$ and $\chains \EE g^r$.
To check that it is $\cyc_r$-equivariant we use \cref{eq:conjugation of little maps} and the fact that $\rho$ acts on $\chains\EE\sym_{2r}$ multiplying on the right by $g(\rho)$.

\subsection{Comparing coproducts}

We now study the failure of the $\cyc$-module quasi-isomorphism $\iota$ to preserve coalgebra structures.
More precisely, we will construct a $\cyc$-module chain homotopy making the following diagram commute
\[
\begin{tikzcd}[column sep=large]
	\cW(r) \arrow[r,"\iota^r"] \arrow[d,"\Delta"] & \chains\EE\cyc_r \arrow[d,"\Delta_{\AW}"]\\
	\cW(r)^{\ot 2} \arrow[r,"(\iota^r)^{\ot 2}"] & \chains\EE\cyc_r^{\ \ot 2}.
\end{tikzcd}
\]
for every $r \in \N$.\todo{Please fill this out.}

Given $K$ as above, the $\cyc$-module map $H_{\psi,2}$ given for each $r \in \N$ by
\[
H_{\psi,2}^r \colon \cW(p) \xra{K^r} \cE(p)^{\ot 2} \xra{\psi^2_0 \ot \id} \cE(2) \ot \cE(r)^{\ot 2} \xra{\bcirc_{\cE}} \cE(2r)
\]
is easily seen to be a $\cyc$-module chain homotopy between $G_\psi$ and $\widetilde G_\psi$ given by
\[
\widetilde G_\psi^r \colon \cW(r) \xra{\psi(r)} \cE(r) \xra{\psi^2_0 \ot \Delta_{\AW}} \cE(2) \ot \cE(r)^{\ot 2} \xra{\bcirc_{\cE}} \cE(2r)
\]
for $r \in \N$.

\subsection{Lower and upper case maps}

It is immediate that
\[
\chains\EE f = F_\psi^r
\]
and that the composition
\[
H_{\psi,3}^r \colon \cW(r) \xra{\psi(r)} \cE(r) \xra{\psi^2_0 \ot \Delta_{\mathrm{SHI}}} \cE(2) \ot \cE(r)^{\ot 2} \xra{\bcirc_{\cE}} \cE(2r)
\]
is a $\cyc_r$-equivariant chain homotopy between $\chains\EE g$ and $\widetilde G_\psi^p$.

Let $H_{\psi,3}^r$ be the $\cyc$-module map defined by these maps, which is a chain homotopy

\subsection{The chain homotopy $K$}

\begin{definition} Let $C_*$ be a chain complex and $c\colon C_*\to C_*$ a homomorphism. A \emph{contraction} of $C_*$ to $c$ is a homotopy $h\colon C_*\to C_{*+1}$ between the identity and $c$, i.e.,
\[\partial h + h\partial = \id-c\] 
\end{definition}
\begin{lemma} Let $D_*$ be a $RG$-chain complex and $h\colon D_*\to D_{*+1}$ an $R$-contraction to some map $c\colon D_*\to D_*$. Let $f,g\colon C_*\to D_*$ be a pair of homotopic chain maps such that $c(f-g) = 0$. Suppose that $C_*$ is a free $RG$-complex with a choice of $RG$-basis $A_\bullet\subset C_*$. Then define recursively the $G$-equivariant map
\begin{align*}
K(a) &= h(f(a)-g(a)-K(\partial a)),
\end{align*} 
and extend linearly and equivariantly. Then $K$ is an $RG$-homotopy between $f$ and $g$.
\end{lemma}
\begin{proof} By construction it is linear and equivariant. We now check that it is a homotopy between $f$ and $g$. Let $a$ have positive degree. Then:
\begin{align*}
\partial K(a) + K\partial(a) 
&= \partial(h(f(a)-g(a)-K(\partial a))) + K\partial(a) \\
&= \partial(h(f(a)))-\partial(h(g(a)))-\partial(h(K(\partial a))) + K\partial(a) \\
&= -h(\partial(f(a))) + f(a) -c(f(a))+ h(\partial(g(a))) - g(a)+c(g(a))  + \\
&\quad + h(\partial(K(\partial(a)))) - K(\partial(a)) + K(\partial(a)) \\
&= f(a) - g(a) + h\left(-f(\partial a) + g(\partial a) + \partial K(\partial(a)) + K(\partial(\partial(a)))\right) \\
&= f(a) - g(a)
\end{align*}
\end{proof}
We can apply this lemma to find the required homotopy $K$. For that we consider the non-equivariant homotopy $h\colon N_*EC_r\otimes N_*EC_r\to EC_r\otimes N_*EC_r\otimes N_*EC_r$ given by $h((a_0,\ldots,a_j)\otimes(b_0,\ldots,b_k)) = (0,a_0,\ldots,a_j)\otimes(b_0,\ldots,b_k)$, which is a contraction to the endomorphism $c$ of $N_*EC_r\otimes N_*EC_r$ that sends $a\otimes b$ to $0\otimes b$ in degree $0$ and all other generators are sent to zero. Since $f(e_0) = g(e_0)$, it holds that $c(f-g) = 0$. We claim that the homotopy $K$ so defined as the following form: We assume $j,k>0$, and denote by $s_q^-$ the entry to the left of the entry $s_q$. Explicitely, we have 
\begin{align*}
    s_q^- &= \begin{cases}s_{q-1}+1 & \text{if $q>1$}\\
    s_q^- = 0 & \text{ if $q=1$ and even degree} \\
    s_q^- = 1 &\text{ if $q=1$ and odd degree.}
    \end{cases}
\end{align*} 
The condition $s_q^-<s_q<t_1<s_q^-$ makes sense with the cyclic order in $\{0,1,\ldots,p-1\}$. In the following sums, we are suming along all possible values of $s_1,\ldots,s_j,t_1,\ldots,t_k$ in $\{0,1,\ldots,p-1\}$.
\begin{align*}
K(e_{2i}) &= \sum_{j+k = i+1}{\sum_{q=1}^j{\sum_{s_q^-<s_q<t_1<s_q^-}{(0,s_1,s_1+1,\ldots,s_j,s_j+1)\otimes(t_1,t_1+1,\ldots,t_k,t_k+1)}}} \\ 
K(e_{2i+1}) &= -\sum_{j+k = i+1}{\sum_{q=1}^j{\sum_{s_q^-<s_q<t_1<s_q^-}{(0,1,s_1,s_1+1,\ldots,s_j,s_j+1)\otimes(t_1,t_1+1,\ldots,t_k,t_k+1)}}}
\end{align*}


Let us check these formulae inductively. 
\begin{align*}
K(e_{2i+1}) &= h(f(e_{2i+1})-g(e_{2i+1})-K(\partial(e_{2i+1})))
\end{align*}
Now, all the summands in $f(e_{2i+1})$ and $g(e_{2i+1})$ start with a zero, therefore we are left with
\[K(e_{2i+1}) = -h(K(\rho e_{2i}))+h(K(e_{2i}))\]
and the second summand is zero, while the first one agrees with the definition of $K(e_{2i+1})$.

\begin{align*}
K(e_{2i}) &= h(f(e_{2i})-g(e_{2i})-K(\partial(e_{2i})))
\end{align*}
Here, the summand $h(g(e_{2i}))$ is zero, the summand $K(\partial(e_{2i}))$ corresponds to the summation with $q>1$ and the summand $h(f(e_{2i}))$ corresponds to the summations with $q=1$.



\subsection{Main construction}

\begin{theorem}
	For any algebra over the Barratt--Eccles operad $\phi \colon \cE \to \End(A)$, the map $H_\psi = \phi \circ (H_{\psi,1} + H_{\psi,2} + H_{\psi,3})$ is a Cartan relator for $(A,\phi\circ\iota)$.\todo{Maybe the signs are not right here.}
\end{theorem}

\subsection{Examples}

We now show some examples computed using the specialized computer algebra system \texttt{ComCH} \cite{medina2021comch} when $A$ is given by the normalized cochains of a standard simplex, and $\phi$ is the $\cE$-algebra structure of Berger and Fresse.

\appendix
\section{Handicraft proof of the homotopy K}

In order to prove that this is a chain homotopy we will check that 
\[\partial(K(e_{i})) + K(\partial e_{i})-f(e_{i}) + g(e_{i}) = 0\]
\subsection*{Proof in odd degrees} Here are the values of these summands: Unless otherwise written, the first tensor factor of each summand has as last terms $s_j,s_j+1$ and the second tensor factor has as last terms $s_k,s_k+1$. The subscript $q$ is there for indexing purposes only.
\begin{align*}
-f(e_{2i+1}) &= -\sum_{j+k=i}\sum (0,s_1,s_1+1,\ldots)\otimes (0,1,t_1,t_1+1,\ldots) + 
\\
&+(0,1,s_1,s_1+1,\ldots)\otimes (1,2,t_1,t_1+1,\ldots) 
\\
g(e_{2i+1}) &= \sum_{j+k=i}\sum (0,1,s_1,s_1+1,\ldots,s_j,s_j+1)\otimes(s_j+1,t_1,t_1+1,\ldots) + 
\\
&+ (0,1,s_1,s_1+1,\ldots,t_1)\otimes(t_1,t_1+1,\ldots)
\\
K\partial(e_{2i+1}) &= \sum_{j+k=i}\sum_{q=1}^j\sum_{s_q^-<s_q<t_1<s_q^-} \left[(1,s_1,s_1+1,\ldots)\otimes(t_1,t_1+1,\ldots)_q\right. -
\\
&- \left.(0,s_1,s_1+1,\ldots)\otimes(t_1,t_1+1,\ldots)_q\right]
\\
\partial K(e_{2i+1}) &= -\partial\sum_{j+k=i}\sum_{q=1}^j\sum_{s_q^-<s_q<t_1<s_q^-}(0,1,s_1,s_1+1,\ldots)\otimes (t_1,t_1+1,\ldots)_q
\end{align*}
We will define an involution on the set of all summands, showing that each summand appears twice with opposite sign. The \emph{left caesuras} of a sequence is the first $i$ such that $s_i\neq 2i-1$. The \emph{right caesuras} of a sequence is the last $i$ such that $s_i\neq s_j-2(j-i)$. We refer to the caesuras of the first factor of a tensor product as the \emph{first caesuras} and to the caesuras of the second factor of the tensor product as the \emph{second caesuras}. The \emph{left follower} of an element $s_i$ of a sequence is the first entry to the left of that element that is not the predecessor of the element to its right. The \emph{right follower} of an element $s_{i}$ is the first entry to the right of that element that is not the succesor of the element to its left. Here are some examples:
\begin{itemize}
\item The right follower of $5$ in the sequence $(0,1,5,6,7,8,11,12,0,1)$ is $8$.
\item The left follower of $8$ in that sequence is $5$.
\end{itemize}
We treat first the summands in $f(e_{2i+1}), g(e_{2i+1})$ and $K(\partial(e_{2i+1}))$:
\begin{itemize}
\item The summand $(1,s_1,s_1+1,\ldots)\otimes(t_1,t_1+1,\ldots)_q$ of $K\partial(e_{2i+1})$ cancels with the summand
\[-(\widehat{0},1,s_1,s_1+1,\ldots)\otimes (t_1,t_1+1,\ldots)_q\]
of $\partial K(e_{2i+1})$.
\item The summand $A = -(0,s_1,s_1+1,\ldots)\otimes(t_1,t_1+1,\ldots)_q$ of $K\partial(e_{2i+1})$ cancels with the summand
\[+(0,1,\ldots,\widehat{2i-1},s_i,s_i+1,\ldots)\otimes (t_1,t_1+1,\ldots)_{q'}\]
of $\partial K(e_{2i+1})$, where $i$ is the first left caesuras of $A$ and $q'=q$ unless $q<i$ and $t_1=2q-2$, in which case $q'=q+1$.
\item The summand $A = -(0,s_1,s_1+1,\ldots)\otimes (0,1,t_1,t_1+1,\ldots)$ of $f(e_{2i+1})$ cancels with the summand
\[(0,1,\ldots,\widehat{2i-1},s_i,s_i+1,\ldots)\otimes (t_1,t_1+1,\ldots)_q\]
of $\partial K(e_{2i+1})$, where $i$ is the first left caesuras of $A$ and $q=1$ and $t_1=0$.
\item The summand $(0,1,s_1,s_1+1,\ldots)\otimes (1,2,t_1,t_1+1,\ldots)$ of $f(e_{2i+1})$ cancels with the summand
\[(0,1,s_1,s_1+1,\ldots)\otimes (\widehat{t}_1,t_1+1,\ldots)_q\]
of $\partial K(e_{2i+1})$, where $q=1$ and $t_1=0$.
\item The summand $A = (0,1,s_1,s_1+1,\ldots,s_j,s_j+1)\otimes(s_j+1,t_1,t_1+1,\ldots)$ of $g(e_{2i+1})$ cancels with the summand
\[(0,1,s_1,s_1+1,\ldots)\otimes (t_1,t_1+1,\ldots,t_{\ell-1},\widehat{t_{\ell-1}+1},t_\ell,t_\ell+1,\ldots)_q\]
of $\partial K(e_{2i+1})$, where $q=j$ and $t_1=s_j+1$ and $\ell$ is the second left caesuras of $A$. 
\item The summand $A = (0,1,s_1,s_1+1,\ldots,s_j,s_j+1,t_1)\otimes(t_1,t_1+1,\ldots)$ of $g(e_{2i+1})$ cancels with the summand
\[(0,1,s_1,s_1+1,\ldots,s_i,s_i+1,\widehat{s_{i+1}-1},s_{i+1},\ldots,s_j,s_j+1)\otimes (t_1,t_1+1,\ldots)_q\]
of $\partial K(e_{2i+1})$, where $q=j$ and $s_j+1=t_1$ and $i$ is the first right caesuras of $A$.
\end{itemize}
Now, the summands of $\partial K(e_{2i+1})$.
\begin{itemize}
\item The summand $-(\hat{0},1,s_1,s_1+1,\ldots)\otimes (t_1,t_1+1,\ldots)_q$ cancels with the summand
\[+(1,s_1,s_1+1,\ldots)\otimes(t_1,t_1+1,\ldots)_q\]
 of $K\partial(e_{2i+1})$.
\item The summand $+(0,1,\ldots,\widehat{2i-1},s_i,s_i+1,\ldots)\otimes (t_1,t_1+1,\ldots)_q$ with $i$ the first left caesuras cancels with the summand
\[(0,s_1,s_1+1,\ldots)\otimes(t_1,t_1+1,\ldots)_{q'}\]
of $K\partial(e_{2i+1})$, where $q'=q$ unless $t_1 = 2q-2$, in which case $q'=q-1$. This is well defined except for $q=1,t_1 = 0$ in which case it cancels with the summand $-(0,s_1,s_1+1,\ldots)\otimes (0,1,t_1,t_1+1,\ldots)$ of $f(e_{2i+1})$.
\item The summand $(0,1,\ldots,\widehat{s_{i-1}},\ldots,s_j,s_j+1)\otimes (t_1,t_1+1,\ldots)_q$ with $t_1=s_j+1$ and $i$ the first right caesuras, cancels with the summand
\[(0,1,s_1,s_1+1,\ldots,s_j,s_j+1,t_1)\otimes(t_1,t_1+1,\ldots) \]
of $g(e_{2i+1})$. 
\item The summand $(0,1,\ldots,\widehat{s}_r,s_r+1,\ldots,s_j,s_j+1)\otimes (t_1,t_1+1,\ldots)_q$ cancels with the summand
\[(0,1,\ldots,\widehat{s'_t+1},\ldots,s_j,s_j+1)\otimes (t_1,t_1+1,\ldots)_{q'}\]
of $K\partial(e_{2i+1})$, where the second summand is obtained from the first one by removing $s_r$ and adding the succesor of the right follower of $s_r$ to its right, which is denoted $s'_t+1$. The number $q'$ equals $q$ unless $t_1 = s_r+1$ and $q=r$, in which case it equals $q+1$. This is well-defined unless $t=j$, $q=j$ and $t_1=s_t+1$, in which case it cancels with a summand of the form $(0,1,\ldots,s_{j},s_{j}+1,t_1)\otimes (t_1,t_1+1,\ldots)$ in $f(e_{2i+1})$. 
\item The summand $(0,1,\ldots,s_t,\widehat{s}_t+1,\ldots,s_j,s_j+1)\otimes (t_1,t_1+1,\ldots)_q$ cancels with the summand
\[(0,1,\ldots,\widehat{s'_r},s'_r+1\ldots,s'_j,s'_j+1)\otimes (t_1,t_1+1,\ldots)_{q'}\]
of $K\partial(e_{2i+1})$, where the second summand is obtained from the first one by removing $s_t+1$ and adding the predecessor of the left follower of $s_t+1$ to its left, which is denoted $s'_r$. The number $q'$ equals $q$ unless $t_1 = s'_r+1$ and $q=r$, in which case it equals $q-1$. This is well-defined, because the exceptional case when $q=1$ is the case of the global caesuras.
\item The summand $(0,1,s_1,s_1+1\ldots)\otimes (t_1,t_1+1,\ldots,\hat{t}_r,t_r+1,\ldots)_q$ cancels with the summand
\[(0,1,s_1,s_1+1\ldots)\otimes (t_1,t_1+1,\ldots,t_s,\widehat{t_s+1},\ldots)_{q'}\]
of $K\partial(e_{2i+1})$, where the second summand is obtained by removing $t_r$ and adding the successor of the right follower to its right, which is denoted $t_{s}+1$. The value of $q'$ is $q$ unless $r=1$ and $t_1 = s_q^--1$, in which case $q'=q-1$. The exceptional case when $q=1$ cancels with the summand $(0,1,s_1,s_1+1,\ldots)\otimes (1,2,t_1,t_1+1,\ldots)$ of $f(e_{2i+1})$.
\item The summand $(0,1,s_1,s_1+1\ldots)\otimes (t_1,t_1+1,\ldots,t_s,\widehat{t_s+1},\ldots)_q$ cancels with the summand
\[(0,1,s_1,s_1+1\ldots)\otimes (t_1,t_1+1,\ldots,\widehat{t}_r,t_r+1,\ldots)_{q'}\]
of $K\partial(e_{2i+1})$, where the second summand is obtained by removing $t_s+1$ and adding the predecessor of the left follower of $t_{s}+1$ to its left, which is denoted $t_{r}$. The value of $q'$ is $q$ unless $r=1$ and $t_1 = s_q^--1$, in which case $q'=q+1$. The exceptional case when $q=j$ cancels with the summand $(0,1,s_1,s_1+1,\ldots,s_j,s_j+1)\otimes(s_j+1,t_1,t_1+1,\ldots)$ of $g(e_{2i+1})$.
\end{itemize}
\subsection*{Proof in even degrees} TO BE WRITTEN.
