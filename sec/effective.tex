% !TEX root = ../odd_cartan.tex

\section{Effective constructions}

In this section we construct an explicit Cartan relator for the cochains of simplicial sets and, more generally, for algebras over the Barratt--Eccles operad.

\subsection{Barratt--Eccles operad}

We review the main constructions introduced by Berger and Fresse in \cite{berger2004combinatorial}.

Let $\Set$ and $\sSet$ denote the categories of sets and simplicial sets respectively.
The symmetric monoidal functor $\EE \colon \Set \to \sSet$ is defined on objects by $\EE X_n = X^{n+1}$ with
\[
\begin{split}
	\face_i(x_0,\dots,x_n) &= (x_0,\dots,\widehat{x}_i,\dots,x_n), \\
	\dege_i(x_0,\dots,x_n) &= (x_0, \dots, x_i, x_i, \dots, x_n),
\end{split}
\]
and on morphisms by $\EE f_n = f^{\times n}$.
We notice that if $X$ is equipped with a group action then $\EE X$ is as well with
\[
(x_0,\dots,x_n) \cdot g = (x_0 \cdot g, \dots, x_n \cdot g).
\]

The usual block permutation map
\[
\bcirc_{\sym} \colon \sym_r \times \sym_{s_1} \times \cdots \times \sym_{s_r} \to \sym_{s_1+\dots+s_r}
\]
defined for any $r,s_1,\dots,s_r \in \N$ provides $\sym = \{\sym_r\}_{r>0}$ with the structure of an operad in $\Set$.

The simplicial Barratt--Eccles operad $\rE\sym$ is obtained by applying the functor $\EE$ to the operad $\sym$.
We denote its composition by
\[
\bcirc_{\rE\sym} \colon \rE\sym(r) \times \rE\sym(s_1) \times\dots\times \rE\sym(s_r) \to \rE\sym(s_1+\dots+s_r)
\]
for any $r,s_1,\dots,s_r \in \N$.

Let $\chains \colon \sSet \to \Ch$ be the functor of (normalized) chains with integer coefficients.
The linear Barratt--Eccles operad $\BE$ is defined by $\BE(r) = \chains\EE(r)$ and
\[
\bcirc_{\BE} \defeq \bcirc_{\EE} \circ \EZ \colon \BE(r) \ot \BE(s_1) \ot\dotsb\ot \BE(s_r) \to \BE(s_1+\dots+s_r)
\]
for any $r,s_1,\dots,s_r \in \N$.

For any simplicial set $X$, Berger and Fresse explicitly constructed a natural operad morphism
\[
\phi \colon \BE \to \End(\cochains(X))
\]
on its (normalized) cochains.
Since only the existence of this effective construction will be used here, we refer to the original source for details, and to the specialized computer algebra system \texttt{ComCH} for an implementation.

\subsection{May--Steenrod structure on Barratt--Eccles algebras}

To provide the cochains of simplicial sets, or more generally any $\BE$-algebra, with a natural May--Steenrod structure, it suffices to define a $\cyc$-module quasi-isomorphism
\[
\iota \colon \cW \to \BE.
\]
We recall from \cite{medina2021may_st} a closed form formula for one such map which factors through the $\cyc$-module $\cC = \{\cC(r)\}_{r\in\N}$.
For $r,n\in\N$
\begin{equation*}
	\iota(e_{n}) =
	\begin{cases}
		\displaystyle{\sum_{s_1, \dots, s_m}} \big(0, {s_1}, {s_1+1}, {s_2}, \dots, {s_{m}}, {s_{m}+1} \big) & n = 2m, \\
		\displaystyle{\sum_{s_1, \dots, s_m}} \big(0, 1, {s_1}, {s_1+1}, \dots, {s_{m}}, {s_{m}+1} \big) & n = 2m+1,
	\end{cases}
\end{equation*}
where the sum is over all $s_1, \dots, s_m \in \{0, \dots, r-1\} \cong \cyc_r$.
Please consult \cref{f:small values of psi} for a few examples.

\begin{table}
	\centering
	\resizebox{0.8\columnwidth}{!}{%
\renewcommand{\arraystretch}{1.2}
\begin{tabular}{|c||c|c|c|}
	\hline
	$r$ & $n=2$ & $n=3$ & $n=4$ \\
	\hline
	2 & (0,1,0) & (0,1,0,1) & (0,1,0,1,0) \\
	\hline
	3 & (0,1,2) + (0,2,0) & (0,1,2,0) + (0,1,0,1) & \phantom{+} (0,1,2,0,1) + (0,1,2,1,2) \\
	& & & + (0,2,0,1,2) + (0,2,0,2,0) \\
	\hline
	4 & (0,1,2) + (0,2,3) & (0,1,2,3) + (0,1,3,0) & \phantom{+} (0,1,2,3,0) + (0,1,2,0,1) \\
	& + (0,3,0) & + (0,1,0,1) &
	+ (0,1,2,1,2) + (0,2,3,0,1) \\
	& & & + (0,2,3,1,2) + (0,2,3,2,3) \\
	& & & + (0,3,0,1,2) + (0,3,0,2,3) \\
	& & & + (0,3,0,3,0) \\
	\hline
\end{tabular}
}
\vspace*{3pt}
	\caption{The elements $\psi(e_n)$ for small values of $r$ and $n$.}
	\label{f:small values of psi}
\end{table}

Given a Barratt--Eccles algebra $\phi \colon \BE \to \End(A)$ we will consider it with the May--Steenrod structure $\psi = \phi \circ \iota$, removing $\psi$ from the notation.

\subsection{Summary}

Our goal is to construct a $\cyc$-module chain homotopy $H \colon \cW \to \End(A)^\vee$ from $\tau F$ to $G$ for any $\BE$-algebra $A$.
We will do so by constructing the chain homotopies $K_1$, $K_2$, and $K_3$ in the following diagram where the top and bottom compositions are respectively $\tau F$ and $G$, and $\iota_0^2$ denotes $\iota(e_0) \in \BE(2)$.

\begin{tikzcd}
	&[0pt] \BE(r) \arrow[r,"\id\, \ot \,e^{\ot r}"]
	&[20pt] \BE(r) \ot \BE(2)^{\ot r}
	\arrow[d,"\bcirc_\BE"]
	&[-20pt] \\
	& & \BE^\vee(r)
	\arrow[d, "\tau"] & \\
	\cW(r)
	\arrow[ruu,bend left,"\iota"]
	\arrow[r,"\iota"]
	\arrow[d,"\Delta"']
	\arrow[dr,phantom,"K_3^r"]
	& \cC(r)
	\arrow[uu,hook]
	\arrow[r,bend left,"\tau \comp \chains\EE f"]
	\arrow[r,bend right,"\chains\EE g"']
	\arrow[d,"\Delta_{\AW}"]
	\arrow[r,phantom,"K_1^r"]
	& \BE^\vee(r)
	\arrow[r,"\phi"]
	& \End(A)^\vee(r) \\
	\cW(r)^{\ot 2}
	\arrow[r,"\iota^{\ot 2}"']
	\arrow[dr,bend right,"\iota^{\ot 2}"']
	& \cC(r)^{\ot 2}
	\arrow[dr,phantom,"K_2^r",shift left=7pt]
	\arrow[d,hook] & & \\
	& \BE(r)^{\ot 2}
	\arrow[r,"e\, \ot\, \id"]
	& \BE(2) \ot \BE(r)^{\ot 2}
	\arrow[uu,"\bcirc_\BE"']
\end{tikzcd}

\subsection{The chain homotopy $K_1$}

For $r \in \N$, consider the following group homomorphisms where $e$ denotes the identity element:
\[
\begin{split}
	&f \colon \cyc_r \to \sym_r \xra{\id \times e^{\times r}} \sym_r \times \sym_2^{\times r} \xra{\bcirc_{\sym}} \sym_{2r}, \\
	&g \colon \cyc_r \to \sym_r \xra{e \times \mathrm{diag}} \sym_2 \times \sym_r \times \sym_r \xra{\bcirc_{\sym}} \sym_{2r}.
\end{split}
\]
Explicitly, the images in $\sym_{2r}$ of the preferred generator $\rho = (1,2,\dots,r)$ are:
\begin{align*}
	f(\rho) &= (1,3,\dots,2r-1)(2,4,\dots,2r), \\
	g(\rho) &= (1,2,\dots,r)(r+1,r+2,\dots,2r).
\end{align*}
We notice that
\begin{equation}\label{eq:conjugation of little maps}
	f(\rho) \tau = \tau g(\rho)
\end{equation}
where $\tau$ is as in \cref{ss:reordering}.

%Applying the functor $\chains\EE$ we obtain $\cyc_r$-equivariant chain maps $\chains\EE f$ and $\chains\EE g$, where the $\cyc_r$ action on $\chains\EE\sym_{2r}$ is defined by \eqref{eq:cyclic to 2-symetric}.

The map defined by
\[
K_1(a_0,\dots,a_n) =
\sum_{i=0}^n \ (-1)^i (f(a_0) \tau, \dots, f(a_i) \tau, g(a_i), \dots, g(a_n))
\]
can be easily seen to fit in the diagram
\[
\begin{tikzcd}[column sep=large]
	\cC(r)
	\arrow[r,bend left,"(\chains\EE f) \cdot \tau"]
	\arrow[r,bend right,"(\chains\EE g)"']
	\arrow[r,phantom,"K_1"]
	& \BE(2r)
\end{tikzcd}
\]
as a chain homotopy from $(\chains \EE f) \cdot \tau$ to $\chains \EE g$.
%
%\[
%K_1 \colon \cC \to \BE^\vee
%\]
%from $(\chains\EE f) \cdot \tau$ to $\chains\EE g$
We can use \cref{eq:conjugation of little maps} and the fact that $\rho$ acts on $\BE(2r)$ multiplying on the right by $g(\rho)$ to observe that $K_1$ is $\cyc_r$-equivariant.

\subsection{The chain homotopy $K_2$} \label{ss:coproduct}

We now study the failure of the $\cyc$-module quasi-isomorphism $\iota$ to preserve coalgebra structures.
More precisely, we will construct a $\cyc$-module chain homotopy $K_2$ making the following diagram commute
\begin{equation}\label{d:coproducts}
	\begin{tikzcd}[column sep=large]
			\cW \arrow[r,"\iota"] \arrow[d,"\Delta"'] \arrow[dr,phantom,"K_2"]&
			\cC \arrow[d,"\Delta_{\AW}"] \\
			\cW^{\ot 2} \arrow[r,"\iota^{\ot 2}"'] &
			\cC^{\ot 2}.
		\end{tikzcd}
\end{equation}

Using the bijection $\cyc_r \cong \set{0,\dots,r-1}$, we denote by $\alpha(a_1,\dots,a_\ell)$ the number of increasing consecutive pairs $a_i < a_{i+1}$ in a basis element of $\cC(r)$.
For two such elements $(s_1,\dots,s_j)$ and $(t_1,\dots,t_k)$ we denote respectively by $\varphi(0,s_1,\dots,s_j;t_1,\dots,t_k)$ and $\varphi(1,s_1,\dots,s_j;t_1,\dots,t_k)$ the following expressions in $\cC(r) \ot \cC(r)$:
\[
\alpha(s_1,\dots,s_j,t_1) \cdot
(0,s_1,s_1+1,\dots,s_j,s_j+1) \ot
(t_1,t_1+1,\dots,t_k,t_k+1),
\]
and
\[
- \alpha(s_1-1,\dots,s_j-1,t_1-1) \cdot
(0,1,s_1,s_1+1,\dots,s_j,s_j+1)\otimes (t_1,t_1+1,\dots,t_k,t_k+1).
\]
Then, the chain homotopy $K_2$ is defined by
\[
\begin{split}
	K_2(e_{2i})   &= \sum \, \varphi(0,s_1,\dots,s_j;t_1,\dots,t_k), \\
	K_2(e_{2i+1}) &= \sum \, \varphi(1,s_1,\dots,s_j;t_1,\dots,t_k),
\end{split}
\]
where the sums are taken over all $s_1,\dots,s_j,t_1,\dots,t_k \in \cyc_r$ with $i+1 = j+k$.

Since it is an arduous computation, we postpone the proof showing that $K_2$ is as claimed until \cref{s:postponed}.
Please consult \cref{f:small values of K} for a few example.

\begin{table}
	\centering
	\resizebox{0.8\columnwidth}{!}{%
\renewcommand{\arraystretch}{1.2}
$\begin{array}{|c||c|c|}
	\hline
	n & r=3 & r=4 \\
	\hline
	2 && \phantom{+}(0,1,2)\ot (2,3) \\
	&(0,1,2)\ot(0,1) &+ (0,1,2)\ot(3,0) \\
	&&+ (0,2,3)\ot (3,0) \\
	\hline
	3 &&\phantom{+} (0,1,2,3)\ot(3,0) \\
	&(0,1,2,0)\ot (0,1)&+ (0,1,2,3)\ot(0,1) \\
	&& + (0,1,3,0)\ot (0,1) \\
	\hline
	4 & \phantom{+} (0,1,2,0,1)\ot (1,2) + (0,1,2,0,1)\ot (2,0)& \\
	&+ (0,1,2,1,2)\ot(2,0) + (0,2,0,1,2)\ot(2,0) & \text{33 terms} \\
	&+ (0,1,2)\ot (2,0,1,2) + (0,1,2)\ot (2,0,2,0) &\\
	\hline
	5 & \phantom{+}(0,1,2,0,1,2)\ot (2,0) + (0,1,2,0,1,2)\ot (0,1)& \\
	& + (0,1,2,0,2,0)\ot(0,1) + (0,1,0,1,2,0)\ot(0,1)& \text{33 terms}\\
	& + (0,1,2,0)\ot (0,1,2,0) + (0,1,2,0)\ot (0,1,0,1)& \\
	\hline
\end{array}$
}
\vspace*{5pt}

	\caption{The elements $K(e_n)$ for small values of $r$ and $n$. For $r=2$ or $n<2$ all vanish. Notice that the indices are flipped with respect to \cref{f:small values of psi}.}
	\label{f:small values of K}
\end{table}

\subsection{The chain homotopy $K_3$}


and that the composition
\[
H_3 \colon \cW(r) \xra{\psi(r)} \BE(r) \xra{\psi^2_0 \ot \Delta_{\Shi}} \BE(2) \ot \BE(r)^{\ot 2} \xra{\bcirc_{\BE}} \BE(2r)
\]
is a $\cyc_r$-equivariant chain homotopy between $\chains\EE g$ and $\widetilde G$.

\anibal{@self - Check this}

\subsection{Main construction}

\begin{theorem}
	For any algebra over the Barratt--Eccles operad $\phi \colon \BE \to \End(A)$, the map $H = \phi \circ (K_1 + H_2 + H_3)$ is a Cartan relator for $(A,\psi)$.
\end{theorem}

\begin{proof}
	%Given $K$, we define the $\cyc$-module chain homotopy
	%\[
	%H_2 \colon \cW(r) \xra{K} \BE(r)^{\ot 2} \xra{\psi^2_0 \ot \id} \BE(2) \ot \BE(r)^{\ot 2} \xra{\bcirc_{\BE}} \BE(2r)
	%\]
	%between $G$ and the $\cyc$-module chain map
	%\[
	%\widetilde G \colon \cW(r) \xra{\psi(r)} \BE(r) \xra{\psi^2_0 \ot \Delta_{\AW}} \BE(2) \ot \BE(r)^{\ot 2} \xra{\bcirc_{\BE}} \BE(2r).
	%\]

	Please notice that
	\[
	\chains\EE f = F
	\]
\end{proof}

\subsection{Examples}

We now show some examples computed using the specialized computer algebra system \texttt{ComCH} \cite{medina2021comch} when $A$ is given by the normalized cochains of a standard simplex, and $\phi$ is the $\BE$-algebra structure of Berger and Fresse.
