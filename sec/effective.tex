% !TEX root = ../odd_cartan.tex

\section{Effective constructions}

In this section we construct explicit Cartan relators for algebras over the Barratt--Eccles operad.
In particular, for the cochains of simplicial sets.

\subsection{May--Steenrod structure on the Barratt--Eccles operad}

The following maps were introduced in \cite{medina2021may_st} and shown to define a May--Steenrod structure on $\cE$ using the identity factorization.
Let $\psi(r) \colon \cW(r) \to \cE(r)$ be the $\k[\cyc_r]$-linear map defined on basis elements by
\begin{equation*}
	\psi(r)(e_{n}) = \begin{cases}
		\displaystyle{\sum_{r_1, \dots, r_m}} \big(\rho^0, \rho^{r_1}, \rho^{r_1+1}, \rho^{r_2}, \dots, \rho^{r_{m}}, \rho^{r_{m}+1} \big) & n = 2m, \\
		\displaystyle{\sum_{r_1, \dots, r_m}} \big(\rho^0, \rho^1, \rho^{r_1}, \rho^{r_1+1}, \dots, \rho^{r_{m}}, \rho^{r_{m}+1} \big) & n = 2m+1,
	\end{cases}
\end{equation*}
where the sum is over all $r_1, \dots, r_m \in \{0, \dots, r-1\}$.
We remark that $\psi(r)$ factors through the subcomplex $\chains\EE\cyc_r$ of $\chains\EE\sym_r = \cE(r)$.
Please consult \cref{f:small values of psi} for a few examples.

\begin{table}
	\centering
	\resizebox{0.8\columnwidth}{!}{%
\renewcommand{\arraystretch}{1.2}
\begin{tabular}{|c||c|c|c|}
	\hline
	$r$ & $n=2$ & $n=3$ & $n=4$ \\
	\hline
	2 & (0,1,0) & (0,1,0,1) & (0,1,0,1,0) \\
	\hline
	3 & (0,1,2) + (0,2,0) & (0,1,2,0) + (0,1,0,1) & \phantom{+} (0,1,2,0,1) + (0,1,2,1,2) \\
	& & & + (0,2,0,1,2) + (0,2,0,2,0) \\
	\hline
	4 & (0,1,2) + (0,2,3) & (0,1,2,3) + (0,1,3,0) & \phantom{+} (0,1,2,3,0) + (0,1,2,0,1) \\
	& + (0,3,0) & + (0,1,0,1) &
	+ (0,1,2,1,2) + (0,2,3,0,1) \\
	& & & + (0,2,3,1,2) + (0,2,3,2,3) \\
	& & & + (0,3,0,1,2) + (0,3,0,2,3) \\
	& & & + (0,3,0,3,0) \\
	\hline
\end{tabular}
}
\vspace*{3pt}
	\caption{The elements $\psi(r)(e_n)$ for small values of $r$ and $n$ where we are denoting $(\rho^{r_0}, \dots, \rho^{r_n})$ simply by $(r_0, \dots, r_n)$.}
	\label{f:small values of psi}
\end{table}

\subsection{The lowercase homomorphisms}

We now describe group homomorphisms inducing equivariantly homotopic chain maps.
We will later relate these to the maps $F,G \colon \cW(p) \to \cE(p)$ associated to the May--Steenrod structure on $\cE$ as in \cref{ss:main maps}.

\sssec

Let
\[
\begin{split}
	&f \colon \cyc_p \to \sym_p \xra{\id \times e^{\times p}} \sym_p \times \sym_2^{\times p} \xra{\bcirc_{\sym}} \sym_{2p}, \\
	&g \colon \cyc_p \to \sym_p \xra{e \times \mathrm{diag}} \sym_2 \times \sym_p \times \sym_p \xra{\bcirc_{\sym}} \sym_{2p}.
\end{split}
\]
Explicitly, if $\rho = (1,2,\dots,p)$ in $\sym_p$ then in $\sym_{2p}$
\begin{align}
	\label{eq:explicitl little f}
	f(\rho) &= (1,3,\dots,2p-1)(2,4,\dots,2p), \\
	\label{eq:explicitl little g}
	g(\rho) &= (1,2,\dots,p)(p+1,p+2,\dots,2p).
\end{align}
We notice that
\begin{equation}\label{eq:conjugation of little maps}
	f(\rho) \tau = \tau g(\rho)
\end{equation}
where $\tau$ is as in \cref{ss:reordering}.

\sssec

Let $H_1 \colon \chains\EE\cyc_p \to \chains\EE\sym_{2p}$ be the degree $1$ linear map defined by
\[
H_1(\sigma_0,\dots,\sigma_n) =
\sum_{i=0}^n \ (-1)^i (f(\sigma_0) \tau, \dots, f(\sigma_i) \tau, g(\sigma_i), \dots, g(\sigma_n)).
\]
It can be easily verified that $H_1$ is a homotopy between $\chains \EE f \cdot \tau$ and $\chains \EE g$.
To check it is $\cyc_p$-equivariant we use \cref{eq:conjugation of little maps} and the fact that, according to \cref{ss:cyclic action} and \eqref{eq:explicitl little g}, $\rho$ acts on $\chains\EE\sym_{2p}$ by right multiplication by $g(\rho)$.

\subsection{Comparing coproducts}

We now study the failure of the May--Steenrod structure map $\psi(p)$ to be a coalgebra map.
More precisely, we will construct a $\cyc_p$-equivariant chain homotopy making the following diagram commute
\[
\begin{tikzcd}[column sep=large]
	\cW(p) \arrow[r,"\psi(p)"] \arrow[d,"\Delta"] & \chains\EE\cyc_p \arrow[d,"\Delta_{\AW}"]\\
	\cW(p)^{\ot 2} \arrow[r,"\psi(p)^{\ot 2}"] & \chains\EE\cyc_p^{\ \ot 2}.
\end{tikzcd}
\]
We will then use this homotopy to deform $G$ equivariantly to another map $G'$.

\sssec

\TBW

\sssec

Given $K$ as above the composition
\[
H_2 \colon \cW(p) \xra{K} \cE(p)^{\ot 2} \xra{\psi^2_0 \ot \id} \cE(2) \ot \cE(p)^{\ot 2} \xra{\bcirc_{\cE}} \cE(2p)
\]
is easily seen to be a $\cyc_p$-equivariant chain homotopy between $G$, as defined in \cref{ss:main maps}, and
\[
G' \colon \cW(p) \xra{\psi(p)} \cE(p) \xra{\psi^2_0 \ot \Delta_{\AW}} \cE(2) \ot \cE(p)^{\ot 2} \xra{\bcirc_{\cE}} \cE(2p).
\]

\subsection{Lower and upper case maps}

We now relate $f$ and $g$ to $F$ and $G$.

\sssec

It is immediate that
\[
\chains\EE f = F.
\]

\sssec

The composition

\[
H_3 \colon \cW(p) \xra{\psi(p)} \cE(p) \xra{\psi^2_0 \ot \Delta_{\mathrm{SHI}}} \cE(2) \ot \cE(p)^{\ot 2} \xra{\bcirc_{\cE}} \cE(2p)
\]
is easily seen to be a $\cyc_p$-equivariant chain homotopy between $G'$ and $\chains\EE g$.

\subsection{Main construction}

The map $H = H_1 + H_2 + H_3$ satisfies the conditions of \cref{ss:main theorem} and consequently defines Cartan relators.