% !TEX root = ../odd_cartan.tex

\appendix
\section{The map $K$}\label{s:closed formula}

\subsection*{old}

\begin{align*}
	K(e_{2i}) &= \sum_{j,k} \, \sum_{q \leq j} \, \sum \,
	(0,s_1,s_1+1,\dots,s_j,s_j+1) \ot (t_1,t_1+1,\dots,t_k,t_k+1) \\
	K(e_{2i+1}) &= - \sum_{j,k} \, \sum_{q \leq j} \, \sum \,
	(0,1,s_1,s_1+1,\dots,s_j,s_j+1) \ot (t_1,t_1+1,\dots,t_k,t_k+1)
\end{align*}
where the first sum is over positive integers $j,k$ with $i+1 = j+k$, the second over $q$ in $\set{1,\dots,j}$, and the third over all $s_1,\dots,s_j, t_1,\dots,t_k$ in $\{0,1,\dots,p-1\}$ satisfying the following condition:

We claim that the homotopy $K$ so defined as the following form: We assume $j,k>0$, and denote by $s_q^-$ the entry to the left of the entry $s_q$.
Explicitly, we have
\begin{align*}
	s_q^- &= \begin{cases}s_{q-1}+1 & \text{if $q>1$}\\
		s_q^- = 0 & \text{ if $q=1$ and even degree} \\
		s_q^- = 1 &\text{ if $q=1$ and odd degree.}
	\end{cases}
\end{align*}
The condition $s_q^-<s_q<t_1<s_q^-$ makes sense with the cyclic order in $\{0,1,\dots,p-1\}$.
In the following sums, we are summing along all possible values of $s_1,\dots,s_j,t_1,\dots,t_k$ in $\{0,1,\dots,p-1\}$.
\begin{align*}
	K(e_{2i}) &= \sum_{j+k = i+1}{\sum_{q=1}^j{\sum_{s_q^-<s_q<t_1<s_q^-}{(0,s_1,s_1+1,\dots,s_j,s_j+1)\ot(t_1,t_1+1,\dots,t_k,t_k+1)}}} \\
	K(e_{2i+1}) &= -\sum_{j+k = i+1}{\sum_{q=1}^j{\sum_{s_q^-<s_q<t_1<s_q^-}{(0,1,s_1,s_1+1,\dots,s_j,s_j+1)\ot(t_1,t_1+1,\dots,t_k,t_k+1)}}}
\end{align*}

Let us check these formulae inductively.
\begin{align*}
	K(e_{2i+1}) &= h(f(e_{2i+1})-g(e_{2i+1})-K(\bd(e_{2i+1})))
\end{align*}
Now, all the summands in $f(e_{2i+1})$ and $g(e_{2i+1})$ start with a zero, therefore we are left with
\[K(e_{2i+1}) = -h(K(\rho e_{2i}))+h(K(e_{2i}))\]
and the second summand is zero, while the first one agrees with the definition of $K(e_{2i+1})$.

\begin{align*}
	K(e_{2i}) &= h(f(e_{2i})-g(e_{2i})-K(\bd(e_{2i})))
\end{align*}
Here, the summand $h(g(e_{2i}))$ is zero, the summand $K(\bd(e_{2i}))$ corresponds to the summation with $q>1$ and the summand $h(f(e_{2i}))$ corresponds to the summations with $q=1$.

\subsection{new}

	The formulas can be written as
\begin{align*}
	K(e_{2i}) &= \sum_{j+k = i+1} \varphi(0,s_1,\ldots,s_j;t_1)\cdot (0,s_1,s_1+1,\ldots,s_j,s_j+1)\otimes(t_1,t_1+1,\ldots,t_k,t_k+1) \\
	K(e_{2i+1}) &= \sum_{j+k = i+1} \varphi(0,1,s_1,\ldots,s_j;t_1)\cdot (0,1,s_1,s_1+1,\ldots,s_j,s_j+1)\otimes(t_1,t_1+1,\ldots,t_k,t_k+1)
\end{align*}
for certain coefficients $\varphi(0,s_1,\ldots,s_j;t_1)$ and $\varphi(0,1,s_1,\ldots,s_j;t_1)$. We will deduce three formulas for these coefficients, the third being the more compact.

Observe that if $s_{i+1} = s_i$ or $s_1 = 0$ (in the even case) or $s_1=1$ (in the odd case) we get degenerate summands, which are zero. So we do not treat these cases in the following computations.

\subsection*{First computation} Given a summand $(0,s_1,s_1+1,\ldots,s_j,s_j+1)\otimes (t_1,t_1+1,\ldots,t_k,t_k+1)$, let $(s_1',\ldots,s'_k)$ be the only sequence of integers such that
\begin{itemize}
	\item $s_1'$ is the only integral representative of $s_1$ with $0< s_1'< p$.
	\item $1<s_{i+1}'-s_i'\leq p$ for all $1\leq i<j$.
\end{itemize}
Then, that summand satisfies the condition $s_q^-<s_q<t_1<s_q^-$ for $q>1$ if and only if the interval $(s_{q-1}',s_q']$ does not contain a representative of $t_1$, and satisfies that condition for $q=1$ if and only if the interval $[0,s_1']$ does not contain a representative of $t_1$. Writing $\hat{\varphi}(0,s_1,\ldots,s_j;t_1)$ for the number of representatives of $t_1$ that lie in the interval $[0,s_j']$, we obtain that the number
\[\varphi(0,s_1,\ldots,s_j;t_1) = j-\hat{\varphi}(s_1,\ldots,s_j;t_1)\]
counts how many times the condition $s_q^-<s_q<t_1<s_q^-$ is satisfied.

For a summand $(0,1,s_1,s_1+1,\ldots,s_j,s_j+1)\otimes (t_1,t_1+1,\ldots,t_k,t_k+1)$, let $(s_1',\ldots,s'_k)$ be the only sequence of integers such that
\begin{itemize}
	\item $s_1'$ is the only integral representative of $s_1$ with $1< s_1'\leq p$.
	\item $1<s_{i+1}'-s_i'\leq p$ for all $1\leq i<j$.
\end{itemize}
Then, that summand satisfies the condition $s_q^-<s_q<t_1<s_q^-$ for $q>1$ if and only if the interval $(s_{q-1}',s_q']$ does not contain a representative of $t_1$, and satisfies that condition for $q=1$ if and only if the interval $(0,s_1]$ does not contain a representative of $t_1$. Writing $\hat{\varphi}(1,s_1,\ldots,s_j;t_1)$ for the number of representatives of $t_1$ that lie in the interval $(0,s_j']$, we obtain that the number
\[\varphi(1,s_1,\ldots,s_j;t_1) = j-\hat{\varphi}(1,s_1,\ldots,s_j;t_1)\]
counts how many times the condition $s_q^-<s_q<t_1<s_q^-$ is satisfied.

\subsection*{Second computation} Let us write $\bar{s}_i$ for the only representative of $s_i$ that lies in the interval $[0,p)$. Let us write $\bar{t}_1$ for the representative of $t_1$ that lies in the interval $[0,p)$ and $\check{t}_1$ for the representative of $t_1$ that lies in the interval $(0,p]$. Let $r$ be the number of consecutive pairs in the sequence $(0,\bar{s}_1,\ldots,\bar{s}_j)$ that are non-increasing and observe that $s'_j = \bar{s}_j + rp$. Then we have that
\begin{align*}
	\hat{\varphi}(0,s_1,\ldots,s_j;t_1) &= \begin{cases}
		r & \text{if $\bar{t}_1>\bar{s}_j$} \\
		r+1 & \text{if $\bar{t}_1\leq \bar{s}_j$}
	\end{cases}
	&
	\hat{\varphi}(1,s_1,\ldots,s_j;t_1) &= \begin{cases}
		r & \text{if $\check{t}_1>\bar{s}_j$} \\
		r+1 & \text{if $\check{t}_1\leq \bar{s}_j$}
	\end{cases}
\end{align*}

\subsection*{Third computation} From the latter formulas, we deduce that $\varphi(0,s_1,\ldots,s_j;t_1)$ is one less than the number of strictly increasing consecutive pairs in the sequence $(0,\bar{s}_1,\ldots,\bar{s}_j,\bar{t}_1)$ and $\varphi(1,s_1,\ldots,s_j;t_1)$ is one less than the number of strictly increasing consecutive pairs in the sequence $(1,\bar{s}_1,\ldots,\bar{s}_j,\check{t}_1)$.

\subsection*{Examples} For $p=7$ we have:
\begin{align*}
	\varphi(0,3,5,2,6,1,0,0;3) &= 3 \\
	\varphi(0,3,1,1,0,2,0,3;0) &= 2 \\
	\varphi(1,0,0,0,0,0,0,0;2) &= 0 \\
	\varphi(1,4,6,4,1,3,6,0;2) &= 3 \\
	\varphi(1,0,3,0,6,6,0,6;0) &= 3
\end{align*}
In the first example the strictly increasing consecutive pairs are $(0,3),(3,5),(2,6),(0,3)$. In the second example the strictly increasing consecutive pairs are $(0,3),(0,2),(0,3)$. In the third example the only such pair is $(0,2)$. In the fourth example we have $(1,4),(4,6),(1,3),(0,2)$. In the last one, $(0,3),(0,6),(0,6),(6,7)$.