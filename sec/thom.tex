% !TEX root = ../odd_cartan.tex

\appendix
\section{An effective proof of Thom's theorem}\label{s:thom}

%In this section we will provide an effective proof of Thom's theorem \cite[218]{steenrod1953cyclic} as stated in \cref{ss:homology_operations}.

Let $(A,\psi)$ be a May--Steenrod complex.
Recall that
\[
\rD_i^p \colon \rH_n(A; \Fp) \to \rH_{np+i}(A; \Fp)
\]
is defined, after extending scalar to $\Fp$, by sending an $n$-cycle $a$ to $\psi_i(a^{\ot p})$.
Thom's theorem (\cite[218]{steenrod1953cyclic}) states that the map $\rD_i^p$ is $0$ unless for some $\varepsilon \in \{0,1\}$ either: $n$ is even and $i+\varepsilon$ is an even multiple of $(p-1)$ or $n$ is odd and $i+\varepsilon$ is an odd multiple of $(p-1)$.
We will lift to the chain level Steenrod's proof of this statement.
Throughout this section $k \in \Fp$ is a non-zero element.

\subsection{Thom $i$-boundary constructions}

A \textit{Thom $i$-boundary construction} at the prime $p$ is a map $\theta_i$ from mod~$p$ cycles in $A$ to $A$, natural with respect to $E_\infty$-algebra maps, such that
\[
\bd \theta_i(a) = k_i(a) \psi_i(a^{\ot p})
\]
for some non-zero $k_i(a) \in \Fp$, unless for some $\varepsilon \in \set{0,1}$ either: $n$ is even and $i+\varepsilon$ is an even multiple of $(p-1)$ or $n$ is odd and $i+\varepsilon$ is an odd multiple of $(p-1)$.

Clearly, the existence of a Thom $i$-boundary construction for each $i \in \N$ implies Thom's theorem.

\subsection{The map $\lambda$}

Let $\lambda \colon \cW \to \cW$ be defined by
\[
\lambda(\rho^s e_{2i}) = k^i \rho^{ks} e_{2i}, \qquad
\lambda(\rho^s e_{2i+1}) = k^i (1+\rho+\dots+\rho^{k-1}) \rho^{ks} e_{2i+1}.
\]
We verify that $\lambda$ is a chain map, which is $\cyc$-linear if the action of $\rho$ on the target is by $\rho^k$.
Using that $\rho N = N$ we have
\[
\bd \comp \lambda(\rho^s e_{2i}) = k^i \rho^{ks} \bd (e_{2i}) = k^i \rho^{ks} N e_{2i-1} = k^i N e_{2i-1}
\]
and
\begin{align*}
	\lambda \comp \bd(\rho^s e_{2i}) &= \lambda(N e_{2i-1}) = k^{i-1}(1+\rho+\dots+\rho^{k-1}) (1+\rho^k+\dots+\rho^{kp-k}) e_{2i-1} \\ &=
	k^i N e_{2i-1}
\end{align*}
since $(1+\rho+\dots+\rho^{k-1}) (1+\rho^k+\dots+\rho^{kp-k}) = kN$.
This proves the claim for even degrees.
For odd degrees, we have
\begin{multline*}
	\bd \comp \lambda(\rho^s e_{2i+1}) = k^i (1+\rho+\dots+\rho^{k-1}) \rho^{ks} \bd (e_{2i+1}) \\
	= k^i (1+\rho+\dots+\rho^{k-1}) \rho^{ks} T e_{2i} = k^i \rho^{ks} (\rho^k-1) e_{2i}
\end{multline*}
and
\[
\lambda \comp \bd(\rho^s e_{2i+1}) = \lambda(\rho^s T e_{2i}) = k^i\rho^{ks}(\rho^k-1) e_{2i}.
\]

\subsection{The element $\gamma$}

Let $\gamma \in \sym_p \cong \mathrm{Bij}(\Fp, \Fp)$ defined by
\[
\gamma(i) = ki.
\]
We remark that $\gamma$ is an odd permutation.
We use the notation $\sigma^\gamma = \gamma \sigma \gamma^{-1}$
%\begin{align*}
%	(-)^\gamma \colon \sym_p &\to \sym_p \\
%	\sigma &\mapsto \gamma \sigma \gamma^{-1}
%\end{align*}
and notice that $\rho^\gamma = \rho^k$.
This implies that the $\EE\,(-)^\gamma \colon \EE\sym_p \to \EE\sym_p$ restricts to a simplicial map $\EE\cyc_p \to \EE\cyc_p$, and we denote the induced chain map by $\gamma_\ast$, which is $\cyc$-equivariant if the action of $\rho$ on the target is by $\rho^k$.

\subsection{Thom relator}

A \textit{Thom relator} $L_\psi^p \colon W \to \Hom(A^{\ot p}, A)$ for a May--Steenrod complex $(A, \psi)$ is a chain homotopy between $\gamma \comp \psi$ and $\psi \comp \lambda$.

\begin{theorem*}
	If $L_\psi^p$ is a Thom relator for $(A, \psi)$ then
	\[
	\theta_i(a) = L_\psi^p(e_i)(a^{\ot p})
	\]
	defines a Thom $i$-boundary construction.
\end{theorem*}

\begin{proof}
	Let us write $L$ instead of $L_\psi^p$.
	Since $a$ is a mod~$p$ cycle,
	\begin{align*}
		\bd \comp\, \theta_i(a) & =
		\bd \comp L(e_i)(a^{\ot p}) \\ & =
		\big(\bd \comp L(e_i) + L(e_i) \comp \bd \big) (a^{\ot p}) \\ & =
		\big(\bd L(e_i)\big)(a^{\ot p}) \\ & =
		\big((\bd L)(e_i) - L(\bd e_i)\big)(a^{\ot p}).
	\end{align*}
	Since
	\begin{align*}
		&L(\bd e_{2k})(a^{\ot p}) = L(e_{2k-1}) N(a^{\ot p}) = 0, \\
		&L(\bd e_{2k+1})(a^{\ot p}) = L(e_{2k}) T(a^{\ot p}) = 0,
	\end{align*}
	we have
	\begin{align*}
		\bd \comp\, \theta_i(a) & =
		(\bd L)(e_i)(a^{\ot p}) \\ & =
		(\psi \comp \lambda - \gamma \comp \psi)(e_i)(a^{\ot p}) \\ & =
		\begin{cases}
			\big(k^i - (-1)^{\bars{a}}\big) \psi_i(a^{\ot p}) & i=2k, \\
			\big(2^{k+1} - (-1)^{\bars{a}}\big) \psi_i(a^{\ot p}) & i=2k+1,
		\end{cases}
	\end{align*}
	where we used that $\gamma$ is an odd permutation and that $\rho$ is an even one.
	The claim follows from the fact that $k^i \equiv 1$ mod $p$ if and only if $k = \ell(p-1)$ for some $\ell$, and $k^i \equiv -1$ mod $p$ if and only if $2k = (2\ell+1)(p-1)$ for some $\ell$.
\end{proof}

\subsection{The homotopy $M$}

\TBW

%\begin{proof}
%	First, notice that $(1-k^i) \equiv 0$ mod $p$ if and only if $i$ is a multiple of $(p-1)$.
%
%
%	\[
%	C^p_\psi(i)(a,b) =
%	\big(\tau F_\psi^p - G_\psi^p\big)(e_i)(a^{\ot p} \ot b^{\ot p}).
%	\]
%	Second, using that in $A \ot \Fp$
%	\begin{align*}
%		H_\psi^p(\bd e_{2i})(a^{\ot p} \ot b^{\ot p}) =&\,
%		(\rho - 1) H_\psi^p(e_{2i})(a^{\ot p} \ot b^{\ot p}) = 0, \\
%		H_\psi^p(\bd e_{2i+1})(a^{\ot p} \ot b^{\ot p}) =&\,
%		(1+\rho+\dots+\rho^{p-1}) H_\psi^p(e_{2i+1})(a^{\ot p} \ot b^{\ot p}) = 0,
%	\end{align*}
%	we have
%	\begin{align*}
%		\bd_A \zeta_i(a,b) =&
%		\bd_A H_\psi^p(e_i)(a^{\ot p} \ot b^{\ot p}) \\ =&
%		\bd_{\End(A)} H_\psi^p(e_i)(a^{\ot p} \ot b^{\ot p}) \\ =& \,
%		\big(\tau F(e_i) - G(e_i)\big)(a^{\ot p} \ot b^{\ot p}) -
%		H_\psi^p(\bd e_i)(a^{\ot p} \ot b^{\ot p}) \\ =&\,
%		C^p_\psi(i)(a,b).\qedhere
%	\end{align*}
%\end{proof}
