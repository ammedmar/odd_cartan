% !TEX root = ../odd_cartan.tex

\appendix
\section{An effective proof of Thom's theorem}\label{s:thom}

%In this section we will provide an effective proof of Thom's theorem \cite[218]{steenrod1953cyclic} as stated in \cref{ss:homology_operations}.

Let $(A,\psi)$ be a May--Steenrod complex.
Recall that
\[
\rD_i^p \colon \rH_n(A; \Fp) \to \rH_{np+i}(A; \Fp)
\]
is defined, after extending scalar to $\Fp$, by sending an $n$-cycle $a$ to $\psi_i(a^{\ot p})$.
Thom's theorem (\cite[218]{steenrod1953cyclic}) states that the map $\rD_i^p$ is $0$ unless for some $\varepsilon \in \{0,1\}$ either: $n$ is even and $i+\varepsilon$ is an even multiple of $(p-1)$ or $n$ is odd and $i+\varepsilon$ is an odd multiple of $(p-1)$.
We will lift to the chain level Steenrod's proof of this statement.

Throughout this section $k \in \Fp$ is a fixed non-zero scalar.

\subsection{Thom $i$-boundary constructions}

A \textit{Thom $i$-boundary construction} at the prime $p$ is a map $\theta_i$ from mod~$p$ cycles in $A$ to $A$, natural with respect to $E_\infty$-algebra maps, such that
\[
\bd \theta_i(a) = k_i(a) \psi_i(a^{\ot p})
\]
for some non-zero $k_i(a) \in \Fp$, unless for some $\varepsilon \in \set{0,1}$ either: $n$ is even and $i+\varepsilon$ is an even multiple of $(p-1)$ or $n$ is odd and $i+\varepsilon$ is an odd multiple of $(p-1)$.

Clearly, the existence of a Thom $i$-boundary construction for each $i \in \N$ implies Thom's theorem.

\subsection{The map $\lambda$}

Let $\lambda \colon \cW \to \cW$ be defined by
\[
\lambda(\rho^s e_{2i}) = k^i \rho^{ks} e_{2i}, \qquad
\lambda(\rho^s e_{2i+1}) = k^i (1+\rho+\dots+\rho^{k-1}) \rho^{ks} e_{2i+1}.
\]
We verify that $\lambda$ is a chain map, which is $\cyc$-linear if the action of $\rho$ on the target is by $\rho^k$.
Using that $\rho N = N$ we have
\[
\bd \comp \lambda(\rho^s e_{2i}) = k^i \rho^{ks} \bd (e_{2i}) = k^i \rho^{ks} N e_{2i-1} = k^i N e_{2i-1}
\]
and
\begin{align*}
	\lambda \comp \bd(\rho^s e_{2i}) &= \lambda(N e_{2i-1}) = k^{i-1}(1+\rho+\dots+\rho^{k-1}) (1+\rho^k+\dots+\rho^{kp-k}) e_{2i-1} \\ &=
	k^i N e_{2i-1}
\end{align*}
since $(1+\rho+\dots+\rho^{k-1}) (1+\rho^k+\dots+\rho^{kp-k}) = kN$.
This proves the claim for even degrees.
For odd degrees, we have
\begin{multline*}
	\bd \comp \lambda(\rho^s e_{2i+1}) = k^i (1+\rho+\dots+\rho^{k-1}) \rho^{ks} \bd (e_{2i+1}) \\
	= k^i (1+\rho+\dots+\rho^{k-1}) \rho^{ks} T e_{2i} = k^i \rho^{ks} (\rho^k-1) e_{2i}
\end{multline*}
and
\[
\lambda \comp \bd(\rho^s e_{2i+1}) = \lambda(\rho^s T e_{2i}) = k^i\rho^{ks}(\rho^k-1) e_{2i}.
\]

\begin{remark*}
	In both \cite[p.219]{steenrod1953cyclic} and \cite[p.159]{may1970general} the map $\cW \to \cW$ defined by the assignment
	\[
	e_{2i} \mapsto k^i e_{2i}, \qquad
	e_{2i+1} \mapsto k^i (1+\rho+\dots+\rho^{k-1}) e_{2i+1}
	\]
	is claimed to be a chain map, but the $\cyc$-equivariant extension of this assignment is not a chain map with the usual action of $\cyc_p$ on the target $\cW$.
\end{remark*}

\subsection{The element $\gamma$}

Let $\gamma \in \sym_p \cong \mathrm{Bij}(\Fp, \Fp)$ be defined by $\gamma(q) = kq$.
We remark that $\gamma$ is an odd permutation.
We use the notation $\sigma^\gamma = \gamma \sigma \gamma^{-1}$
and notice that $\rho^\gamma = \rho^k$.
This implies that the induced simplicial map $\EE\sym_p \to \EE\sym_p$ restricts to a map $\EE\cyc_p \to \EE\cyc_p$.
We denote its induced chain map by $\gamma_\ast \colon \cC(p) \to \cC(p)$, which is $\cyc$-equivariant if the action of $\rho$ on the target is by $\rho^k$.

\subsection{Thom relator}

A \textit{Thom relator} $L_\psi^p \colon \cW \to \Hom(A^{\ot p}, A)$ for a May--Steenrod complex $(A, \psi)$ is a $\cyc$-equivariant chain homotopy between $\gamma \comp \psi$ and $\psi \comp \lambda$ where the action of $\rho$ on the target is by $\rho^k$.
Explicitly, this means that $L_\psi^p(\rho e_i) = \rho^k L_\psi^p(e_i)$.

\begin{theorem*}
	If $L_\psi^p$ is a Thom relator for $(A, \psi)$ then
	\[
	\theta_i(a) = L_\psi^p(e_i)(a^{\ot p})
	\]
	defines a Thom $i$-boundary construction.
\end{theorem*}

\begin{proof}
	Let us write $L$ instead of $L_\psi^p$.
	Since $a$ is a mod~$p$ cycle,
	\begin{align*}
		\bd \comp\, \theta_i(a) & =
		\bd \comp L(e_i)(a^{\ot p}) \\ & =
		\big(\bd \comp L(e_i) + L(e_i) \comp \bd \big) (a^{\ot p}) \\ & =
		\big(\bd L(e_i)\big)(a^{\ot p}) \\ & =
		\big((\bd L)(e_i) - L(\bd e_i)\big)(a^{\ot p}) \\ & =
		\big((\psi \comp \lambda - \gamma \comp \psi)(e_i) - L(\bd e_i)\big) (a^{\ot p}).
	\end{align*}
	Since for any $\ell \in \N$,
	\begin{align*}
		&L(\bd e_{2\ell})(a^{\ot p}) = L(N e_{2\ell-1}) = (1+\rho^k+\dots+\rho^{kp-k})L(e_{2\ell-1}) (a^{\ot p}) = 0, \\
		&L(\bd e_{2\ell+1})(a^{\ot p}) = L(T e_{2\ell}) = (\rho^k - 1) L(e_{2\ell})(a^{\ot p}) = 0,
	\end{align*}
	we have
	\begin{align*}
		\bd \comp\, \theta_i(a) & =
		(\psi \comp \lambda - \gamma \comp \psi)(e_i)(a^{\ot p}) \\ & =
		\begin{cases}
			\big(k^\ell - \gamma\big) \psi_i(a^{\ot p}) & i = 2\ell, \\
			\big(k^\ell (1+\rho\dots+\rho^k) - \gamma\big) \psi_i(a^{\ot p}) & i = 2\ell+1,
		\end{cases}
		\\ & =
		\begin{cases}
			\big(k^\ell - (-1)^{\bars{a}}\big) \psi_i(a^{\ot p}) & i = 2\ell, \\
			\big(k^{\ell+1} - (-1)^{\bars{a}}\big) \psi_i(a^{\ot p}) & i = 2\ell+1,
		\end{cases}
	\end{align*}
	where we used that $\gamma$ and $\rho$ are an even and an odd permutation respectively.

	The claim follows from the fact that $k^\ell \equiv 1$ mod $p$ if and only if $k = r(p-1)$ for some $r$, and $k^\ell \equiv -1$ mod $p$ if and only if $2\ell = (2r+1)(p-1)$ for some $r$.
\end{proof}

\subsection{Thom relators for Barratt--Eccles algebras}

Recall that $\cC(p) = \chains \EE \cyc_p$ and $\BE(p) = \chains \EE \sym_p$.
Let $A$ be a Barratt--Eccles algebra with structure map $\phi \colon \BE \to \End(A)$, and recall from \cref{ss:may-steenrod on barratt-eccles} the May--Steenrod structure on the Barratt--Eccles operad $\cW \xra{\iota} \cC \hookrightarrow \BE$.
The goal of this subsection if to construct explicit equivariant homotopies making the following diagram commute:
\[
\begin{tikzcd}
	\cW(p) \arrow[r, "\iota"] \arrow[d,"\lambda"] \arrow[dr,phantom,"L_1"] &
	\cC(p) \arrow[r, hook] \arrow[d, "\gamma_\ast"] \arrow[dr,phantom,"L_2"] &
	\BE(p) \arrow[r,"\phi"] \arrow[d,"\gamma"] &
	\Hom(A^{\ot p}, A) \arrow[d,"\gamma"] \\
	\cW(p) \arrow[r, "\iota"] &
	\cC(p) \arrow[r, hook] &
	\BE(p) \arrow[r,"\phi"] &
	\Hom(A^{\ot p}, A),
\end{tikzcd}
\]
where the last two vertical maps are given by the action of $\gamma \in \sym_p$.

\subsubsection{The chain homotopy $L_1$}

Let us consider the endomorphisms $\xi$ and $\eta$ of $\cC(p)$ given by
\[
\xi(\rho^{s_0},\dots,\rho^{s_n}) =
\begin{cases}
	(\rho^0) & \text{if } n = 0, \\
	\hfil0 & \text{otherwise}.
\end{cases}
\]
and
\[
\eta(\rho^{s_0},\dots,\rho^{s_n}) = (0,\rho^{s_0},\dots,\rho^{s_n})
\]
respectively, and denote by $\mu = \iota \comp \lambda$ and $\nu = \gamma \comp \iota$.
The following claim is easily verified.

\begin{lemma*}
	The pair $(\xi,\eta)$ is an idempotent homotopy identity and
	\[
	\xi \circ (\mu - \nu) = 0.
	\]
\end{lemma*}
Since $\cW$ has a preferred basis, \cref{t:recursive_homotopy} applies and produces the desired equivariant chain homotopy $L_1$.

\subsubsection{The homotopy $L_2$}

Since $\gamma_\ast$ is the restriction chain map $\chains\EE\,(-)^\gamma \colon \chains\EE\sym_p \to \chains\EE\sym_p$ induced by the conjugation $(-)^\gamma$, it suffices to construct a $\cyc$-equivariant chain map, where $\rho$ acts as $\rho^k$ on the target, from it to the map defined by the (right) action of $\gamma$.
We proceed analogously to \cref{ss:K1}.
We define the chain homotopy
\[
L_1(\sigma_0,\dots,\sigma_n) = \sum (-1)^i (\sigma_0\gamma,\dots,\sigma_i\gamma,\sigma_i^\gamma,\dots,\sigma_n^\gamma)
\]
and verify it is $\cyc$-equivariant as follows:
\begin{align*}
	(L_1 \comp \rho)(\rho^{s_0},\dots,\rho^{s_n}) &=
	L_1(\rho^{s_0}\rho, \dots, \rho^{s_n}\rho) \\ &=
	\sum (-1)^i (\rho^{s_0} \rho\,\gamma,\dots,\rho^{s_i} \rho\,\gamma,(\rho^{s_i} \rho)^\gamma,\dots,(\rho^{s_n} \rho)^\gamma) \\ &=
	\sum (-1)^i (\rho^{s_0} \gamma \rho^k,\dots,\rho^{s_i} \gamma \rho^k,(\rho^{s_i})^\gamma \rho^k ,\dots,(\rho^{s_n})^\gamma \rho^k) \\ &=
	(\rho^k \comp L_1)(\rho^{s_0},\dots,\rho^{s_n}).
\end{align*}