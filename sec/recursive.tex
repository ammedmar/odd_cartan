% !TEX root = ../odd_cartan.tex

\section{Homotopies recursively}\label{s:equivariant homotopy general}

\noindent In this section we give a recursive procedure to construct equivariant homotopies for maps between a bounded-below chain complex of free $\Z[G]$-modules and a chain complex equipped with a contraction.

A \textit{homotopy identity} is a pair $(\xi,\eta)$, where $\xi \colon D \to D$ is a chain map and $\eta$ is a homotopy from $\xi$ to $\id_D$.
We say that $(\xi, \eta)$ is \textit{idempotent} if $\xi \comp \eta = 0$.
The composition $\xi \comp \xi$ is homotopic to $\xi$ via the homotopy $\xi \comp \eta$.
If the homotopy identity is idempotent, then the homotopy vanishes, thus $\xi \comp \xi = \xi$.

\begin{theorem}\label{t:recursive_homotopy}
	Let $C$ a bounded-below chain complex of free $\Z[G]$-modules with a basis.
	Let $D$ be a chain complex of $\Z[G]$-modules together with a $(\Z$-linear$)$ idempotent homotopy identity $(\xi, \eta)$.
	Then, for any pair of $\Z[G]$-linear chain maps $\mu,\nu \colon C \to D$ with $\xi \comp (\mu -\nu) = 0$, the $\Z[G]$-linear map $H \colon C \to D$ recursively defined on basis elements by
	\begin{equation}\label{eq:lemma def}
		H(b) = \eta \comp (\mu - \nu - H \comp \bd)(b)
	\end{equation}
	is a $G$-equivariant homotopy from $\nu$ to $\mu$.
\end{theorem}

\begin{proof}
	We will use an induction argument on the degree of $b$.
	Suppressing composition symbols we have
	\begin{align*}
		(\bd H + H \bd)(b)
		&= \bd \eta(\mu - \nu - H \bd)(b) + H \bd (b) \\
		&= (\id_D - \xi - \eta \bd)(\mu - \nu - H \bd)(b) + H \bd(b) \\
		&= (\mu-\nu)(b) - (\xi + \eta \bd)(\mu - \nu - H \bd)(b).
	\end{align*}
	Therefore, the claim follows from proving the identity
	\begin{equation}\label{eq:lemma want}
		(\xi + \eta \bd)(\mu - \nu - H \bd)(b) = 0.
	\end{equation}
	In the lowest degree $b$, $\mu(b)$ and $\nu(b)$ are cycles, so \eqref{eq:lemma want} reduces to $\xi(\mu-\nu) = 0$, which gives the base case of the induction.	For the induction step, observe first that $\xi H$ vanishes on basis elements because $\xi\eta = 0$, and thus vanishes on all elements. As $\xi (\mu-\nu) = 0$, the identity \eqref{eq:lemma want} reduces to
    \[
        \eta\bd(\mu-\nu-H\bd)(b) = 0.
     \]
    Since $\bd(\mu-\nu-H\bd)(b) = (\mu-\nu-\bd H)(\bd b)$, we may apply the induction hypothesis to equate the latter to $(H\bd)(\bd b)$, which vanishes.
\end{proof}

\begin{remark*}
	A \emph{strong deformation retract} \cite{Lambe-Stasheff} of a chain complex $D$ is a tuple $(D',f,g,\eta)$, where $f\colon D' \to D$ and $g\colon D\to D'$ are chain maps, $g \comp f = \id_{D'}$ and $\eta$ is a homotopy from $\xi=f \comp g$ to $\id_D$. If $D'$ is a subcomplex of $D$, then the data of a strong deformation retract can be summarised as a homotopy identity $(\xi,\eta)$ such that $\xi\comp\xi = \xi$. A \emph{contraction} \cite{EML,gonzalez2003computation} is a strong deformation retract that additionally satisfies the following:
	\[
	\text{(i)}\ \xi \comp \eta = 0, \quad
	\text{(ii)}\ \eta \comp \xi = 0, \quad
	\text{(iii)}\ \eta \comp \eta = 0.
	\]
	Using this language, an idempotent homotopy identity is a strong deformation retract satisfying (i), and we remark that \cref{t:recursive_homotopy} is an adaptation of \cite[p.367]{Lambe-Stasheff}.
\end{remark*}
