% !TEX root = ../odd_cartan.tex

\section{May--Steenrod structures}\label{s:may_st}

\noindent We now present a summary of \cite{medina2021may_st}, which makes explicit the definitions of \cite{steenrod1953cyclic} from the operadic viewpoint of \cite{may1970general}.
We assume familiarity with the notion of operad as presented for example in \cite{may1997operads} or \cite{loday2012operads}.

\subsection{\pdfC-modules}

A \textit{$\cyc$-module} $\cP$ is a set $\{\cP(r)\}_{r>0}$ with $\cP(r)$ a chain complex with a right action of $\cyc_r$.
A \textit{$\cyc$-equivariant map} $\phi \colon \cP_1 \to \cP_2$ is a set $\set{\phi^r \colon \cP_1(r) \to \cP_2(r)}_{r>0}$ with each $\phi^r$ a $\cyc_r$-equivariant map.
We say $\phi$ is a chain map or a quasi-isomorphism if each $\phi(r)$ is one.
A homotopy is a collection $\set{\cP_1(r) \to \cP_2(r)}_{r>0}$ of degree $1$ maps.
%When convenient, we will abuse notation and denote $\phi(r)$ simply as $\phi$.
There are evident forgetful functors to the category of $\cyc$-modules from those of $\sym$-modules and operads, which we will use without further notice.

\subsection{Minimal cyclic resolution}

By $\cW(r)$ we denote the minimal free resolution of $\Z$ by $\Z[\cyc_r]$-modules
\[
\Z[\cyc_r]\{e_0\} \xla{T} \Z[\cyc_r]\{e_1\} \xla{N} \Z[\cyc_r]\{e_2\} \xla{T} \cdots
\]
where
\[
\begin{split}
	T(e_{2i+1}) &= (\rho-1)e_{2i}, \\
	N(e_{2i}) &= (1 + \rho + \cdots + \rho^{r-1})(e_{2i-1}),
\end{split}
\]
for any $i \in \N$.
The complex $\cW(r)$ is equipped with the structure of a $\cyc_r$-equivariant counital coalgebra
\[
\varepsilon \colon \cW(r) \to \Z, \qquad
\Delta \colon \cW(r) \to \cW(r) \ot \cW(r)
\]
where $\Z$ is acted on trivially by $\cyc_r$ and $\cW(r) \ot \cW(r)$ diagonally.
These maps are defined by $\varepsilon(e_0) = 1$ and
\begin{align*}
	\Delta(e_{2i}) &=
	\sum_{\ \mathclap{i=j+k}} e_{2j} \ot e_{2k} \ + \!\! \sum_{i-1=j+k} \ \sum_{0 \leq s < t < r} \rho^s e_{2j+1} \ot \rho^t e_{2k+1}, \\
	\Delta(e_{2i+1}) &=
	\sum_{\ \mathclap{i=j+k}} e_{2j} \ot e_{2k+1} + e_{2j+1} \ot \rho e_{2k}.
\end{align*}

We denote the set $\set{\cW(r)}_{r > 0}$  by $\cW$ regarded as a $\cyc$-module by $(\rho^s e_i) \cdot \rho = \rho^{s+1} e_i$.

\subsection{May--Steenrod structures}

Recall that an $E_\infty$-algebra structure on a chain complex $A$ is an operad morphism to $\End(A) = \{\Hom(A^{\ot r}, A)\}_{r>0}$ from an operad $\cR$ such that, for each $r > 0$, the following holds: \textit{i}) the chain complex $\cR(r)$ has the homology of a point, and \textit{ii}) its $\sym_r$-action is free.

A \textit{May--Steenrod structure} on an chain complex $A$ is an $E_\infty$-algebra structure $\phi \colon \cR \to \End(A)$ on $A$ together with a $\cyc$-equivariant quasi-isomorphism $\iota \colon \cW \to \cR$.
We write $\psi$ for $\phi \comp \iota$ and abusively use it to denote the May--Steenrod structure.
We refer to a chain complex with a May--Steenrod structure as a \textit{May--Steenrod complex}.

A May--Steenrod complex $(A,\psi)$ has a preferred product $\psi^2(e_0) \colon A \ot A \to A$ which we denote for elements $a, b \in A$ simply as $a \cp b$.
Generalizing this, we refer to $\psi^r(e_i) \colon A^{\ot r} \to A$ as the \textit{cup-$(r,i)$ product}, denoting it $\psi_i^r$.

\subsection{Homology operations}\label{ss:homology_operations}

Let $(A,\psi)$ be a May--Steenrod complex.
After extending scalars to $\Fp$ the assignment sending $a$ to $\psi_i(a^{\ot p})$ sends cycles to cycles and we denote its induced linear map in homology by
\[
\rD_i^p \colon \rH_n(A; \Fp) \to \rH_{np+i}(A; \Fp).\footnote{
	The facts that $\rD_i^p$ is well-defined on homology and linear are rather non-trivial consequences of the fact that $\psi \colon \cW(p) \to \Hom(A^{\ot p}, A)$ is a $\cyc_p$-equivariant chain map.
	Thus the adjoint $\Psi \colon \cW(p) \ot A^{\ot p} \to A$ is equivariant, which means it factors through the coinvariants $\cW(p) \ot_\Cp A^{\ot p}$.
	It is also important that $\rho$ is an even permutation, so	$a^{\ot p}$ is $\Cp$-invariant regardless of the degree of $a$.
	Thus by passing to coinvariants and using the boundary formulas for the $e_i$ one sees that $\rD_i^p(a) = \Psi(e_i \ot_\Cp a^{\ot p})$ is a cycle if $a$ is a cycle.
	That $\rD_i^p$ takes boundaries to boundaries and is linear on homology requires further non-trivial arguments, involving $(\bd a)^{\ot p}$ and $(a + b)^{\ot p}$.}
\]
We will focus on these homology operations rather than the operations $P_s$ and $\beta P_s$ which we define below.
An essential statement that underpins their definition was attributed to Thom by Steenrod in \cite[Theorem~4.8]{steenrod1953cyclic}.

\begin{theorem}[Thom]\label{t:thom}
	The map $\rD_i^p$ is $0$ unless for some $\varepsilon \in \{0,1\}$ either: $n$ is even and $i+\varepsilon$ is an even multiple of $(p-1)$ or $n$ is odd and $i+\varepsilon$ is an odd multiple of $(p-1)$.
\end{theorem}

For completeness, in \cref{s:thom} we will prove Thom's theorem effectively by providing explicit coboundary formulas expressing the vanishing of the indicated cohomology classes.

With this in mind, one defines the Steenrod operations
\begin{align*}
	P_s \colon& H_\bullet(A; \Fp) \to H_{\bullet + 2s(p-1)}(A; \Fp), \\
	\beta P_s \colon& H_\bullet(A; \Fp) \to H_{\bullet + 2s(p-1) - 1}(A; \Fp),
\end{align*}
by sending the class represented by a cycle $a \in A \ot \Fp$ of degree $n$ to the classes represented respectively for $\varepsilon \in \{0,1\}$ by
\begin{equation*}
	(-1)^s \nu(n) \, \rD^p_{(2s-n)(p-1)-\varepsilon}[a] \,,
\end{equation*}
where $\nu(n) = (-1)^{n(n-1)m/2}(m!)^n$ and $m = \frac{p-1}{2}$.\footnote{
A note about degrees:
The results of this paper apply to arbitrary $E_\infty$-algebras.
But the subscripts $i$ of the $\rD_i$ cannot be negative, so for a class a of degree $n$ the $P_s$ can be defined only for $2s-n \geq 0$, that is, $n/2 \leq s$.
In the classical case of cochains on simplicial sets, all classes have homological degrees $n \leq 0$.
Naturality constraints imply that no natural operation can raise these negative degrees.
Therefore, non-zero operations $P_s$ on simplicial sets are constrained by $s \leq 0$.
Thus, with the homological grading, non-zero operations $P_s(a)$ on simplicial set classes $a$ of degree $n \leq 0$ are constrained by two conditions $n/2 \leq s \leq 0$.
Historically, of course, cohomology was graded positively, and the Steenrod operations $P^s$ with $s \geq 0$ raised degree by $2s(p-1)$.
}
The notation $\beta P_s$ is motivated by the Bockstein relation $\beta P_s = \beta\, \comp P_s$, which is easily verified using the boundary formula in $\cW(p)$ with $\Z$ coefficients,
\[
\bd(e_{2i}) = (1 + \rho + \dots + \rho^{p-1})e_{2i-1}.
\]
The motivation for the constants $\nu(n)$ will be discussed in \cref{fn:mu} following the statement of the Cartan relations in \cref{s:cartan}.