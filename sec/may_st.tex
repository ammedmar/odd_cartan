% !TEX root = ../odd_cartan.tex

\section{May--Steenrod structures}

We now present a summary of \cite{medina2021may_st}, which makes explicit the definitions of \cite{steenrod1953cyclic} from the general viewpoint of \cite{may1970general}.

\subsection{Minimal cyclic resolution}

Let $\cyc_r$ denote the cyclic group of order $r$ thought of as the subgroup of the symmetric group $\sym_r$ generated by $\rho = (1,2,\dots,r)$.
When convenient we will simplify our notation using the bijection
\[
\{0,\dots,r-1\} \ni i \mapsto \rho^i \in \cyc_r.
\]

By $\cW(r)$ we denote the minimal free resolution of $\Z$ by $\Z[\cyc_r]$-modules
\begin{equation}\label{eq: minimal resolution}
	\Z[\cyc_r]\{e_0\} \xla{T} \Z[\cyc_r]\{e_1\} \xla{N} \Z[\cyc_r]\{e_2\} \xla{T} \cdots
\end{equation}
where
\begin{equation} \label{eq: T and R definition}
	\begin{split}
		T &= \rho - 1, \\
		N &= 1 + \rho + \cdots + \rho^{r-1}.
	\end{split}
\end{equation}
It is equipped with the structure of a coalgebra in the category of $\cyc_r$-equivariant chain complexes, defined explicitly by
\begin{align*}
	\varepsilon(e_0) &= 1, \\
	\Delta(e_{2i}) &=
	\sum_{i=j+k} e_{2j} \ot e_{2k} \ + \sum_{i-1=j+k} \ \sum_{0 \leq r \leq s \leq p} \rho^r e_{2j+1} \ot \rho^s e_{2k+1}, \\
	\Delta(e_{2i+1}) &=
	\sum_{i=j+k} e_{2j} \ot e_{2k+1} \ +\ e_{2j+1} \ot \rho e_{2k}.
\end{align*}

%Let $\Fp(q)$ \TBW.
%It follows from a straightforward computation that for any prime $p$ and integer $q$
%\begin{equation*}
%	H_i(\cyc_p; \Fp(q)) = \Fp.
%\end{equation*}

\subsection{Cyclic modules}

Similarly to how a group can be regarded as category, we assemble the cyclic groups into a groupoid $\cyc$ whose set of objects is $\N$ and morphisms are given by
\[
\cyc(r,s) =
\begin{cases}
	\cyc_r & r = s, \\
	\hfil \emptyset & r \neq s.
\end{cases}
\]

The category of $\cyc$-modules is that of contravariant functors from $\cyc$ to the category of chain complexes.
Explicitly, a $\cyc$-module $\cP$ is the data of a chain complex $\cP(r)$ with a right $\cyc_r$-action for each $r > 0$, and a morphism is given by a collection of equivariant maps.
The main example of a $\cyc$-module is given by $\cW = \{\cW(r)\}_{r>0}$.
We say that a $\cyc$-module morphisms is a chain complex or a quasi-isomorphism if each of the maps defining it is.
The definition of $\sym$-modules is similar.
Notice that an $\sym$-module is canonically a $\cyc$-module, and that (algebraic) operads are examples.

\subsection{May--Steenrod structures}

A \textit{May--Steenrod structure} on an chain complex $A$ is an $E_\infty$-structure $\phi \colon \cR \to \End(A)$ on $A$ together with a $\cyc$-module quasi-isomorphism $\iota \colon \cW \to \cR$.
We write $\psi$ for $\phi \circ \iota$ and abusively use it to denote the May--Steenrod structure.
We refer to a chain complex with a May--Steenrod structure as a \textit{May--Steenrod complex}.

A May--Steenrod complex $(A,\psi)$ has a preferred product $\psi(2)(e_0) \colon A \ot A \to A$ which we denote for elements $a, b \in A$ simply as $a \smallsmile b$.
Generalizing this, we refer to $\psi(r)(e_i) \colon A^{\ot r} \to A$ as the \textit{cup-$(r,i)$ product} denoting it $\psi_i^r$ or simply $\psi_i$ when $r$ is clear from the context.

\subsection{Homology operations}

Let $(A,\psi)$ be a May--Steenrod complex.
After extending scalars to $\Fp$ the assignment sending $a$ to $\psi_i(a^{\ot p})$ is linear and maps cycles to cycles.
We denote its induced map on the mod $p$ homology of $A$ by
\begin{equation}\label{e:D_i^p}
	\rD_i^p \colon \rH_n(A; \Fp) \to \rH_{np+i}(A; \Fp).
\end{equation}
We will emphasize these homology operations as opposed to the operations $P_s$ and $\beta P_s$ that we now define.
The map $\rD_i^p$ in \eqref{e:D_i^p} is $0$ unless $n$ is even (resp. odd) and $i+\varepsilon$ is an even (resp. odd) multiple of $(p-1)$ for some $\varepsilon \in \{0,1\}$.
Please consult \cite[Proposition 2.3. (iv)]{may1970general} for a proof.
With this in mind one defines the \textit{Steenrod operations}
\begin{align*}
	P_s \colon& H_\bullet(A; \Fp) \to H_{\bullet + 2s(p-1)}(A; \Fp), \\
	\beta P_s \colon& H_\bullet(A; \Fp) \to H_{\bullet + 2s(p-1) - 1}(A; \Fp),
\end{align*}
by sending the class represented by a cycle $a \in A \ot \Fp$ of degree $n$ to the classes represented respectively for $\varepsilon \in \{0,1\}$ by
\begin{equation*}
	(-1)^s \nu(n) \, \rD^p_{(2s-n)(p-1)-\varepsilon}[a]
\end{equation*}
where $\nu(n) = (-1)^{n(n-1)m/2}(m!)^n$ and $m = \floor{p/2} = (p-1)/2$.
The constant $\nu(n)$ and notation $\beta P_s$ are motivated respectively by the unstable $\rD_{n(p-1)}^p[\alpha] = \nu(n)[\alpha]$ and Bockstein relations $\beta P_s = \beta \circ P_s$ respectively.
The first holding in the cohomology of spaces and the second in general.