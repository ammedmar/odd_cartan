% !TEX root = ../odd_cartan.tex

\section{May--Steenrod structures}\label{s:may_st}

We now present a summary of \cite{medina2021may_st}, which makes explicit the definitions of \cite{steenrod1953cyclic} from the general viewpoint of \cite{may1970general}.
We assume familiarity with the notion of operad as presented for example in \cite{may1997operads} or \cite{loday2012operads}.

\subsection{\pdfC-modules}

For $r > 0$, we denote by $\cyc_r$ the cyclic group of order $r$ thought of as the subgroup of the symmetric group $\sym_r$ generated by the cycle permutation $\rho = (1,2,\dots,r)$.
When convenient, we will simplify our notation using the bijection
\[
\begin{tikzcd}[column sep=small,row sep=-5]
	\{0,\dots,r-1\} \rar["\cong\ "] & \cyc_r \\
	\qquad i \rar[maps to] & \rho^i.
\end{tikzcd}
\]
A \textit{$\cyc$-module} $\cP$ is a set $\{\cP(r)\}_{r>0}$ with $\cP(r)$ a chain complex with a right action of $\cyc_r$.
A \textit{$\cyc$-equivariant map} $\phi \colon \cP_1 \to \cP_2$ is a set $\set{\phi(r) \colon \cP_1(r) \to \cP_2(r)}_{r>0}$ with $\phi(r)$ a $\cyc_r$-equivariant map.
We say $f$ is a chain map or a quasi-isomorphism if each $\phi(r)$ is.
A similar convention is applied for chain homotopies.
When convenient we will abuse notation and denote $\phi(r)$ simply as $\phi$.
%$\sym$-modules and $\sym$-equivariant maps are defined similarly.
There are evident forgetful functors to the category of $\cyc$-modules from those of $\sym$-modules and operads, which we will use without further notice.

\subsection{Minimal cyclic resolution}

By $\cW(r)$ we denote the minimal free resolution of $\Z$ by $\Z[\cyc_r]$-modules
\begin{equation}\label{eq: minimal resolution}
	\Z[\cyc_r]\{e_0\} \xla{T} \Z[\cyc_r]\{e_1\} \xla{N} \Z[\cyc_r]\{e_2\} \xla{T} \cdots
\end{equation}
where
\begin{equation} \label{eq: T and R definition}
	\begin{split}
		T &= \rho - 1, \\
		N &= 1 + \rho + \cdots + \rho^{r-1}.
	\end{split}
\end{equation}
It is equipped with the structure of a $\cyc_r$-equivariant coalgebra defined by:
\begin{align*}
	\varepsilon(e_0) &= 1, \\
	\Delta(e_{2i}) &=
	\sum_{i=j+k} e_{2j} \ot e_{2k} \ + \! \sum_{i-1=j+k} \ \sum_{0 \leq s < t < r} \rho^s e_{2j+1} \ot \rho^t e_{2k+1}, \\
	\Delta(e_{2i+1}) &=
	\sum_{i=j+k} e_{2j} \ot e_{2k+1} \ +\ e_{2j+1} \ot \rho e_{2k},
\end{align*}
where $\Z$ is acted on trivially and $\cW(r) \ot \cW(r)$ diagonally.
We denote by $\cW$ the $\cyc$-module defined by $\set{\cW(r)}_{r > 0}$ which is a coalgebra in the category of $\cyc$-modules.

\subsection{May--Steenrod structures}

Recall that an $E_\infty$-algebra structure on a chain complex $A$ is an operad morphism to $\End(A) = \{\Hom(A^{\ot r}, A)\}_{r>0}$ from an operad $\cR$ such that, for each $r > 0$, the chain complex $\cR(r)$ is contractible and its $\sym_r$-action is free.

A \textit{May--Steenrod structure} on an chain complex $A$ is an $E_\infty$-structure $\phi \colon \cR \to \End(A)$ on $A$ together with a $\cyc$-equivariant quasi-isomorphism $\iota \colon \cW \to \cR$.
We write $\psi$ for $\phi \circ \iota$ and abusively use it to denote the May--Steenrod structure.
We refer to a chain complex with a May--Steenrod structure as a \textit{May--Steenrod complex}.

A May--Steenrod complex $(A,\psi)$ has a preferred product $\psi(2)(e_0) \colon A \ot A \to A$ which we denote for elements $a, b \in A$ simply as $a \cp b$.
Generalizing this, we refer to $\psi(r)(e_i) \colon A^{\ot r} \to A$ as the \textit{cup-$(r,i)$ product}, denoting it $\psi_i^r$ or simply $\psi_i$ when $r$ is clear from the context.

\subsection{Homology operations}\label{ss:homology_operations}

Let $(A,\psi)$ be a May--Steenrod complex.
After extending scalars to $\Fp$ the assignment sending $a$ to $\psi_i(a^{\ot p})$ sends cycles to cycles and induces a linear map in homology.
We denote its induced map on the mod $p$ homology of $A$ by
\begin{equation}\label{e:D_i^p}
	\rD_i^p \colon \rH_n(A; \Fp) \to \rH_{np+i}(A; \Fp).
\end{equation}
We will focus on these homology operations rather than the operations $P_s$ and $\beta P_s$ which we define below.
An essential observation that underpins their definition is attributed by Steenrod to Thom \cite[Theorem~4.8]{steenrod1947products}.
For completeness, we will provide an effective proof of this observation in \cref{s:thom}.
%In the notation we use, it is expressed as follows.
\begin{theorem*}
	The map $\rD_i^p$ in \eqref{e:D_i^p} is $0$ unless $n$ is even (resp. odd) and $i+\varepsilon$ is an even (resp. odd) multiple of $(p-1)$ for some $\varepsilon \in \{0,1\}$.
\end{theorem*}
%Please consult \cite[Proposition 2.3. (iv)]{may1970general} for a proof.
With this in mind, one defines the \textit{Steenrod operations}
\begin{align*}
	P_s \colon& H_\bullet(A; \Fp) \to H_{\bullet + 2s(p-1)}(A; \Fp), \\
	\beta P_s \colon& H_\bullet(A; \Fp) \to H_{\bullet + 2s(p-1) - 1}(A; \Fp),
\end{align*}
by sending the class represented by a cycle $a \in A \ot \Fp$ of degree $n$ to the classes represented respectively for $\varepsilon \in \{0,1\}$ by
\begin{equation*}
	(-1)^s \nu(n) \, \rD^p_{(2s-n)(p-1)-\varepsilon}[a]
\end{equation*}
where $\nu(n) = (-1)^{n(n-1)m/2}(m!)^n$ and $m = \floor{p/2} = (p-1)/2$.
The constant $\nu(n)$ and notation $\beta P_s$ are motivated respectively by the unstable $\rD_{n(p-1)}^p[\alpha] = \nu(n)[\alpha]$ and Bockstein relations $\beta P_s = \beta \circ P_s$ respectively.
The first holding in the cohomology of spaces and the second in general.