% !TEX root = ../odd_cartan.tex

\section{Preliminaries}\label{s:preliminaries}

\subsection{Conventions}

Throughout this work $p$ denotes an odd prime and $\Fp$ the field with $p$ elements.
We assume the ground ring to be the integers $\Z$ unless stated otherwise.
We use homological grading for chain complexes.
Differentials always have degree $-1$.
The linear dual of a chain group of degree $d$ is placed in degree $-d$.
In particular, for simplicial sets, cochains and cohomology are	indexed by non-positive integers.
As usual, a homotopy from $h_0$ to $h_1$ is a linear map $h$ of degree~$1$ such that $\bd h = \bd \comp \,h + h \comp \bd = h_1 - h_0$.

We denote by $\cyc_r$ the cyclic group of order $r$ thought of as the subgroup of the symmetric group $\sym_r$ generated by the cycle permutation $\rho = (1,2,\dots,r)$.
When convenient, we will simplify our notation using the bijection
\[
\begin{tikzcd}[column sep=small,row sep=-5]
	\{0,\dots,r-1\} \rar & \cyc_r \\
	\qquad i \rar[maps to] & \rho^i.
\end{tikzcd}
\]

\subsection{The shuffle $\tau$}\label{ss:shuffle}

For any $r > 0$, the element $\tau_r \in \sym_{2r}$ is the shuffle permutation mapping the first and second ``decks'' to odd and even integers respectively.
Explicitly, for $\ell \in \{1,\dots,2r\}$ we have
\begin{equation*}
	\tau_r(\ell) =
	\begin{cases}
		2\ell-1 & \ell \leq r, \\
		2(\ell-r) & \ell > r.
	\end{cases}
\end{equation*}
When $r$ is clear from the context we will omit it from the notation.
We will use that for any two elements $a$ and $b$ in a graded module, the Koszul sign convention implies that
\[
\tau(a^{\ot r} \ot b^{\ot r}) =
(-1)^{\frac{(r-1)r}{2}\bars{a}\bars{b}} \, (a \ot b)^{\ot r}.
\]

\subsection{The homomorphisms $f$ and $g$}\label{ss:f and g}

For $r > 0$, let $f,g \colon \cyc_r \to \sym_{2r}$ be the group homomorphisms:
\[
\begin{split}
	&f \colon \cyc_r \hookrightarrow \sym_r \xra{\id \times e^r} \sym_r \times \sym_2^r \xra{\bcomp_{\sym}} \sym_{2r}\,, \\
	&g \colon \cyc_r \hookrightarrow \sym_r \xra{e \times \rD} \sym_2 \times \sym_r \times \sym_r \xra{\bcomp_{\sym}} \sym_{2r}\,,
\end{split}
\]
where $e \in \sym_2$ denotes the identity element, $\rD$ the diagonal homomorphism, and $\bcomp_{\sym}$ the usual block permutation map.
Explicitly,
\begin{align*}
	f(\rho) &= (1,3,\dots,2r-1)(2,4,\dots,2r), \\
	g(\rho) &= (1,2,\dots,r)(r+1,r+2,\dots,2r).
\end{align*}
We remark that $f$ and $g$ are group homomorphisms despite $\bcomp_\sym$ not being one.

\begin{lemma}\label{l:conjugated}
	These homomorphisms are conjugated by the shuffle $\tau$ of {\rm \cref{ss:shuffle}}, i.e.,
	\[
	f \comp \tau = \tau \comp g.
	\]
\end{lemma}

\begin{proof}
	We need to show that the permutations $f(\rho) \tau$ and $\tau g(\rho)$ agree.
	It can be directly checked that their action on $\ell \in \set{1,\dots,2r}$ is given by
	\[
	\ell \mapsto
	\begin{cases}
		2\ell+1 & \ell<r,\\
		1 & \ell = r,\\
		2(\ell-r)+2 & r < \ell <2r,\\
		2 & \ell = 2r.
	\end{cases}\qedhere
	\]
\end{proof}

\subsection{Eilenberg--Zilber contraction}

The functor of (normalized) chains $\chains \colon \sSet \to \Ch$ from simplicial sets to chain complexes does not preserve products, but there are natural chain maps
\[
\begin{tikzcd}
	\chains(X \times Y) \arrow[r,shift left=2pt,"\AW"] &
	\chains(X) \ot \chains(Y) \arrow[l,shift left=2pt,"\EZ"]
\end{tikzcd}
\]
and a natural homotopy
\[
\Shi \colon \chains(X \times Y) \to \chains(X \times Y)
\]
satisfying
\[
\id - \AW \comp \EZ = 0, \qquad
\id - \EZ \comp \AW = \bd \comp \Shi + \Shi \comp \bd.
\]
Closed form formulas for these maps can be found for example in \cite[56]{real2000homological}.
They are given by sums of compositions of face and degeneracy maps on each factor.

Let $\rD \colon X \to X \times X$ be the diagonal map of simplicial sets.
We denote the compositions $\AW \comp \chains(\rD)$ by $\Delta_{\AW}$.