% !TEX root = ../odd_cartan.tex

\section{Preliminaries}\label{s:preliminaries}

\subsection{Conventions}

Throughout this work $p$ denotes an odd prime and $\Fp$ the field with $p$ elements.
%We use $G$ to denote an arbitrary group.
We assume the ground ring to be $\Z$ unless stated otherwise.
We use homological grading, with the linear dual of an element in degree $d$ placed in degree $-d$.

We denote by $\cyc_r$ the cyclic group of order $r$ thought of as the subgroup of the symmetric group $\sym_r$ generated by the cycle permutation $\rho = (1,2,\dots,r)$.
When convenient, we will simplify our notation using the bijection
\[
\begin{tikzcd}[column sep=small,row sep=-5]
	\{0,\dots,r-1\} \rar & \cyc_r \\
	\qquad i \rar[maps to] & \rho^i.
\end{tikzcd}
\]

\subsection{The shuffle $\tau$}\label{ss:shuffle}

For any $r \in \N$, the element $\tau_r \in \sym_{2r}$ is the shuffle permutation mapping the first and second ``decks'' to odd and even integers respectively.
Explicitly, for $\ell \in \{1,\dots,2r\}$ we have
\begin{equation*}
	\tau_r(\ell) =
	\begin{cases}
		2\ell-1 & \ell \leq r, \\
		2(\ell-r) & \ell > r.
	\end{cases}
\end{equation*}
When $r$ is clear from the context we will omit it from the notation.
Please notice that in a graded module we have
\[
\tau(a^{\ot r} \ot b^{\ot r}) =
(-1)^{\frac{(r-1)r}{2}\bars{a}\bars{b}} \, (a \ot b)^{\ot r}.
\]

\subsection{The homomorphisms $f$ and $g$}\label{ss:f and g}

For $r \in \N$, let $f,g \colon \cyc_r \to \sym_{2r}$ be the group homomorphisms:
\[
\begin{split}
	&f \colon \cyc_r \hookrightarrow \sym_r \xra{\id \times e^r} \sym_r \times \sym_2^r \xra{\bcirc_{\sym}} \sym_{2r}\,, \\
	&g \colon \cyc_r \hookrightarrow \sym_r \xra{e \times \rD} \sym_2 \times \sym_r \times \sym_r \xra{\bcirc_{\sym}} \sym_{2r}\,,
\end{split}
\]
where $e$ denotes the identity element.
Explicitly,
\begin{align*}
	f(\rho) &= (1,3,\dots,2r-1)(2,4,\dots,2r), \\
	g(\rho) &= (1,2,\dots,r)(r+1,r+2,\dots,2r).
\end{align*}
%where $\rho = (1,2,\dots,r)$.

\begin{lemma}
	These homomorphisms are conjugated by the shuffle $\tau$ of \cref{ss:shuffle}, i.e.,
	\[
	f(\rho) \tau = \tau g(\rho).
	\]
\end{lemma}

\subsection{Eilenberg--Zilber contraction}

The functor of (normalized) chains $\chains \colon \sSet \to \Ch$ from simplicial sets to chain complexes does not preserve products, but there are natural chain maps
\[
\begin{tikzcd}
	\chains(X \times Y) \arrow[r,shift left=2pt,"\AW"] &
	\chains(X) \ot \chains(Y) \arrow[l,shift left=2pt,"\EZ"]
\end{tikzcd}
\]
and a natural linear map
\[
\Shi \colon \chains(X \times X) \to \chains(X \times X)
\]
satisfying
\[
\id - \AW \circ \EZ = 0, \qquad
\id - \EZ \circ \AW = \bd \circ \Shi + \Shi \circ \bd.
\]
Closed form formulas describing these maps can be found for example in \cite[56]{real2000homological}.

Let $\rD \colon X \to X \times X$ be the diagonal map of simplicial sets.
We denote the compositions $\AW \circ \chains\rD$ by $\Delta_{\AW}$.