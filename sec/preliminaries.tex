% !TEX root = ../odd_cartan.tex

\section{Preliminaries}\label{s:preliminaries}

Throughout this work $p$ will denote an odd prime and $\Fp$ the field with $p$ elements.
We use $G$ to denote an arbitrary group.
We will assume the ground ring to be $\Z$ unless stated otherwise.
We use homological grading, with the linear dual of an element in degree $d$ placed in degree $-d$.

\subsection{Eilenberg--Zilber contraction}

The functor of (normalized) chains $\chains \colon \sSet \to \Ch$ from simplicial sets to chain complexes does not preserve products, but there are natural chain maps
\[
\begin{tikzcd}
	\chains(X \times Y) \arrow[r,shift left=2pt,"\AW"] &
	\chains(X) \ot \chains(Y) \arrow[l,shift left=2pt,"\EZ"]
\end{tikzcd}
\]
and a natural linear map
\[
\Shi \colon \chains(X \times X) \to \chains(X \times X)
\]
satisfying
\[
\EZ \circ \AW = \id, \qquad
\AW \circ \EZ = \id + \bd \circ \Shi + \Shi \circ \bd.
\]
Closed form formulas describing these maps can be found for example in \cite[56]{real2000homological}.

Let $\rD \colon X \to X \times X$ be the diagonal map of simplicial sets.
We denote the compositions $\AW \circ \chains\rD$ by $\Delta_{\AW}$.

\subsection{The basic reordering}\label{ss:reordering}

For any $r \in \N$, the element $\tau_r \in \sym_{2r}$ is the shuffle permutation mapping the first and second ``decks'' to odd and even integers respectively.
Explicitly, for $\ell \in \{1,\dots,2r\}$ we have
\begin{equation*}
	\tau_r(\ell) =
	\begin{cases}
		2\ell-1 & \ell \leq r, \\
		2(\ell-r) & \ell > r.
	\end{cases}
\end{equation*}
When $r$ is clear from the context we will omit it from the notation.
Please notice that in a graded module we have
\[
\tau(a^{\ot r} \ot b^{\ot r}) =
(-1)^{\frac{(r-1)r}{2}\bars{a}\bars{b}} \, (a \ot b)^{\ot r}.
\]