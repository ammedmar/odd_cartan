% !TEX root = ../odd_cartan.tex

\section{Preliminaries}

Throughout this work $p$ will denote an odd prime, $\Fp$ the field with $p$ elements, and $\k$ a commutative ring, typically either $\Z$ or $\Fp$.
Let us assume the ground ring to be $\Z$ unless stated otherwise.
We use homological grading, with the linear dual of an element in degree $d$ placed in degree $-d$.

\subsection{Eilenberg--Zilber contraction}

\TBW

\subsection{Barratt--Eccles operad}

We review the main constructions of \cite{berger2004combinatorial}.

\sssec

Let $\Set$ and $\sSet$ denote the categories of sets and simplicial sets respectively.
The symmetric monoidal functor $\EE \colon \Set \to \sSet$ is defined on objects by $\EE X_n = X^{n+1}$ with
\[
\begin{split}
	\face_i(x_0,\dots,x_n) &= (x_0,\dots,\widehat{x}_i,\dots,x_n), \\
	\dege_i(x_0,\dots,x_n) &= (x_0, \dots, x_i, x_i, \dots, x_n),
\end{split}
\]
and on morphisms by $\EE f_n = f^{\times n}$.

We notice that if $X$ is equipped with a group action $X \times G \to X$ then $\EE X$ is as well with
\[
(x_0,\dots,x_n) \cdot g = (x_0 \cdot g, \dots, x_n \cdot g).
\]

\sssec

Let $\sym_r$ be the symmetric group and
\[
\bcirc_{\sym} \colon \sym_r \times \sym_{s_1} \times \cdots \times \sym_{s_r} \to \sym_{s_1+\dots+s_r}
\]
the usual block permutation map, which provides $\sym = \{\sym_r\}_{r>0}$ with the structure of an operad in $\Set$.

\sssec

The simplicial Barratt--Eccles operad $\EE\sym$ is obtained by applying the functor $\EE$ to the operad $\sym$.
We denote $\EE\sym(r) = \EE\sym_r$ simply by $\EE(r)$ and its compositions by
\[
\bcirc_{\EE} \colon \EE(r) \times \EE(s_1) \times\dots\times \EE(s_r) \to \EE(s_1+\dots+s_r).
\]

\sssec

Let $\chains \colon \sSet \to \Ch$ be the functor of (normalized) chains with integer coefficients.
The linear Barratt--Eccles operad $\cE$ is defined in arity $r$ by $\cE(r) = \chains\EE(r)$ with compositions,
\[
\bcirc_{\cE} \colon \cE(r) \ot \cE(s_1) \ot\dotsb\ot \cE(s_r) \to \cE(s_1+\dots+s_r)
\]
defined by $\bcirc_{\cE} = \bcirc_{\EE} \circ \EZ$.

\sssec

The table reduction morphism \TBW

\subsection{May--Steenrod cup-$(p,i)$ products}

We now present a summary of \cite{medina2021may_st}, which makes explicit the definitions of \cite{steenrod1953cyclic} from the general viewpoint of \cite{may1970general}.

\sssec\label{sss:minimal resolution}

Let $\cyc_r$ denote the cyclic group of order $r$ thought of as a subgroup of $\sym_r$ with a preferred generator $\rho$.
Let $\cW(r)$ by the non-negatively graded chain complex
\begin{equation} \label{eq: minimal resolution}
	\k[\cyc_r]\{e_0\} \xla{T} \k[\cyc_r]\{e_1\} \xla{N} \k[\cyc_r]\{e_2\} \xla{T} \cdots
\end{equation}
where
\begin{equation} \label{eq: T and R definition}
	\begin{split}
		T &= \rho - 1, \\
		N &= 1 + \rho + \cdots + \rho^{r-1}.
	\end{split}
\end{equation}
This complex is equipped with a counital coproduct given by
\begin{align*}
	\Delta(e_{2i}) &=
	\sum_{i=j+k} e_{2j} \ot e_{2k} \ + \sum_{i-1=j+k} \ \sum_{0 \leq r < s \leq p} \rho^r e_{2j+1} \ot \rho^s e_{2k+1}, \\
	\Delta(e_{2i+1}) &=
	\sum_{i=j+k} e_{2j} \ot e_{2k+1} \ +\ e_{2j+1} \ot \rho e_{2k}.
\end{align*}

\sssec

Let $\Fp(q)$ \TBW.
It follows from a straightforward computation that for any prime $p$ and integer $q$
\begin{equation*}
	H_i(\cyc_p; \Fp(q)) = \Fp.
\end{equation*}

\sssec

A May--Steenrod structure on an operad $\cO$ is a collection
\[
\set[\big]{\psi(r) \colon \cW(r) \to \cO(r)}_{r>0}
\]
with each $\psi(r)$ a chain map of $\Z[\cyc_r]$-modules factoring through the arity $r$ part of an $E_\infty$-operad
\[
\psi(r) \colon \cW(r) \xra{\iota(r)} \cR(r) \xra{\phi(r)} \cO(r)
\]
such that $\iota(r)$ is a quasi-isomorphism and $\{\phi(r)\}_{r>0}$ is a morphism of operads.
When $r$ is clear from the context we will omit it from the notation.

\sssec

Let $\cO$ be endowed with a May--Steenrod structure $\{\psi(r)\}_{r>0}$.
Given an $\cO$-algebra structure $\cO \to \End(A)$ on a chain complex $A$, its cup-$(r,i)$ product
\[
\psi_i^r \colon A^{\ot r} \to A
\]
is the image of $\psi(r)(e_i)$ in $\Hom(A^{\ot r})$. %where $e_i$ is the basis of $\cW(r)_i$.
When $r$ is clear from the context we will omit it from the notation.

\sssec

We denote the cup-$(2,0)$ of $A$ acting on an element $a \ot b$ simply as $a \smallsmile b$.
It induces a commutative coassociative coalgebra structure on the homology of $A$.
Explicitly, this product is defined by
\[
[a][b] \defeq \big[a \smallsmile b\big].
\]

\sssec

When the ground ring is $\Fp$, the assignment sending $a$ to $\psi_i^p(a^{\ot p})$ is linear and maps cycles to cycles.
We denote the induced map on homology by $\rD_i^p$.
Explicitly, it is given by
\[
\rD_i^p[a] \defeq \big[\psi_i^p(a \ot\dotsb\ot a)\big].
\]

\sssec

For an integer $s$, the Steenrod operations
\begin{align*}
	P_s \colon& H_\bullet(A; \mathbb{F}_p) \to H_{\bullet + 2s(p-1)}(A; \mathbb{F}_p), \\
	\beta P_s \colon& H_\bullet(A; \mathbb{F}_p) \to H_{\bullet + 2s(p-1) - 1}(A; \mathbb{F}_p),
\end{align*}
are defined by sending the class represented by a cycle $a \in (A \otimes \mathbb{F}_p)$ of degree $q$ to the classes represented respectively for $\varepsilon \in\{0,1\}$ by
\begin{equation*}
	(-1)^s \nu(q) \rD^p_{(2s-q)(p-1)-\varepsilon}[a]
\end{equation*}
where $\nu(q) = (-1)^{q(q-1)m/2}(m!)^q$ and $m = \floor{p/2} = (p-1)/2$.

\begin{remark*}[{\cite[(6.1)]{steenrod1953cyclic}}]
	For a degree $q$ cochain $a$ of a space one has the identity $\rD_{q(p-1)}^p[a] = \nu(q)[a]$.
\end{remark*}

\begin{remark*}
	The notation $\beta P_s$ is motivated by the relationship of this operator and the Bockstein of the reduction $\Z \to \Z/p\Z$.
\end{remark*}