% !TEX root = ../odd_cartan.tex

\section{Cartan relation at the chain level}\label{s:cartan}

\subsection{The Cartan relation}

We recall two presentation of the Cartan relation in the mod $p$ homology of an $E_\infty$-algebra.

\sssec

For any integer $s$:
\begin{align}
	\rP_s\big([a][b]\big) =&
	\sum_{i+j=s} \rP_i[a] \, \rP_j[b] \\
	\beta\rP_s\big([a][b]\big) =&
	\sum_{i+j=s} \beta\rP_{i+1}[a] \, \rP_j[b] \ +\ (-1)^{\bars{a}} \rP_i[a] \, \beta\rP_{j+1}[b].
\end{align}

\sssec

A straightforward computation shows that the Cartan relation is equivalent to the following identities holding for any integer $i$:
\begin{align}
	\rD_{2i}\big([a][b]\big) =\ &
	(-1)^{\floor{p/2}\bars{a}\bars{b}} \sum_{i=j+k} \rD_{2j}[a] \, \rD_{2k}[b], \\
	\rD_{2i+1}\big([a][b]\big) =\ &
	(-1)^{\floor{p/2}\bars{a}\bars{b}} \sum_{i=j+k} \rD_{2j+1}[a] \, \rD_{2k}[b] \ +\ (-1)^{\bars{a}}\rD_{2j}[a] \, \rD_{2k+1}[b].
\end{align}

\anibal{Maybe write a proof?}

\subsection{Cartan relation at the chain level}

We can lift these relations to the chain level.
...
\begin{align*}
	\bd\zeta_{2i}(a,b) = \psi_{2i}\big((a \smallsmile b)^{\ot p}\big) +
	(-1)^{\floor{p/2}\bars{a}\bars{b}} \sum_{i=j+k}\big(\psi_{2j}(a^{\ot p})\big) \smallsmile \big(\psi_{2k}(b^{\ot p})\big),
\end{align*}
\begin{multline*}
	\bd\zeta_{2i+1}(a,b) =
	\psi_{2i+1}\big((a \smallsmile b)^{\ot p}\big) + (-1)^{\floor{p/2}\bars{a}\bars{b}} \\
	\cdot \sum_{i=j+k} \big(\psi_{2j+1}(a^{\ot p})\big) \smallsmile \big(\psi_{2k}(b^{\ot p})\big) \ +\
	(-1)^{\bars{a}}\big(\psi_{2j}(a^{\ot p})\big) \smallsmile \big(\psi_{2k+1}(b^{\ot p})\big).
\end{multline*}

\subsection{The basic reordering}

The element $\tau \in \sym_{2n}$ is the shuffle permutation mapping the first and second ``decks'' to odd and even integers respectively.
Explicitly, it is defined for $\ell \in \{1,\dots,2p\}$ by
\begin{equation*}
	\tau(\ell) =
	\begin{cases}
		2\ell-1 & \ell \leq p, \\
		2(\ell-p) & \ell > p.
	\end{cases}
\end{equation*}
Notice that for two elements $a$ and $b$ in a chain complex we have
\[
\tau(a^{\ot p} \ot b^{\ot p}) = (-1)^{\frac{(p-1)p}{2}\bars{a}\bars{b}} (a \ot b)^{\ot p}.
\]

\subsection{The main maps}

\sssec

Let $F$ be the composition
\[
\cW(p) \xra{\psi} \cE(p) \xra{\id\, \ot e_0 \ot e_0} \cE(p) \ot \cE(2) \ot \cE(2) \xra{\bcirc_{\cE}} \cE(2p).
\]
For any element $a$ and $b$ in an $\cE$-algebra we have
\[
\tau F(e_s)\big(a^{\ot p} \ot b^{\ot p}\big) =
(-1)^{\frac{(p-1)p}{2}\bars{a}\bars{b}} \, \psi_{s}\big((a \smallsmile b)^{\ot p}\big).
\]

\sssec

Let $G$ be the composition
\[
\cW(p) \xra{\Delta}
\cW(p) \ot \cW(p) \xra{\psi \ot \psi}
\cE(p) \ot \cE(p) \xra{e_0 \ot\, \id}
\cE(2) \ot \cE(p) \ot \cE(p) \xra{\bcirc_{\cE}}
\cE(2p).
\]
For elements $a$ and $b$ in an $\cE$-algebra and an integer $i$ we have that
\begin{equation}
	G(e_{2i})\big(a^{\ot p} \ot b^{\ot p}\big) =
	\sum_{i=j+k}\big(\psi_{2j}(a^{\ot p})\big) \smallsmile \big(\psi_{2k}(b^{\ot p})\big),
\end{equation}
\begin{equation}
	\begin{split}
		&G(e_{2i})\big(a^{\ot p} \ot b^{\ot p}\big) = \\
		&\sum_{i=j+k} \big(\psi_{2j+1}(a^{\ot p})\big) \smallsmile \big(\psi_{2k}(b^{\ot p})\big) \ +\
		(-1)^{\bars{a}}\big(\psi_{2j}(a^{\ot p})\big) \smallsmile \big(\psi_{2k+1}(b^{\ot p})\big)
	\end{split}
\end{equation}

RECALL
\begin{align*}
	\Delta(e_{2i}) &=
	\sum_{i=j+k} e_{2j} \ot e_{2k} \ + \sum_{i-1=j+k} \ \sum_{0 \leq r < s \leq p} \rho^r e_{2j+1} \ot \rho^s e_{2k+1}, \\
	\Delta(e_{2i+1}) &=
	\sum_{i=j+k} e_{2j} \ot e_{2k+1} \ +\ e_{2j+1} \ot \rho e_{2k}
\end{align*}
