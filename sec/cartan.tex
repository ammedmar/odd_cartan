% !TEX root = ../odd_cartan.tex

\section{The Cartan formula}\label{s:cartan}

\noindent Throughout this section $(A,\psi)$ denotes a May--Steenrod complex.

\subsection{The $(-)^\ev$ construction}

Recall from \cref{ss:f and g} the group homomorphism $g \colon \cyc_r \to \sym_{2r}$ defined for every $r > 0$.
Given an $\sym$-module $\cS$ we denote by $\cS^\ev$ the $\cyc$-module $\set{\cS(2r)}_{r>0}$ with the $\cyc_r$-action on $\cS(2r)$ defined by $g$.
In particular, for any $\cyc$-module $\cC$, a map $\cC \to \cS^\ev$ is $\cyc$-equivariant if and only if $\cC(r) \to \cS(2r)$ is $g$-equivariant for each $r > 0$.

\subsection{The maps $F$ and $G$}

Let us denote the operad $\End(A)$ by $\cO$.
Then, for $r > 0$, define, omitting the superscripts of $\psi^r$ and $\psi_0^2$,
\begin{align*}
	\tau \comp F_\psi \colon& \cW(r) \xra{\psi} \cO(r) \xra{\id\, \ot \psi_0^{\ot r}}
	\cO(r) \ot \cO(2)^{\ot r} \xra{\bcomp_{\cO}}
	\cO(2r) \xra{\tau} \cO(2r), \\
	G_\psi \colon& \cW(r) \xra{\Delta}
	\cW(r)^{\ot 2} \xra{\psi^{\ot 2}}
	\cO(r)^{\ot 2} \xra{\psi_0 \ot\, \id}
	\cO(2) \ot \cO(r)^{\ot 2} \xra{\bcomp_{\cO}}
	\cO(2r)
\end{align*}
where $\cO(2r) \xra{\tau} \cO(2r)$ stands for the right action of $\tau$.\footnote{\label{fn:symmetric_action_on_hom}
We remind the reader that the right action of $\sigma \in \sym_r$ on $u \in \cO(r) = \End(A)(r) = \Hom(A^{\ot r}, A)$ produces the composition $A^{\ot r} \xra{\sigma} \ot A^{\ot r} \xra{u} A$.}
We will write $\tau \comp F_\psi$ for $\set{\tau \comp F_\psi^r}_{r>0}$ and $G$ for $\set{G_\psi^r}_{r>0}$.

\begin{lemma*}
	The sets $\tau\, \comp F$ and $G$ define $\cyc$-equivariant chain maps from $\cW$ to $\End(A)^\ev$.
\end{lemma*}

\begin{proof}
	Let us fix $r>0$.
	Since the maps $\tau \comp F^r_\psi$ and $G^r_\psi$ are defined as compositions of chains maps, it suffices to verify that they are $g$-equivariant.
	We will omit the superscripts of $\psi^r$ and $\psi_0^2$.
	For $\tau \comp F^r_\psi$ we have:
	\begin{align*}
		(\tau \comp F^r_\psi)(e_i \cdot \rho) &=
		(\bcomp_{\End(A)}(\psi(e_i \cdot \rho) \ot \psi_0^{\ot r})) \cdot \tau \\ &=
		(\bcomp_{\End(A)}(\psi(e_i) \cdot \rho \ot \psi_0^{\ot r})) \cdot \tau \\ &=
		(\bcomp_{\End(A)}(\psi(e_i) \ot \psi_0^{\ot r})) \cdot (\bcomp_{\sym} (\rho \times e^{\times r}))\tau \\ &=
		(\bcomp_{\End(A)}(\psi(e_i) \ot \psi_0^{\ot r})) \cdot f(\rho)\tau \\ &=
		(\bcomp_{\End(A)}(\psi(e_i) \ot \psi_0^{\ot r})) \cdot \tau g(\rho) \\ &=
		(\tau \comp F^r_\psi)(e_i) \cdot g(\rho)
	\end{align*}
	where first and last equalities hold by definition, the second one by the $\cyc_r$-equivariance of $\psi^r$, the third one by the equivariance axiom of operadic composition, the fourth one by the definition of $f$ as given in \cref{ss:f and g}, and the fifth one by \cref{l:conjugated}.
	For $G^r_\psi$ we have:
	\begin{align*}
		G^r_\psi(e_i \cdot \rho) &=
		\big(\bcomp_{\End(A)} \comp (\psi_0 \ot \psi \ot \psi) \comp \Delta)(e_i \cdot \rho \big) \\ &=
		\bcomp_{\End(A)} \big(\psi_0 \ot \psi(\Delta^{(1)}(e_i \cdot \rho)) \ot \psi(\Delta^{(2)}(e_i \cdot \rho)) \big) \\ &=
		\bcomp_{\End(A)} \big(\psi_0 \ot \psi(\Delta^{(1)}(e_i)) \cdot \rho \ot \psi(\Delta^{(2)}(e_i)) \cdot \rho \big) \\ &=
		\bcomp_{\End(A)} \big(\psi_0 \ot \psi(\Delta^{(1)}(e_i)) \ot \psi(\Delta^{(2)}(e_i)) \big) \cdot \bcomp_{\sym}(e \times \rho \times \rho) \\ &=
		G^r_\psi(e_i) \cdot g(\rho)
	\end{align*}
	where the first equality holds by definition, the second one simply introduces Sweedler's notation, the third equality holds by the $\cyc_r$-equivariance of $\psi$ and $\Delta$, the fourth one by the equivariance axiom of operadic composition, and the last one by the definition of $g$ as given in \cref{ss:f and g}.
\end{proof}

\subsection{Cartan relators}\label{ss:cartan_relators}

A \textit{Cartan relator} for $(A,\psi)$ is a $\cyc$-equivariant homotopy from $G_\psi$ to $\tau \comp F_\psi$.
Explicitly, a Cartan relator for $(A,\psi)$ is a collection of natural degree one linear maps
\[
\set{H^r_\psi \colon \cW(r) \to \Hom(A^{\ot 2r}, A)}_{r>0}
\]
satisfying, for every $r > 0$,
\[
\bd_{\Hom(A^{\ot 2r},\, A)} \comp \,H^r_\psi + H^r_\psi \comp \bd_{\cW(r)} = \tau \comp F^r_\psi - G^r_\psi,
\]
and
\[
H^r_\psi(e_i \cdot \rho) = H^r_\psi(e_i) \cdot g(\rho)
\]
for every $i \in \N$.

We will omit $\psi$ from the notation when clear from the context.

\subsection{Cartan formulas}

As we will see, the existence of a Cartan relator implies the Cartan formulas:
\begin{align*}
	\rP_{s}\big([a][b]\big) =&
	\sum_{j+k=s} \rP_j[a] \, \rP_k[b], \\
	\beta \rP_{s+1}\big([a][b]\big) =&
	\sum_{j+k=s} \beta \rP_{j+1}[a] \, \rP_k[b] \ +\ (-1)^{\bars{a}} \rP_j[a] \, \beta \rP_{k+1}[b],
\end{align*}
where $s \in \Z$ and $a$ and $b$ are any two mod $p$ cycles in $A$.

By Thom's \cref{t:thom}, the Cartan formulas are equivalent to the following equations:
\begin{equation}\label{eq:cartan_lift}
	\begin{split}
		0 &= (-1)^{\frac{(p-1)}{2}\bars{a}\bars{b}} \, \rD_{2i}\big([a][b]\big) \,-\!
		\sum_{i=j+k} \rD_{2j}[a] \, \rD_{2k}[b], \\
		0 &= (-1)^{\frac{p-1}{2}\bars{a}\bars{b}} \, \rD_{2i+1}\big([a][b]\big) \\ & \quad
		-\sum_{\mathclap{i=j+k}} \big(\rD_{2j+1}[a] \, \rD_{2k}[b] \, +\, (-1)^{\bars{a}}\rD_{2j}[a] \, \rD_{2k+1}[b]\big),
	\end{split}
\end{equation}
ranging over $i \in \N$.
Explicitly, if one takes $2i = (2s-(|a|+|b|))(p-1)$ and multiplies the first identity above by $(-1)^s \nu(|a|)\nu(|b|)$ then one has an equation of the form
\[
0 = \rP_s([a][b]) - \sum_{\mathclap{s=j+k}} \rP_j[a] \rP_k[b] - \text{(other terms)}.
\]
Here, the other terms are constants times products $D_\ell[a] D_m[b]$, where the subscripts are such that at least one of the factors, maybe both, are coboundaries by Thom’s theorem.
\cref{s:thom} succeeds in writing the appropriate factor terms as coboundaries, and then the products are coboundaries in explicit ways.
A similar discussion applies to the formulas involving $\beta \rP_s([a][b])$.\footnote{\label{fn:mu}
The definition of the constants $\nu(n)$ is motivated by two facts.
First, it can be easily shown that $\rP_s([a]) = [a]^p$ for a cohomology class $[a]$ of degree $2s$.
More subtly, it can be shown that $\rP_0([a]) = [a]$ in all degrees for the specific $E_\infty$-algebras $A = \cochains(X)$, the normalized cochains of simplicial sets.
This result follows from the rather difficult formula for positive $n$ and a simplicial set class $[a]$ of degree $-n$ that $\rD_n(p-1)([a]) = (-1)^{mn} \nu(n)[a]$.
}

\subsection{Chain level lift}

A lift to the chain level of the right hand side of \cref{eq:cartan_lift} is provided by the following expressions in $A$:

\begin{align*}
	&C_\psi^p(2i)(a,b) \defeq (-1)^{\frac{p-1}{2}\bars{a}\bars{b}} \psi_{2i}\big((a \cp b)^{\ot p}\big) \ -
	\sum_{i=j+k}\big(\psi_{2j}(a^{\ot p})\big) \cp \big(\psi_{2k}(b^{\ot p})\big), \\
	&C_\psi^p(2i+1)(a,b) \defeq (-1)^{\frac{p-1}{2}\bars{a}\bars{b}} \psi_{2i+1}\big((a \cp b)^{\ot p}\big) \\
	&\quad -\sum_{i=j+k} \Big(
	\big(\psi_{2j+1}(a^{\ot p})\big) \cp \big(\psi_{2k}(b^{\ot p})\big)\ +\
	(-1)^{\bars{a}}\big(\psi_{2j}(a^{\ot p})\big) \cp \big(\psi_{2k+1}(b^{\ot p})\big)
	\Big).
\end{align*}

\begin{lemma}\label{l:F_and_G_give_C}
	For any $a,b \in A$ and $i \in \N$ the following identity holds in $A \ot \Fp$
	\[
	C^p_\psi(i)(a,b) =
	\big(\tau \comp F^p_\psi - G^p_\psi\big)(e_i)(a^{\ot p} \ot b^{\ot p}).
	\]
\end{lemma}

\begin{proof}
	Recall that $\rho$ is an even permutation and that $\psi \colon \cW(p) \to \End(A)(p)$ is $\cyc_p$-equivariant.
	We will use the notation $\psi^p(e_i) = \psi_i$ and $\psi^2_0(a \ot b) = a \smallsmile b$ freely.
	Let us first study
	\[
	G_\psi^p \colon \cW(p) \xra{\Delta}
	\cW(p)^{\ot 2} \xra{\psi_0 \ot \psi^{\ot 2}}
	\End(A)(2) \ot \End(A)(p)^{\ot 2} \xra{\bcomp_{\End(A)}}
	\End(A)(2p).
	\]
	Notice that for any $j,k \in \N$ and $a,b \in A$ we have
	\[
	(\psi_j \smallsmile \psi_k)(a^{\ot p} \ot b^{\ot p}) = (-1)^{k\bars{a}}(\psi_j(a^{\ot p}) \smallsmile \psi_k(b^{\ot p}))
	\]
	by the Koszul sign convention.
	Recall that
	\[
	\Delta(e_{2i}) =
	\sum_{\mathclap{\ i=j+k}} \, e_{2j} \ot e_{2k} \ + \!\!
	\sum_{i-1=j+k} \ \sum_{0 \leq s<t<r} \rho^s e_{2j+1} \ot \rho^t e_{2k+1},
	\]
	and notice that the number of integers $s$ and $t$ satisfying $0 \leq s < t < p$ is divisible by $p$, being in fact equal to $\frac{p(p-1)}{2}$.
	Since $a^{\ot p}$ and $b^{\ot p}$ are fixed by $\rho$,
	\[
	\sum_{i-1=j+k} \ \sum_{0 \leq s < t < r} (-1)^{\bars{a}} \, \psi_{2j+1}(a^{\ot p} \cdot \rho^s) \smallsmile \psi_{2k+1}(b^{\ot p} \cdot \rho^t)
	= 0
	\]
	in $A \ot \Fp$.
	Therefore,
	\[
	G_\psi^p(e_{2i})(a^{\ot p} \ot b^{\ot p}) =
	\sum_{\mathclap{\ i=j+k}} \ \psi_{2j}(a^{\ot p}) \smallsmile \psi_{2k}(b^{\ot p}).
	\]
	Additionally, since,
	\[
	\Delta(e_{2i+1}) =
	\sum_{\mathclap{\ i=j+k}} \ e_{2j+1} \ot \rho e_{2k} + e_{2j} \ot e_{2k+1},
	\]
	we have
	\begin{multline*}
		G_\psi^p(e_{2i+1})(a^{\ot p} \ot b^{\ot p}) = \\
		\sum_{i=j+k} \Big(
		\big(\psi_{2j+1}(a^{\ot p})\big) \cp \big(\psi_{2k}(b^{\ot p})\big)\ +\
		(-1)^{\bars{a}}\big(\psi_{2j}(a^{\ot p})\big) \cp \big(\psi_{2k+1}(b^{\ot p})\big)
		\Big).
	\end{multline*}
	Let us now consider
	\[
	\tau \comp F_\psi^p \colon \cW(p) \xra{\!\psi \ot \psi_0^{\ot p}\hspace*{-5pt}}
	\End(A)(p) \ot \End(A)(2)^{\ot r} \xra{\!\bcomp_{\End(A)}}
	\End(A)(2p) \xra{\tau} \End(A)(2p).
	\]
	Recall from \cref{ss:shuffle} that for any $a,b \in A$,
	\begin{align*}
		\tau(a^{\ot p} \ot b^{\ot p}) &=
		(-1)^{\frac{(p-1)p}{2}\bars{a}\bars{b}} \, (a \ot b)^{\ot p} \\ &=
		(-1)^{\frac{p-1}{2}\bars{a}\bars{b}} \, (a \ot b)^{\ot p}.
	\end{align*}
	Therefore,
	\begin{align*}
		(\tau \comp F_\psi^p)(e_i)(a^{\ot p} \ot b^{\ot p}) &=
		(-1)^{\frac{p-1}{2}\bars{a}\bars{b}} F_\psi^p(e_i)\big((a \ot b)^{\ot p}\big) \\ &=
		(-1)^{\frac{p-1}{2}\bars{a}\bars{b}} \psi_{i}\big((a \cp b)^{\ot p}\big),
	\end{align*}
	from which the lemma follows.
\end{proof}

\subsection{Cartan boundary constructions}\label{ss:cartan_coboundary}

A \textit{Cartan $i$-boundary construction for $(A,\psi)$ at $p$} is an assignment of an element $\zeta_i^p(a,b)$ in $A$ to any pair $(a,b)$ of mod~$p$ cycles in $A$ such that, after extending scalars to $\Fp$, we have
\[
\bd_A \zeta_i^p(a,b) = C_\psi^p(i)(a, b).
\]

\begin{theorem}
	If $H_\psi$ is a Cartan relator for $(A, \psi)$ then
	\[
	(a,b) \mapsto H_\psi^p(e_i)(a^{\ot p} \ot b^{\ot p})
	\]
	is a Cartan $i$-boundary construction for $(A, \psi)$ at $p$.
\end{theorem}

\begin{proof}
	Let us simplify our notation by removing the superscripts $p$ and subscripts $\psi$, and writing $\bd_{\End}$ and $\bd_\cW$ instead of $\bd_{\End(A)(p)}$, and $\bd_{\cW(p)}$.
	Recall that
	\[
	\bd_{\End} \comp\, H - H \comp \bd_{\cW} = \tau \comp F - G.
	\]
	Since $\rho$ is an even permutation,
	\[
	(a^{\ot p} \ot b^{\ot p}) \cdot g(\rho) = (a^{\ot p} \cdot \rho \ot b^{\ot p} \cdot \rho ) = (a^{\ot p} \ot b^{\ot p}),
	\]
	and
	\begin{align*}
		&H(\bd_{\cW} e_{2i})(a^{\ot p} \ot b^{\ot p}) =\,
		(1+g(\rho)+\dots+g(\rho^{p-1})) \comp H(e_{2i-1})(a^{\ot p} \ot b^{\ot p}) = 0, \\
		&H(\bd_{\cW} e_{2i+1})(a^{\ot p} \ot b^{\ot p}) =\,
		(g(\rho) - 1) \comp H(e_{2i})(a^{\ot p} \ot b^{\ot p}) = 0.
	\end{align*}
	Therefore, since $a$ and $b$ are mod $p$ cycles,
	\begin{align*}
		(\bd_A \comp H(e_i))(a^{\ot p} \ot b^{\ot p}) =&
		\bd_{\End} H(e_i)(a^{\ot p} \ot b^{\ot p}) - (H(e_i) \comp \bd_{A^{\ot 2p}})(a^{\ot p} \ot b^{\ot p}) \\ =& \,
		\big(\tau F - G\big)(e_i)(a^{\ot p} \ot b^{\ot p}) -
		H(\bd_\cW e_i)(a^{\ot p} \ot b^{\ot p}) \\ =&\,
		C(i)(a,b),
	\end{align*}
	where the last equality is the content of \cref{l:F_and_G_give_C}.
\end{proof}