% !TEX root = ../odd_cartan.tex

\section{The Cartan formula}\label{s:cartan}

\subsection{Cartan relators}

Given an $\sym$-module $\cS$ we denote by $\cS^\wedge$ the $\cyc$-module $\cS^\wedge(r) = \cS(2r)$ with
action defined by
\begin{equation}\label{eq:cyclic to 2-symetric}
	\cyc_r \xra{\rD}
	\cyc_r \times \cyc_r \hookrightarrow
	\sym_r \times \sym_r \xra{\mathrm{split}}
	\sym_{2r}.
\end{equation}
For a May--Steenrod complex $(A,\psi)$ we define two $\cyc$-equivariant chain maps
\[
F_\psi,\, G_\psi \colon \cW \to \End(A)^\wedge
\]
as follows.
Let us denote the operad $\End(A)$ by $\cO$.
Then,
\begin{align*}
	F_\psi \colon& \cW(r) \xra{\psi(r)} \cO(r) \xra{\id\, \ot \psi_0^{\ot r}}
	\cO(r) \ot \cO(2)^{\ot r} \xra{\bcirc_{\cO}}
	\cO(2r), \\
	G_\psi \colon& \cW(r) \xra{\Delta}
	\cW(r)^{\ot 2} \xra{\psi^{\ot 2}}
	\cO(r)^{\ot 2} \xra{\psi_0 \ot\, \id}
	\cO(2) \ot \cO(r)^{\ot 2} \xra{\bcirc_{\cO}}
	\cO(2r).
\end{align*}

\begin{definition*}
	A \textit{Cartan relator} for a May--Steenrod complex $(A, \psi)$ is a $\cyc$-equivariant chain homotopy between $\tau F_\psi$ and $G_\psi$ where $\tau$ is the canonical reordering of \cref{ss:reordering}.
\end{definition*}

\subsection{Cartan boundary constructions}\label{ss:cartan_coboundary}

As we will see, the existence of a Cartan relator implies the Cartan formulas:
\begin{align*}
	\rP_s\big([a][b]\big) =&
	\sum_{i+j=s} \rP_i[a] \, \rP_j[b], \\
	\beta\rP_s\big([a][b]\big) =&
	\sum_{i+j=s} \beta\rP_{i+1}[a] \, \rP_j[b] \ +\ (-1)^{\bars{a}} \rP_i[a] \, \beta\rP_{j+1}[b],
\end{align*}
holding for each integer $s$ and mod $p$ cycles $a$ and $b$.
Please observe that the Cartan formulas are equivalent to the following equations:
\begin{align*}
	0 &= (-1)^{\floor{p/2}\bars{a}\bars{b}} \, \rD_{2i}\big([a][b]\big) \,-\!
	\sum_{i=j+k} \rD_{2j}[a] \, \rD_{2k}[b], \\
	0 &= (-1)^{\floor{p/2}\bars{a}\bars{b}} \, \rD_{2i+1}\big([a][b]\big) \,-\!
	\sum_{i=j+k} \rD_{2j+1}[a] \, \rD_{2k}[b] \ +\ (-1)^{\bars{a}}\rD_{2j}[a] \, \rD_{2k+1}[b],
\end{align*}
ranging over $i \in \N$.
A lift of the right hand side of these to the chain level is given by the following formulas:
\begin{align*}
	&C_\psi^p(2i)(a,b) \defeq (-1)^{\floor{p/2}\bars{a}\bars{b}} \psi_{2i}\big((a \cp b)^{\ot p}\big) \ -
	\sum_{i=j+k}\big(\psi_{2j}(a^{\ot p})\big) \cp \big(\psi_{2k}(b^{\ot p})\big), \\
	&C_\psi^p(2i+1)(a,b) \defeq (-1)^{\floor{p/2}\bars{a}\bars{b}} \psi_{2i+1}\big((a \cp b)^{\ot p}\big) \\
	&\quad -\sum_{i=j+k} \big(\psi_{2j+1}(a^{\ot p})\big) \cp \big(\psi_{2k}(b^{\ot p})\big)\ +\
	(-1)^{\bars{a}}\big(\psi_{2j}(a^{\ot p})\big) \cp \big(\psi_{2k+1}(b^{\ot p})\big).
\end{align*}

\begin{definition*}
	Let $(A,\psi)$ be a May--Steenrod complex.
	A \textit{Cartan $i$-boundary construction} is a map $\zeta_i$ from pairs of mod~$p$ cycles to $A$, natural with respect to $E_\infty$-algebra chain maps, such that
	\[
	\bd \zeta_i(a,b) = C_\psi^p(i)(a, b)
	\]
	in $A \ot \Fp$.
\end{definition*}

\begin{theorem*}
	If $H_\psi$ is a Cartan relator for $(A, \psi)$ then
	\[
	\zeta_i(a,b) \defeq H_\psi^p(e_i)(a^{\ot p} \ot b^{\ot p})
	\]
	defines a Cartan $i$-boundary construction.
\end{theorem*}

\begin{proof}
	First, please observe that
	\[
	C^p_\psi(i)(a,b) =
	\big(\tau F_\psi^p - G_\psi^p\big)(e_i)(a^{\ot p} \ot b^{\ot p}).
	\]
	Second, using that in $A \ot \Fp$
	\begin{align*}
		H_\psi^p(\bd e_{2i})(a^{\ot p} \ot b^{\ot p}) =&\,
		(\rho - 1) H_\psi^p(e_{2i})(a^{\ot p} \ot b^{\ot p}) = 0, \\
		H_\psi^p(\bd e_{2i+1})(a^{\ot p} \ot b^{\ot p}) =&\,
		(1+\rho+\dots+\rho^{p-1}) H_\psi^p(e_{2i+1})(a^{\ot p} \ot b^{\ot p}) = 0,
	\end{align*}
	we have
	\begin{align*}
		\bd_A \zeta_i(a,b) =&
		\bd_A H_\psi^p(e_i)(a^{\ot p} \ot b^{\ot p}) \\ =&
		\bd_{\End(A)} H_\psi^p(e_i)(a^{\ot p} \ot b^{\ot p}) \\ =& \,
		\big(\tau F(e_i) - G(e_i)\big)(a^{\ot p} \ot b^{\ot p}) -
		H_\psi^p(\bd e_i)(a^{\ot p} \ot b^{\ot p}) \\ =&\,
		C^p_\psi(i)(a,b).\qedhere
	\end{align*}
\end{proof}

We now turn to the question of effectively constructing a natural one for the cochains of spaces.