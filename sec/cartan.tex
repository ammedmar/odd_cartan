% !TEX root = ../odd_cartan.tex

\section{The Cartan formula}\label{s:cartan}

Throughout this section $(A,\psi)$ denotes a May--Steenrod complex.

\subsection{The $(-)^\vee$ construction}

%Consider the group homomorphism $g \colon \cyc_r \to \sym_{2r}$ defined by
%\[
%\rho \mapsto (1,2,\dots,r)(r+1,r+2,\dots,2r).
%\]
Given an $\sym$-module $\cS$ we denote by $\cS^\vee$ the $\cyc$-module $\set{\cS(2r)}_{r>0}$ with $\cyc_r$-action on $\cS(2r)$ given by the action of $g(\rho)$, as defined in \cref{ss:f and g}.

\subsection{The maps $F$ and $G$}

We define two $\cyc$-equivariant chain maps
\[
\tau \comp F_\psi,\, G_\psi \colon \cW \to \End(A)^\vee
\]
as follows.
Let us denote the operad $\End(A)$ by $\cO$.
Then, for $r \in \N$,
\begin{align*}
	\tau \comp F_\psi^r \colon& \cW(r) \xra{\psi} \cO(r) \xra{\id\, \ot \psi_0^{\ot r}}
	\cO(r) \ot \cO(2)^{\ot r} \xra{\bcirc_{\cO}}
	\cO(2r) \xra{\tau} \cO(2r), \\
	G_\psi^r \colon& \cW(r) \xra{\Delta}
	\cW(r)^{\ot 2} \xra{\psi^{\ot 2}}
	\cO(r)^{\ot 2} \xra{\psi_0 \ot\, \id}
	\cO(2) \ot \cO(r)^{\ot 2} \xra{\bcirc_{\cO}}
	\cO(2r).
\end{align*}
We will remove $\psi$ from the notation when it is clear from context.

\subsection{Cartan relators}\label{ss:cartan_relators}

A \textit{Cartan relator} for $(A,\psi)$ is a $\cyc$-equivariant homotopy from $G_\psi$ to $\tau \comp F_\psi$.
Explicitly, a Cartan relator for $(A,\psi)$ is a natural collection of linear maps
\[
\set{H^r \colon \cW(r) \to \Hom(A^{\ot 2r}, A)}_{r>0}
\]
satisfying, for every $r > 0$,
\[
\bd \circ H^r + H^r \comp \bd = \tau \comp F^r - G^r,
\]
and
\[
H^r \circ \rho = g(\rho) \circ H^r.
\]

\subsection{Cartan formulas}

As we will see, the existence of a Cartan relator implies the Cartan formulas:
\begin{align*}
	\rP_s\big([a][b]\big) =&
	\sum_{i+j=s} \rP_i[a] \, \rP_j[b], \\
	\beta\rP_s\big([a][b]\big) =&
	\sum_{i+j=s} \beta\rP_{i+1}[a] \, \rP_j[b] \ +\ (-1)^{\bars{a}} \rP_i[a] \, \beta\rP_{j+1}[b],
\end{align*}
holding for each integer $s$ and mod $p$ cycles $a$ and $b$ in $A$.
Please observe that by Thom's \cref{t:thom}, the Cartan formulas are equivalent to the following equations:
\begin{equation}\label{eq:cartan_lift}
	\begin{split}
		0 &= (-1)^{\floor{p/2}\bars{a}\bars{b}} \, \rD_{2i}\big([a][b]\big) \,-\!
		\sum_{i=j+k} \rD_{2j}[a] \, \rD_{2k}[b], \\
		0 &= (-1)^{\floor{p/2}\bars{a}\bars{b}} \, \rD_{2i+1}\big([a][b]\big)  \\ & \quad
		\,-\!\sum_{i=j+k} \rD_{2j+1}[a] \, \rD_{2k}[b] \, +\, (-1)^{\bars{a}}\rD_{2j}[a] \, \rD_{2k+1}[b],
	\end{split}
\end{equation}
ranging over $i \in \N$.

\subsection{Lift of the Cartan formulas}

A lift to the chain level of the right hand side of \cref{eq:cartan_lift} is provided in the following expression:
\begin{align*}
	&C_\psi^p(2i)(a,b) \defeq (-1)^{\floor{p/2}\bars{a}\bars{b}} \psi_{2i}\big((a \cp b)^{\ot p}\big) \ -
	\sum_{i=j+k}\big(\psi_{2j}(a^{\ot p})\big) \cp \big(\psi_{2k}(b^{\ot p})\big), \\
	&C_\psi^p(2i+1)(a,b) \defeq (-1)^{\floor{p/2}\bars{a}\bars{b}} \psi_{2i+1}\big((a \cp b)^{\ot p}\big) \\
	&\quad -\sum_{i=j+k} \big(\psi_{2j+1}(a^{\ot p})\big) \cp \big(\psi_{2k}(b^{\ot p})\big)\ +\
	(-1)^{\bars{a}}\big(\psi_{2j}(a^{\ot p})\big) \cp \big(\psi_{2k+1}(b^{\ot p})\big).
\end{align*}
%
%\begin{definition}
%	For any $a,b \in A$ and $i \in \N$ let
%
%\end{definition}

\begin{lemma}
	For any $a,b \in A$ and $i \in \N$ we have
	\[
	C^p_\psi(i)(a,b) =
	\big(\tau \comp F - G\big)(e_i)(a^{\ot p} \ot b^{\ot p}).
	\]
\end{lemma}

\begin{proof}
	This follows from direct inspection.
\end{proof}

\subsection{Cartan boundary constructions}\label{ss:cartan_coboundary}

A \textit{Cartan $i$-boundary construction for $(A,\psi)$ at $p$} is an assignment of an element $\zeta_i^p(a,b)$ in $A$ to any pair $(a,b)$ of mod~$p$ cycles in $A$ such that, after extending scalars to $\Fp$, we have
\[
\bd \zeta_i^p(a,b) = C_\psi^p(i)(a, b).
\]

\begin{theorem}
	If $H_\psi$ is a Cartan relator for $(A, \psi)$ then
	\[
	(a,b) \mapsto H_\psi^p(e_i)(a^{\ot p} \ot b^{\ot p})
	\]
	is a Cartan $i$-boundary construction for $(A, \psi)$ at $p$.
\end{theorem}

\begin{proof}
	Let us simplify our notation from $H_\psi^p, F_\psi^p, G_\psi^p$ to $H, F, G$.
	Second, since $\rho$ is an even permutation,
	\[
	(a^{\ot p} \ot b^{\ot p}) g(\rho) = (a^{\ot p} \rho \ot b^{\ot p} \rho ) = (a^{\ot p} \ot b^{\ot p}),
	\]
	consequently, over $\Fp$\,,
	\begin{align*}
		H(\bd e_{2i})(a^{\ot p} \ot b^{\ot p}) =&\,
		(g(\rho) - 1) \comp H(e_{2i})(a^{\ot p} \ot b^{\ot p}) = 0, \\
		H(\bd e_{2i+1})(a^{\ot p} \ot b^{\ot p}) =&\,
		(1+g(\rho)+\dots+g(\rho^{p-1})) \comp H(e_{2i+1})(a^{\ot p} \ot b^{\ot p}) = 0.
	\end{align*}
	Therefore,
	\begin{align*}
		\bd_A H(e_i)(a^{\ot p} \ot b^{\ot p}) =&
		\bd_A H(e_i)(a^{\ot p} \ot b^{\ot p}) \\ =&
		\bd_{\End(A)} H(e_i)(a^{\ot p} \ot b^{\ot p}) \\ =& \,
		\big(\tau F - G\big)(e_i)(a^{\ot p} \ot b^{\ot p}) -
		H(\bd e_i)(a^{\ot p} \ot b^{\ot p}) \\ =&\,
		C^p_\psi(i)(a,b).\qedhere
	\end{align*}
\end{proof}