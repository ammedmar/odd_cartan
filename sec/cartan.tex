% !TEX root = ../odd_cartan.tex

\section{The Cartan relation}\label{s:cartan}

Throughout this section we assume that $A$ is equipped with a May--Steenrod structure~$\psi$.

\subsection{Homology level}

We recall two equivalent presentation of the Cartan relation in the mod $p$ homology of $A$.
The first states that
\begin{align*}
	\rP_s\big([a][b]\big) =&
	\sum_{i+j=s} \rP_i[a] \, \rP_j[b], \\
	\beta\rP_s\big([a][b]\big) =&
	\sum_{i+j=s} \beta\rP_{i+1}[a] \, \rP_j[b] \ +\ (-1)^{\bars{a}} \rP_i[a] \, \beta\rP_{j+1}[b],
\end{align*}
for any integer $s$ and classes $[a],[b] \in \rH(A;\Fp)$.
The second, that
%A straightforward computation shows that the Cartan relation is equivalent to the following identities
\begin{align*}
	%	\label{eq:even degree cartan}
	\rD_{2i}\big([a][b]\big) =\ &
	(-1)^{\floor{p/2}\bars{a}\bars{b}} \sum_{i=j+k} \rD_{2j}[a] \, \rD_{2k}[b], \\
	%	\label{eq:odd degree cartan}
	\rD_{2i+1}\big([a][b]\big) =\ &
	(-1)^{\floor{p/2}\bars{a}\bars{b}} \sum_{i=j+k} \rD_{2j+1}[a] \, \rD_{2k}[b] \ +\ (-1)^{\bars{a}}\rD_{2j}[a] \, \rD_{2k+1}[b].
\end{align*}
for any $i \geq 0$ and classes $[a],[b] \in \rH(A;\Fp)$.

For $a,b \in A \ot \Fp$ and $i \in \N$, let
\begin{multline*}
	C_{2i}^p(a \ot b) \defeq (-1)^{\floor{p/2}\bars{a}\bars{b}} \psi_{2i}\big((a \smallsmile b)^{\ot p}\big) \ -
	\sum_{i=j+k}\big(\psi_{2j}(a^{\ot p})\big) \smallsmile \big(\psi_{2k}(b^{\ot p})\big), \hfill \\
	C_{2i+1}^p(a \ot b) \defeq (-1)^{\floor{p/2}\bars{a}\bars{b}} \psi_{2i+1}\big((a \smallsmile b)^{\ot p}\big) \hfill \\ -
	\sum_{i=j+k} \big(\psi_{2j+1}(a^{\ot p})\big) \smallsmile \big(\psi_{2k}(b^{\ot p})\big)\ +\
	(-1)^{\bars{a}}\big(\psi_{2j}(a^{\ot p})\big) \smallsmile \big(\psi_{2k+1}(b^{\ot p})\big).
\end{multline*}

\subsection{Main theorem}\label{ss:main maps}

Let $A$ be equipped with a May--Steenrod structure $\psi$.
We will define for each prime $p$ two chain maps
\[
F_\psi^p,\, G_\psi^p \colon \cW(p) \to \Hom(A^{\ot 2p}, A).
\]
To simplify notation let us denote the operad $\End(A)$ by $\cO$.
These maps are then defined by the following compositions:
\begin{align*}
	F_\psi^p \colon& \cW(p) \xra{\psi(p)} \cO(p) \xra{\id\, \ot (\psi_0^2)^{\ot p}}
	\cO(p) \ot \cO(2)^{\ot p} \xra{\bcirc_{\cO}}
	\cO(2p), \\
	G_\psi^p \colon& \cW(p) \xra{\Delta}
	\cW(p)^{\ot 2} \xra{\psi(p)^{\ot 2}}
	\cO(p)^{\ot 2} \xra{\psi_0^2 \ot\, \id}
	\cO(2) \ot \cO(p)^{\ot 2} \xra{\bcirc_{\cO}}
	\cO(2p).
\end{align*}
Considering $\cO(2p) = \Hom(A^{\ot 2p}, A)$ endowed the action of $\cyc_p$ induced by the group homomorphism
\[
\cyc_p \xra{\mathrm{diag}} \cyc_p \times \cyc_p \xra{\mathrm{incl}} \sym_p \times \sym_p \xra{\mathrm{split}} \sym_{2p},
\]
it can be observed that both $F$ and $G$ are $\cyc_p$-equivariant.

A Cartan relator for $A$ at $p$ is a $\cyc_p$-equivariant homotopy $H_\psi^p \colon \cW(p) \to \Hom(A^{\ot 2p}, A)$ between $\tau F$ and $G$.

\begin{theorem*}
	If $H_\psi^p$ is Cartan relator for $A$ at $p$ then
	\[
	\bd \big(H_\psi^p(e_i)(a^{\ot p} \ot b^{\ot p})\big) = C_\psi^p(a \ot b)
	\]
	for any $i \in \N$ and cycles $a,b \in A \ot \Fp$
\end{theorem*}

%\begin{proof}
%	Let $a, b \in A \ot \Fp$ be cocycles.
%	By assumption
%	\[
%	\bd H(e_s) + H(\bd e_s) = (\tau F - G)(e_s).
%	\]
%	Since $\rho H(e_s)()$
%	If $s$ is even then $H(\bd e_s) = H(1+\rho+\dots+\rho^{p-1})e_{s-1} = (1+\rho+\dots+\rho^{p-1})H(e_{s-1})$.
%	Since
%	\anibal{finish this}
%
%	Notice that
%	\[
%	\tau F(e_s)\big(a^{\ot p} \ot b^{\ot p}\big) =
%	(-1)^{\frac{(p-1)p}{2}\bars{a}\bars{b}} \, \psi_{s}\big((a \smallsmile b)^{\ot p}\big)
%	\]
%	for any $s \in \N$ and $a,b \in A$.
%
%
%	Notice that if $\cO = \End(A)$ then
%	\begin{equation*}
%		\begin{split}
%			G(e_{2i})\big(a^{\ot p} \ot b^{\ot p}\big) =&
%			\sum_{i=j+k}\big(\psi_{2j}(a^{\ot p})\big) \smallsmile \big(\psi_{2k}(b^{\ot p})\big), \\
%			G(e_{2i+1})\big(a^{\ot p} \ot b^{\ot p}\big) =&
%			\sum_{i=j+k} \big(\psi_{2j+1}(a^{\ot p})\big) \smallsmile \big(\psi_{2k}(b^{\ot p})\big) \\ \ +&
%			\sum_{i=j+k} (-1)^{\bars{a}} \big(\psi_{2j}(a^{\ot p})\big) \smallsmile \big(\psi_{2k+1}(b^{\ot p})\big),
%		\end{split}
%	\end{equation*}
%	for any $i \in \N$ and $a,b \in A \ot \Fp$.
%	This follows directly from \cref{sss:minimal resolution} using that the number of pairs $r<s$ between $0$ and $p$ is divisible by $p$.
%\end{proof}

