% !TEX root = ../odd_cartan.tex

\section{Equivariant homotopies from contractions}\label{s:equivariant homotopy general}

In this section we give a general procedure to construct equivariant homotopies of maps from a free chain complex to a bounded-below chain complex equipped equipped with a contraction.

Recall that a \textit{contraction} of a chain complex $D$ is the data of two chain maps
\[
\begin{tikzcd}
	D \arrow[r,"\varepsilon", shift left=2pt] & \Z \arrow[l, "\iota", shift left=2pt]
\end{tikzcd}
\]
with $\varepsilon \circ \iota = \id_\Z$ and a homotopy $h \colon D \to D$ from $\iota \circ \varepsilon$ to $\id_D$.

\begin{lemma}
	Let $C$ a bounded-below chain complex of free $\Z[G]$-modules with a basis.
	Let $D$ be a chain complex of $\Z[G]$-modules together with a $(\Z$-linear$)$ contraction $(\varepsilon, \iota, h)$ satisfying $\iota \circ \varepsilon \circ h = 0$.
	Then, for any pair of $\Z[G]$-linear chain maps $\mu,\nu \colon C \to D$ with $\iota \circ \varepsilon \circ (\mu -\nu) = 0$, the $\Z[G]$-linear map $H \colon C \to D$ recursively defined on basis elements by
	\begin{equation}\label{eq:lemma def}
		H(b) = h \circ (\mu - \nu - H \circ \bd)(b)
	\end{equation}
	is a homotopy from $\mu$ to $\nu$.
\end{lemma}

\begin{proof}
	We will use an induction argument on the degree of $b$.
	Suppressing composition symbols we have
	\begin{align*}
		(\bd H + H \bd)(b)
		&= \bd h(\mu - \nu - H \bd)(b) + H \bd (b) \\
		&= (\id_D - \iota\varepsilon - h \bd)(\mu - \nu - H \bd)(b) + H \bd(b) \\
		&= (\mu-\nu)(b) - (\iota\varepsilon + h \bd)(\mu - \nu - H \bd)(b).
	\end{align*}
	Therefore, the claim follows from proving the identity
	\begin{equation}\label{eq:lemma want}
		(\iota\varepsilon + h \bd)(\mu - \nu - H \bd)(b) = 0.
	\end{equation}
	If $b$ is a cycle, for example if it is in lowest degree, \eqref{eq:lemma want} follows from $\iota\varepsilon(\mu-\nu) = 0$.
	Consequently, we have the base case of our induction argument.
	Using \eqref{eq:lemma def} we have
	\[
	H\bd (b) = H(\bd b) = h(\mu-\nu-H\bd)(\bd b) = h(\mu-\nu)(\bd b),
	\]
	and, since $\iota\varepsilon h = 0$, \eqref{eq:lemma want} is equivalent to the identity
	\[
	h \bd (\mu-\nu - H\bd)(b) = 0,
	\]
	which in turn is equivalent to
	\[
	h (\mu-\nu - \bd H)(\bd b) = 0.
	\]
	By the induction hypothesis, this is a direct consequence of $\bd^2 = 0$.
\end{proof}