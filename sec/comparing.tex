% !TEX root = ../odd_cartan.tex

\section{A proof of \cref{l:K3}}\label{s:postponed}

\noindent In this section we prove that $K_3 \colon \cW \to \cC \ot \cC$, as defined in \cref{ss:coproduct}, is a $\cyc$-equivariant homotopy from $\Delta_{\AW} \comp \iota$ to $(\iota \ot \iota) \comp \Delta$, where the $\cyc$-module structure on $\cC \ot \cC$ is defined by the diagonal action. To lighten the notation, we will write $K$ instead of $K_3$.

\subsection{The recursively defined homotopy $K$}

We can apply the recursive construction of the previous section to obtain a $\cyc$-equivariant homotopy $K$ making the following diagram commute, where $\cyc$ acts diagonally on the tensor products.:
\begin{equation*}
	\begin{tikzcd}[column sep=large]
		\cW \arrow[r,"\iota"] \arrow[d,"\Delta"'] \arrow[dr,phantom,"K"]&
		\cC \arrow[d,"\Delta_{\AW}"] \\
		\cW^{\ot 2} \arrow[r,"\iota^{\ot 2}"'] &
		\cC^{\ot 2}.
	\end{tikzcd}
\end{equation*}
To do so, we consider the endomorphisms $\xi$ and $\eta$ of $\cC(r) \ot \cC(r)$ given by
\[
\xi(s \ot t) =
\begin{cases}
	(0) \ot t & \text{if } \deg(s) = 0, \\
	\hfil0 & \text{otherwise}.
\end{cases}
\]
and
\[
\eta \big((s_0,\dots,s_j) \ot (t_0,\dots,t_k)\big) = (0,s_0,\dots,s_j) \ot (t_0,\dots,t_k)
\]
respectively, and denote by $\mu = (\iota \ot \iota) \comp \Delta$ and $\nu = \Delta_{\AW} \comp \iota$.

It can be easily verified that the pair $(\xi,\eta)$ is an idempotent homotopy identity with $\xi \comp (\mu - \nu) = 0$ and, since $\cW(r)$ has a preferred basis, \cref{t:recursive_homotopy} applies and produces the desired equivariant homotopy $K$.

\subsection{A closed formula for $K$}\label{ss:closed formula for K}

Using the bijection $\cyc_r \cong \set{0,1,\dots,r-1} \subset \Z$ the group $\cyc_r$ receives a total order.
It additionally has a cyclic order, which is the ternary relation defined on elements $a,b,c \in \cyc_r$ as follows:
$a \prec b \prec c$ if there are representatives $\bar{a},\bar{b},\bar{c}$ of the classes $[a],[b],[c]$ in $\Z/r\Z$ such that $\bar{a} < \bar{b} < \bar{c} < \bar{a}+r$.

{For $(s_1,\dots,s_j)$ in $\EE\cyc_r$ and $t\in \cyc_r$, we use the following notation:
\begin{align*}
    \theta_q(s_1,\ldots,s_j;t) &= \begin{cases}
        1 & \text{if $q>1$ and $s_{q-1}+1\prec s_q\prec t$}\\
        1 & \text{if $q=1$ and $0\prec s_q\prec t$} \\
        0 & \text{otherwise}
    \end{cases}
    % \\
    % \theta_q^{\mathrm{odd}}(s_1,\ldots,s_j;t) &= \begin{cases}
    %     1 & \text{if $q>1$ and $s_{q-1}\prec s_q\prec t$}\\
    %     1 & \text{if $q=1$ and $1\prec s_q\prec t$} \\
    %     0 & \text{otherwise}
    % \end{cases}
\end{align*}

% \color{blue}
% For $(s_1,\dots,s_j)$ in $\EE\cyc_r$ we use the following notation:
% \[
% s_q^- =
% \begin{cases}
% 	s_{q-1}+1 & \text{if } q>1, \\
% 	0 & \text{if } q=1,
% \end{cases}
% \]
%  and for each $1\leq q\leq j$, write $\theta_q^{\mathrm{even}}(s_1,\dots,s_j;t_1,\dots,t_k)$ for
% \[
% (0,s_1,s_1+1,\dots,s_j,s_j+1)\ot (t_1,t_1+1,\dots,t_k,t_k+1)
% \]
% if $s_q^- \prec s_q \prec t_1$ and $0$ otherwise, and $\theta_q^{\mathrm{odd}}(s_1,\dots,s_j;t_1,\dots,t_k)$ for
% \[
% (0,1,s_1,s_1+1,\dots,s_j,s_j+1)\ot (t_1,t_1+1,\dots,t_k,t_k+1)
% \]
% if $s_q^- \prec s_q \prec t_1$ and $0$ otherwise. 
% }

Lemma \ref{l:K3} is a consequence of the following two lemmas.

\begin{lemma}
	The map $K$ is defined by the following expressions
	\begin{align} \label{eq:homotopyK'1}
		K(e_{2i}) &= \sum_{i+1 = j+k} \sum_{q=1}^j \
		\theta_q(s_1,\dots,s_j;t_1)\cdot \varphi^{\mathrm{even}}(s_1,\dots,s_j;t_1,\dots,t_k), \\ \label{eq:homotopyK'2}
		K(e_{2i+1}) &= -\sum_{i+1 = j+k} \sum_{q=1}^j \
		\theta_q(s_1-1,\dots,s_j-1;t_1-1)\cdot \varphi^{\mathrm{odd}}(s_1,\dots,s_j;t_1,\dots,t_k),
	\end{align}
	where the sum is taken over all $s_1,\dots,s_j,t_1,\dots,t_k\in \{0,1,\dots,r-1\}$.
\end{lemma}

\begin{proof}
 	Let us proceed by induction on the degree. In degree $0$ we have
    \begin{align*}
        \mu(e_0) &= (\iota\otimes \iota)\circ \Delta(e_0) = (\iota\otimes \iota)(e_0\otimes e_0) = (0)\otimes (0),\\
        \nu(e_0) &= \Delta_{\AW}\circ \iota(e_0) = \Delta_{\AW}((0)) = (0)\otimes (0).
    \end{align*}
	Therefore
	\[
	K(e_0) = \eta \comp (\mu-\nu-K \comp \partial)(e_0) = 0.
	\]
	which agrees with \eqref{eq:homotopyK'1} for $i=0$. For the induction step, we distinguish two cases, depending on the parity of the degree:

 \medskip
    
    \noindent\emph{Case 1: Odd degree.} Suppose that we have proven that \eqref{eq:homotopyK'1} holds in some degree $2i$ with $i\geq 0$. Let us prove that \eqref{eq:homotopyK'2} holds in degree $2i+1$. By definition
	\[K(e_{2i+1}) = \eta \comp (\mu-\nu-K \comp \partial)(e_{2i+1}).\]
	Observe now that all summands in $\nu(e_{2i+1})$ start with $0$, and so do all summands in $\mu(e_{2i+1})$. As a consequence, $\eta \comp \nu(e_{2i+1}) = \eta \comp \mu(e_{2i+1}) = 0$, so we are left with
	\[K(e_{2i+1}) = -\eta \comp K \comp \partial(e_{2i+1}) = -\eta \comp K (\rho e_{2i}) + \eta \comp K (e_{2i}).
	\]
	The summand $\eta \comp K (e_{2i})$ vanishes, because all terms in $K(e_{2i})$ start with $0$. We now claim that $\eta \comp K (\rho e_{2i})$ equals the right-hand side of \eqref{eq:homotopyK'2}. By the induction hypothesis, we can compute $K(e_{2i})$ using \eqref{eq:homotopyK'1}. The claim then follows after noticing that $\eta(\rho\varphi^{\mathrm{even}}(s_1,\ldots,s_j;t_1,\ldots,t_k))$ equals
 \begin{align*}
      &\varphi^{\mathrm{odd}}(s_1+1,\ldots,s_j+1;t_1+1,\ldots,t_k+1).
\end{align*}

\medskip
    
    \noindent\emph{Case 2: Even degree.} Suppose that we have proven that \eqref{eq:homotopyK'2} holds in some degree $2i-1$ with $i>0$. Let us prove that \eqref{eq:homotopyK'1} holds in degree $2i$. By definition,
	\begin{equation}\label{eq:rota}
    \begin{split}
	    K(e_{2i}) &= \eta \comp (\mu-\nu-K \comp \partial)(e_{2i})\\
     &= \eta \comp\mu(e_{2i})- \eta\comp\nu(e_{2i})- \eta\comp K \comp \partial(e_{2i}).
	\end{split}
    \end{equation}
	Again, the middle summand vanishes
    \[\eta \comp \nu(e_{2i}) = 0
    \]
    because all terms of $\nu(e_{2i})$ start with $0$. Regarding the first summand, we have:
    \begin{equation*}%\label{eq:hK1}
    \begin{split}
        \eta\comp \mu(e_{2i}) &= \sum_{j+k=i+1}\sum_{0\leq s_1<t_1<r} (0,s_1,s_1+1,\ldots,s_j,s_j+1)\otimes (t_1,t_1+1,\ldots,t_k,t_k+1)
        \\
    &= \sum_{j+k=i+1}\theta_1(s_1,\dots,s_j;t_1)\cdot \varphi^{\mathrm{even}}(s_1,\dots,s_j;t_1,\dots,t_k).
	\end{split}
    \end{equation*}
	By the induction hypothesis, we can use \eqref{eq:homotopyK'2} to compute the third summand $-\eta \comp K \comp \partial(e_{2i})$, which equals the following:
	\begin{equation*}%\label{eq:hK2}
    \begin{split}
        \eta\circ N&\sum_{j+k=i}\sum_{q=1}^{j}\theta_q(s_1-1,\dots,s_j-1;t_1-1)\cdot \varphi^{\mathrm{odd}}(s_1,\dots,s_j;t_1,\dots,t_k) \\
     	&= \sum_{q=1}^{j}\sum_{j+k=i}\theta_q(s_1-1,\dots,s_j-1;t_1-1)\cdot \eta\circ N\varphi^{\mathrm{odd}}(s_1,\dots,s_j;t_1,\dots,t_k) \\
     &= \sum_{q=1}^{j}\sum_{j+k=i}\theta_{q+1}(s_0,s_1,\dots,s_{j};t_1)\cdot \varphi^{\mathrm{even}}(s_0,s_1,\dots,s_{j};t_1\dots,t_k)
\\ &= \sum_{j+k = i+1}\sum_{q=2}^{j}\theta_q(s_1,\dots,s_{j};t_1)\cdot \varphi^{\mathrm{even}}(s_1,\dots,s_{j};t_1,\dots,t_k).
	\end{split}
    \end{equation*}
	Adding the three summands of \eqref{eq:rota} we obtain identity \eqref{eq:homotopyK'1}.
\end{proof}

\begin{lemma} For any $(s_1,\ldots,s_j,t_1)$ in $\EE\cyc_r$ we have
    \[
    \alpha(s_1,\ldots,s_j,t_1) = \sum_{q=1}^j\theta_q(s_1,\ldots,s_j;t_1).
    \]
\end{lemma}
\begin{proof}
	% Observe first that if $s_{i+1} = s_i+1$ for some $i$, or $s_1 = 0$ (in the first case) or $s_1=1$ (in the second case) we get degenerate summands, so we do not treat these cases in the following computations.
	For an integer $m$, we will write $[m]$ for its class in $\Z/r\Z$. Let $s_1,\ldots,s_j,t_1$ be elements of $\{0,1,\ldots,r-1\}$ and write $(s_1',\dots,s'_j)$ for the only sequence of integers such that:
	\begin{enumerate}
		\item $[s'_i] = [s_i]$,
		\item $0< s_1'< r$,
		\item $1 < s_{i+1}' - s_i' \leq r$.
	\end{enumerate}
    Then we can rewrite the definition of $\theta_q$ as follows:
    \begin{align*}
        \theta_1(s_1,\ldots,s_j;t_1) &= 
        \begin{cases}
            1 & \text{if $\not\exists t\in [t_1], t\in [0,s_1']$} \\
            0 & \text{if $\exists t\in [t_1], t\in [0,s_1']$}
        \end{cases}
        \\
        \theta_q(s_1,\ldots,s_j;t_1) &= 
        \begin{cases}
            1 & \text{if $\not\exists t\in [t_1], t\in (s_{q-1}',s_q']$}\\
            0 & \text{if $\exists t\in [t_1], t\in (s_{q-1}',s_q']$} 
        \end{cases} 
        & q>1
    \end{align*}
%	Then $\theta_q(0,s_1,\dots,s_j;t_1,\dots,t_k)$ will vanish for $q>1$ if and only if the interval $(s_{q-1}',s_q']$ contains a representative of $[t_1]$, and will vanish for $q=1$ if and only if the interval $[0,s_1']$ contains a representative of $[t_1]$.
	Therefore the number of vanishing summands in $\sum_{q=1}^j\theta_q(s_1,\dots,s_j;t_1)$ equals the number $\ell$ of representatives of $[t_1]$ that lie in the interval $[0,s_j']$. If $s_j'=kr+s_j$ is the integer division of $s_j'$ by $r$, then 
    \[
        \ell = \begin{cases}
                k & \text{if $s_j<t_1$} \\
                k+1 & \text{if $s_j\geq t_1$}
        \end{cases}
    \]
    Now, $k$ equals the number of non-decreasing consecutive pairs in the sequence $(s_1,\ldots,s_j)$ and $\ell$ equals the number of non-decreasing consecutive pairs in the sequence $(s_1,\ldots,s_j,t_1)$.	
    
    Finally, $\sum_{q=1}^j\theta_q(s_1,\dots,s_j;t_1) = j-\ell$ equals the number of increasing consecutive pairs in the sequence $(s_1,\ldots,s_j,t_1)$, which is $\alpha(s_1,\dots,s_j,t_1)$.
\end{proof}