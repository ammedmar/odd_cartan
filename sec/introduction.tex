% !TEX root = ../odd_cartan.tex

\section{Introduction WIP} \label{s:introduction}

In \cite{steenrod1947products}, Steenrod introduced his celebrated squares
\begin{equation*}
	\Sq^k \colon \rH^\vee(X; \Ftwo) \to \rH^\vee(X; \Ftwo)
\end{equation*}
acting naturally on the mod-$2$ cohomology of a space $X$.
To do so, he presented explicit formulas defining cup-$i$ products on its complex of integral cochains
\[
\smallsmile_i \colon \cochains(X) \ot \cochains(X) \to \cochains(X)
\]
with $\Sq^{i-\bars{\alpha}}[\alpha] \defeq [\alpha \smallsmile_i \alpha]$.
The relationship between the Steenrod squares and the cup product in cohomology is encoded by the \textit{Cartan formula}
\[
\Sq^k\!\big([\alpha] [\beta]\big) = \sum_{i+j=k} \Sq^i [\alpha] \Sq^j[\beta].
\]
Since the algebra structure on cohomology is induced by the cup-$0$ product, it is natural to ask for a proof of the Cartan relation at the cochain level.
Specifically, an effective construction of a
%Motivated by its use in the study of symmetry protected topological phases, A. Kapustin asked for such proof.
which was given in \cite{medina2020cartan}.
The purpose of this work is to extend the constructions of that paper from the even to odd primes.
To explain our result further let us return to the 1950's.

Soon after Steenrod's construction, a viewpoint grounded on group homology was used to define Steenrod squares and their generalizations to odd primes, the Steenrod operations
\begin{align*}
	P_s \colon& \rH^\vee(X; \Fp) \to \rH^\vee(X; \Fp), \\
	\beta P_s \colon& \rH^\vee(X; \Fp) \to \rH^\vee(X; \Fp),
\end{align*}

The interaction between the Steenrod squares and the cup product in cohomology is constrained the \textit{Cartan formula}
\[
\Sq^k(x y) = \sum_{i+j=k} \Sq^i (x) \Sq^j(y)
\]
%\begin{enumerate}
%	\item $Sq^k$ is natural,
%	\item $Sq^0$ is the identity,
%	\item $Sq^k(x) = x^2$ for $x \in H^{-k}(X; \Ftwo)$,
%	\item $Sq^k(x) = 0$ for $x \in H^{-n}(X; \Ftwo)$ with $n<k$,
%	\item $Sq^k(xy) = \sum_{i+j=k} Sq^i (x) Sq^j(y)$.
%\end{enumerate}
Axiom 5., known as the \textbf{Cartan formula}, is the focus of this work.

To describe our viewpoint and present the contributions of this paper in context, let us revisit some of the history of Steenrod's construction.

In the late thirties, Alexander, Whitney, and \v{C}ech defined the ring structure on cohomology
\begin{equation} \label{equation: definition cup product}
	[\alpha] [\beta] = [\alpha \smallsmile_{0} \beta]
\end{equation}
using a cochain level construction
\begin{equation*}
	\smallsmile_0 \, : N^*(X; \mathbb Z) \otimes N^*(X; \mathbb Z) \to N^*(X; \mathbb Z)
\end{equation*}
dual to a choice of simplicial chain approximation to the diagonal inclusion. Here $N^*$ means the normalized cochains, which will be defined in the next section, and $[\alpha]$ denotes the cohomology class represented by a cocycle $\alpha$.

Steenrod then showed that $\smallsmile_0$ is commutative up to coherent homotopies by effectively constructing for $i \geq 0$ \textbf{cup-$i$ products}
\begin{equation*}
	\smallsmile_i\, : N^*(X; \mathbb Z) \otimes N^*(X; \mathbb Z) \to N^*(X; \mathbb Z)
\end{equation*}
enforcing its derived commutativity. (Axioms for these and connections with higher category theory can be found in \cite{medina2018axiomatic} and \cite{medina2019globular}). Then, with coefficients in $\Ftwo$, Steenrod defined
\begin{equation} \label{equation: definition steenrod squares}
	Sq^k\big([\alpha]\big) = [\alpha \smallsmile_{n-k} \alpha]
\end{equation}
where $[\alpha] \in H^{-n}(X; \Ftwo)$. This definition of the Steenrod squares makes the Cartan formula equivalent to
\begin{equation} \label{equation: Cartan 1}
	0 =
	\Big[ (\alpha \smallsmile_0 \beta) \smallsmile_i (\alpha \smallsmile_0 \beta)\ + \sum_{i=j+k} (\alpha \smallsmile_j \alpha) \smallsmile_0 (\beta \smallsmile_k \beta) \Big].
\end{equation}
The goal of this work is to effectively construct for any $i \geq 0$ and an arbitrary pair of cocycles $\alpha, \beta \in N^*(X; \Ftwo)$ a natural cochain $\zeta_i(\alpha, \beta)$ such that
\begin{equation} \label{equation: Cartan 2}
	\delta \zeta_i(\alpha, \beta) =
	(\alpha \smallsmile_0 \beta) \smallsmile_i (\alpha \smallsmile_0 \beta)\ + \sum_{i=j+k} (\alpha \smallsmile_j \alpha) \smallsmile_0 (\beta \smallsmile_k \beta).
\end{equation}
Following \cite{may70generalapproach}, we take a more general approach and in doing so we describe a non-necessarily effective construction for any \mbox{$E_\infty$-algebra}. We work over a fixed algebraic model $\mathcal E$ of the \mbox{$E_\infty$-operad }known as the Barratt-Eccles operad. This model, introduced by Berger-Fresse in \cite{berger04combinatorial}, is equipped with a diagonal map and the natural $\mathcal E$-algebra structure defined by these authors on the normalized cochains of simplicial sets is suitable for effective constructions.

This paper is part of an ongoing effort spearheaded by Greg Brumfiel and John Morgan to build effective models of classical homotopy-theoretic concepts, see for example \cite{brumfiel2016threedimensional, brumfiel2018quadratic, brumfiel2018fourdimensional}. Motivation for this specific project came from a question of physicist Anton Kapustin. See \cite{kapustin2017fermionic} for an instance where one of our formulas is used in the context of topological phases of matter.

Open-source software implementing the constructions of this paper as well as state-of-the-art algorithms for the computation of Steenrod squares and cup-$i$ products, as introduced in \cite{medina2018persistence}, can be found in the author's website.