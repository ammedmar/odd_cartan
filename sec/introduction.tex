% !TEX root = ../odd_cartan.tex

\section{Introduction} \label{s:introduction}

Steenrod introduced in \cite{steenrod1947products} his celebrated squares
\begin{equation*}
	\Sq^k \colon \rH^\vee(X; \Ftwo) \to \rH^\vee(X; \Ftwo)
\end{equation*}
on the mod 2 cohomology of simplicial sets via an effective construction of cup-$i$ products on their integral cochains
\[
\cp_i \colon \cochains(X) \ot \cochains(X) \to \cochains(X),
\]
which are degree $i$ products satisfying:
\[
\Sq^{i-\bars{\alpha}}[\alpha] \defeq [\alpha \cp_i \alpha],
\]
for $\alpha$ a mod $2$ cocycle and $[\alpha]$ its represented cohomology class.
The cup-$0$ product induces the cup product in cohomology
\[
[\alpha] \cdot [\beta] \defeq [\alpha \cp_0 \beta],
\]
whose interaction with the Steenrod squares is given by the Cartan formula:
\[
\Sq^k\!\big([\alpha] \cdot [\beta]\big) = \sum_{i+j=k} \Sq^i [\alpha] \cdot \Sq^j[\beta].
\]
This relation in cohomology implies the existence, for any pair of mod $2$ cocycles, of a coboundary $\delta\zeta_i(\alpha,\beta)$ satisfying:
\begin{equation}\label{e:lift cartan even}
	\delta\zeta_i(\alpha, \beta) =
	(\alpha \smallsmile_0 \beta) \cp_i (\alpha \cp_0 \beta) \ +
	\sum_{i=j+k} (\alpha \smallsmile_j \alpha) \cp_0 (\beta \cp_k \beta).
\end{equation}

Inspired by the application of cup-$i$ products in the theory of symmetry-protected topological phases (refer to \cite{kapustin2015cobordism, gaiotto2016spin, kapustin2017fermionic} for examples), Kapustin inquired about the construction of natural cochains $\zeta_i(\alpha, \beta)$ as in \eqref{e:lift cartan even}.
This was presented in \cite{medina2020cartan} and later employed in \cite{barkeshli2021classification} to classify fermionic $(2+1)$-dimensional invertible phases.

Not long after Steenrod's construction was introduced, a non-effective perspective based on group homology was developed, which allowed the definition not only of Steenrod squares but also of Steenrod operations
\begin{align*}
	P^s \colon& \rH^\vee(X; \Fp) \to \rH^\vee(X; \Fp), \\
	\beta P^s \colon& \rH^\vee(X; \Fp) \to \rH^\vee(X; \Fp),
\end{align*}
for odd primes.
The interaction of these with the cup product are also controlled by Cartan formulas:
\begin{align*}
	\rP_s\big([a][b]\big) =&
	\sum_{i+j=s} \rP_i[a] \, \rP_j[b], \\
	\beta\rP_s\big([a][b]\big) =&
	\sum_{i+j=s} \beta\rP_{i+1}[a] \, \rP_j[b] \ +\ (-1)^{\bars{a}} \rP_i[a] \, \beta\rP_{j+1}[b].
\end{align*}

By extending the concept of cup-$i$ products to other primes via the introduction of cup-$(p,i)$ products, an effective construction of Steenrod operations was introduced in \cite{medina2021may_st} and implemented in the specialized computer algebra system \texttt{ComCH} \cite{medina2021comch}.
The goal of this paper is to utilize these cup-$(p,i)$ products to lift the Cartan formulas to the cochain level, and to effectively construct natural coboundaries that generalize $\zeta_i(a,b)$ in \cref{e:lift cartan even} for odd primes.
Our method for constructing these odd Cartan coboundaries is solely based on the combinatorial structure of simplices, which enables a computer to carry out the process locally, focusing on individual simplices sequentially.
An actual implementation is left to future work.