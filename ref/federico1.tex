\textcolor{blue}{For Federico to deal with following the style of this referee replay.} Subsection 5.8. 

    \subitem The referee has not been able to check the rather difficult details of Section 7.2 and 7.3, purportedly giving a proof of Lemma 5.8.1 and related results. There are other ways to deal with the square on the left of the diagram labeled $K_3$, exploiting a retraction $\xi\colon \cC(r)\to \cW(r)$, but the authors closed formula for $K_3$ is a nice achievement if it can be made correct.

    \ar In Section 5.8 We have changed the notation used to define $K_3$ (Def. 5.8.1) and we have added two examples (5.8.2 and 5.8.3) to help the understanding. After the second example we have added a paragraph with some comments. In Section 7.2 we have changed the notation for $\theta_q$ as well and rewritten the proofs of Lemma 7.2.1 and Lemma 7.2.2 with the new notation (the argument is the same as before, but the presentation is different). 

    \subitem In the definition of $K_3$, it should be said that the definition on basis elements $e_i$ is extended equivariantly.

    \ar CAS.

    \subitem The referee tried to compare formulas for $K_3$ with Table 2 and ran into problems. In Table 2 it looks like minus signs are missing in the entries for odd $n=3,5$. Either that or something is amiss with the minus sign in front of the $\alpha$ in the last line on page $12$. But there were other problems. In an attempt to work with $e_3$ and $r=3$, things didn't work unles the entry for $K_3(e_2)$ is changed to $(0,1,2)\otimes (2,0)$, which actually seems to be what the formula gives. Maybe all the table entries are wrong.

    \ar We have double-checked the formulas, and found that the problems indicated by the referee were the only ones. Accordingly, we have changed the entry for $r=3,n=2$ for $(0,1,2)\otimes (2,0)$ and added the minus signs when $n=3,5$. Thanks for the careful check.

    \subitem The trouble here is that $K_3$ is very complicated, both in this section and in Section 7. Fussing around with sign changes and corrections is like a whak-a-mole. You 'fix' one thing and other things break down. Of course, the recursive homotopy method has to work, (although see correction paragraph 24 below), but proving a correct closed formula may be really tricky.

    \ar 



