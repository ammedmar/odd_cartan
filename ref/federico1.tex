    \subitem The referee has not been able to check the rather difficult details of Section 7.2 and 7.3, purportedly giving a proof of Lemma 5.8.1 and related results. There are other ways to deal with the square on the left of the diagram labeled $K_3$, exploiting a retraction $\xi\colon \cC(r)\to \cW(r)$, but the authors closed formula for $K_3$ is a nice achievement if it can be made correct.

	\ar To enhance the readability of this section, we have modified the notation used to define $K_3$ and included illustrative examples and comments. It now reads:

	\medskip\noindent\textbf{The homotopy $K_3$}

	\medskip\noindent Let us focus on the following part of Diagram (3):
	\[
	\begin{tikzcd}
		\cW(r)
		\arrow[r,"\iota"]
		\arrow[d,"\Delta"']
		\arrow[dr,phantom,"K_3^r"]
		& \cC(r)
		\arrow[d,"\Delta_{\AW}"] \\
		\cW(r)^{\ot 2}
		\arrow[r,"\iota^{\ot 2}"']
		& \cC(r)^{\ot 2}
	\end{tikzcd}
	\]
	We will construct a $g$-equivariant map $K_3^r$ making it homotopy commutative for each~$r$.

	Using the bijection $\cyc_r \cong \set{0,\dots,r-1}$, denote by $\alpha(s_1,\dots,s_\ell)$ the number of increasing consecutive pairs $s_i < s_{i+1}$ in a sequence $(s_1,\ldots,s_\ell)$ of elements of $\cyc_r$.
	For example, to compute $\alpha(2,4,1,3,1,0,4)$ we count the number of increasing consecutive pairs $(2,4),(1,3),(0,4)$, which is $3$, whereas $\alpha(3,1,0,2,4,3) = 2$ with increasing consecutive pairs $(0,2)$ and $(2,4)$.

	Let $K_3^r \colon \cW(r) \to \cC(r)^{\ot 2}$ be defined by equivariantly extending the following assignment on basis elements:
	\[
	\begin{split}
		K_3^r(e_{2i}) &= \sum \, \alpha(s_1,\ldots,s_j,t_1)\cdot\varphi^{\mathrm{even}}(s_1,\dots,s_j;t_1,\dots,t_k), \\
		K_3^r(e_{2i+1}) &= -\sum \, \alpha(s_1-1,\ldots,s_j-1,t_1-1)\cdot \varphi^{\mathrm{odd}}(s_1,\dots,s_j;t_1,\dots,t_k),
	\end{split}
	\]
	where the sums are taken over all $s_1,\dots,s_j,t_1,\dots,t_k \in \cyc_r$ with $j+k = i+1$ and
	\begin{align*}
		\varphi^{\mathrm{even}}(s_1,\dots,s_j&;t_1,\dots,t_k) \\ &=
		(0,s_1,s_1+1,\dots,s_j,s_j+1) \ot (t_1,t_1+1,\dots,t_k,t_k+1), \\
		\varphi^{\mathrm{odd}}(s_1,\dots,s_j&;t_1,\dots,t_k) \\ &=
		(0,1,s_1,s_1+1,\dots,s_j,s_j+1) \ot (t_1,t_1+1,\dots,t_k,t_k+1).
	\end{align*}
	For example, the summand indexed by $(s_1,\ldots,s_6;t_1,t_2,t_3) = (2,4,1,3,1,0;4,2,1)$ in $K_3^5(e_{16})$ is
	\[
	3 \cdot (0,2,3,4,0,1,2,3,4,1,2,0,1) \otimes (4,0,2,3,1,2),
	\]
	whereas the summand indexed by $(s_1,\ldots,s_5;t_1,t_2) =(4,2,1,3,0;4,1)$ in $K_3^5(e_{13})$ is
	\[
	-2 \cdot (0,1,4,0,2,3,1,2,3,4,0,1) \otimes (4,0,1,2).
	\]
	Please consult Table 2 for a the first few non-trivial examples.

	We remark that the numbers $\alpha(s_1,\ldots,s_\ell)$ are always non-negative, and therefore all non-trivial summands in $K_3^r(e_{2i})$ have positive coefficients and all non-trivial summands in $K_3^r(e_{2i+1})$ have negative coefficients.
	Observe also that the summands in $K_3^r(e_{2i})$ are in bijection with the summands in $K_3^r(e_{2i+1})$ and the bijection changes the sign of the coefficient of each summand.

	\begin{lemma}\label{l:K3}
		$K_3$ is a $\cyc$-equivariant homotopy from $\Delta_{\AW} \comp \iota$ to $(\iota \ot \iota) \comp \Delta$.
	\end{lemma}

	\begin{proof}
		We devote Section 7 to the proof of this statement.
	\end{proof}

	\ar Additionally, in Section 7.2, we revised the notation for $\theta_q$ and restructured the proofs of Lemma 7.2.1 and Lemma 7.2.2 using this new notation.
	The underlying arguments remain unchanged, but the presentation has been improved for clarity.

    \subitem In the definition of $K_3$, it should be said that the definition on basis elements $e_i$ is extended equivariantly.

    \ar CAS.

    \subitem The referee tried to compare formulas for $K_3$ with Table 2 and ran into problems. In Table 2 it looks like minus signs are missing in the entries for odd $n=3,5$. Either that or something is amiss with the minus sign in front of the $\alpha$ in the last line on page $12$. But there were other problems. In an attempt to work with $e_3$ and $r=3$, things didn't work unless the entry for $K_3(e_2)$ is changed to $(0,1,2)\otimes (2,0)$, which actually seems to be what the formula gives.
    Maybe all the table entries are wrong.

    \ar We have double-checked the formulas, and found that the problems indicated by the referee were the only ones. Accordingly, we have changed the entry for $r=3,n=2$ for $(0,1,2) \otimes (2,0)$ and added the minus signs when $n=3,5$.
    Thanks for the careful check.

    \subitem The trouble here is that $K_3$ is very complicated, both in this section and in Section 7. Fussing around with sign changes and corrections is like a whak-a-mole. You 'fix' one thing and other things break down. Of course, the recursive homotopy method has to work, (although see correction paragraph 24 below), but proving a correct closed formula may be really tricky.

    \ar We hope the new presentation makes our constructions less opaque.