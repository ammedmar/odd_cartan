\textcolor{blue}{For Federico to deal with following the style of this referee replay.} Subsection 7.1

\subitem Add to line 3. ``...following diagram commute, where $\cyc$ acts diagonally on the tensor products.''

\ar CAS.

\subitem Line 7. The definition of $\xi$ and $\eta$ do not work to define a contraction. To begin with, if $\xi$ made sense, $Image(\xi) = \mathbb{Z}\{\rho^0\}\otimes \mathbb{Z}[\cyc_r]$, all in degree $0$, which does not even have the same homology as $\cC(r)\otimes \cC(r)$. If one tries to check the homotopy condition $d\eta+\eta d = \id-\xi$, it will fail.

There are two possible choices for $D'=Image(\xi)$. One is $Image(\xi) = \mathbb{Z}\{\rho^0\}\otimes \cC(r)$. Then, writing the cyclic group elements as integers $s_i$ from $0$ to $p-1$ and $\cW$ simplices as tuples $s = (s_0,\ldots,s_k)$, one gets $\xi(s\otimes t) = (0)\otimes t$ if $deg(s) = 0$ and $0$ otherwise. For this choice, the authors $\eta$ does work. In fact, one can begin with their $\eta$ and simply compute $d\eta+\eta d$ to find the correct $\xi$. 

The other choice for $\xi$ is $\xi(s\otimes t) = (0)\otimes (0)$ if $deg(s\otimes t)=0$ and $0$ otherwise. With image $\mathbb{Z}\{\rho^0\}\otimes \mathbb{Z}\{\rho^0\}$. For this choice a somewhat different $\eta$ must be used. Based on what comes later, the referee assumes the first choice is intended. It is the better choice.

\ar The first choice was the one intended. Thus, we have changed the condition $deg(s\otimes t) = 0$ for the condition $deg(s) = 0$ in the definition of $\xi$.