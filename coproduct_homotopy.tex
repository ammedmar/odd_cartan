Recall that
\begin{align*}
\Delta_1(e_k) &= \sum_{i+j=k} e_i\otimes e_j \\
\Delta_2(\sigma_0,\ldots,\sigma_k) &= \sum_{i} (\sigma_0,\ldots,\sigma_i)\otimes (\sigma_i,\ldots,\sigma_k).
\end{align*}

The following diagram
\[\xymatrix{
W(p)\ar[d]^{\Delta_1} \ar[r]^{\Psi} & \cE(r)\ar[d]^{\Delta_2} \\
W(p)\otimes W(p)\ar[r]^{\Psi\otimes\Psi} \cE(r)\otimes \cE(r)
}
\]
commutes up to a $C_p$-equivariant homotopy. Let us define one such homotopy, which is a homomorphism
\[H\colon W(p)\lra \cE(r)\otimes \cE(r)\]
of degree $1$ such that 
\[\partial H + H\partial = \sum_{i} \sum_{i} \Delta_2\Psi(e_i)-\Psi(e_i)\otimes \Psi(e_{k-i})
\]
This difference consists on those summands $a\otimes b$ of $\sum_i\Delta_2\Psi(e_i)$ such that the last element of $a$ is not $e$.


Consider first the homotopy
\[h\colon \cE(r)\to \cE(r)\]
that sends a tuple $(\sigma_0,\ldots,\sigma_k)$ to the tuple $(-1)^{k+1}(\sigma_0\ldots,\sigma_k,e)$, where $e\in \Sigma_r$ is the identity.


Here it is:
\[H(e_k) = \sum_{i+j=k-1} (h\psi(e_i))\otimes \psi(e_j)\]
Now, $\partial H(e_k) = A + B$ where $A$ consists of the summands that result from a summand $a\otimes b$ of $H(e_k)$ by removing the last element of the tuple $a$ (this element is always $e$), and $B$ is the sum of the remaining summands. Then a closer look reveals that 
\begin{align*}
A &=  \sum_{i} \Psi(e_i)\otimes \Psi(e_{k-i}) - \sum_{i} \Delta_2\Psi(e_i) \\
B &= -H\partial(e_k). %(yes quite, the signs seem to match completely)
\end{align*}
